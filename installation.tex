\subsection{Installation}

The interior of the cryostat will be prepared prior to the installation of the TPC.  A series of I-beam support rails will be suspended below the top surface of the cryostat membrane by a series of hangers.  These hangers will be structurally supported by an independent structure above the cryostat.  Decoupling the TPC support from the cryostat structure eliminates the movement of the TPC with the flexure of the cryostat structure from the filling and internal pressure changes of the Argon inside.  The hangers will pass through the top of the cryostat to the independent structure inside a bellows type feedthrough.  These feedthroughs need to be designed to minimize the heat flow into the cryogenic volume.  For the CPAs, the support rails and hangers need to be electrically isolated due to high voltage concerns.  To preserve the ability to reverse the order of the TPC components, all of the support points will be designed to the maximum set of requirements regarding loads and clearances.  

There will be a series of feedthrough flanges located along each of the support rails.  These will be cryogenic flanges where the services for the TPC components can pass through the top of the cryostat.  It is foreseen that each CPA row will require one feedthrough for the high voltage probe to bring in the drift voltage.  The drift voltage is 500 V/cm.  For a drift distance of 3.6~m and 2.5~m, the probe voltages will be 180~kV and 125~kV respectively.  There will be one service feedthrough for each of the APAs.  These feedthroughs will include high speed data connections, bias voltages for the wire planes, control and power for the cold electronics.  

The main TPC components will be installed through large hatches in the top of the cryostat.  This is 
similar to the installation method intended for the detector at the DUNE far site.  These hatches will have an 
aperture approximately 2.0 m wide and 3.5 m long.  Each APA and CPA panel will be carefully tested after transport into the clean area and before installation into the cryostat. Immediately after a panel is installed it will be rechecked. The serial installation of the APAs along the rails means that removing and replacing one of the early panels in the row after others are installed would be very costly in effort and time. Therefore, to minimize the risk of damage, as much work around already installed panels as possible will be completed before proceeding with further panels.
The installation sequence is planned to proceed as follows:

\begin{enumerate}
\item Install the monorail or crane in the staging area outside the cryostat, near the equipment hatch.
\item Install the relay racks on the top of the cryostat and load with the DAQ and power supply crates.
\item Dress cables from the DAQ on the top of the cryostat to remote racks.
\item Construct the clean-room enclosure outside the cryostat hatch.
\item Install the raised-panel floor inside the cryostat. 
\item Insert and assemble the stair tower and scaffolding in the cryostat.
\item Install the staging platform at the hatch entrance into the cryostat.
\item Install protection on (or remove) existing cryogenics instrumentation in the cryostat.
\item Install the cryostat feedthroughs and dress cables inside the cryostat along the support beams.
\item Install TPC panels:
   \begin{enumerate}
   \item Install the CPA panels supported from the center support rail.  These will be installed from the floor of the cryostat.  Access to the top edge will be required by scaffolding.  
   \item Install and connect HV probe for the CPAs.
   \item Perform electrical tests on the connectivity of the probe to the CPAs.
   \item Install first end wall of vertical field cage at the non-access end of the cryostat.  These will be installed from the floor of the cryostat.  Scaffolding will be needed to install the supporting structure and then attach the panels to the structure.  
   \item Test the inner connections of the field cage panels.
   \item Install the first APA and connect to the far end field cage support.
   \item Connect power and signal cables.  This will require scaffolding to access the top edge of the APA.
   \item Test each APA wire for expected electronics noise. Spot-check electronics noise while cryogenics equipment is operating.
   \item Install the upper field cage panels for the first APA between the APA and CPAs  This will require scaffolding to access the upper edge of the APA, CPA and field cage structure.
   \item Perform electrical tests on upper field cage panels.
   \item Repeat steps (f) through (j) for the next five APAs.
   \item Install the lower field cage panels between the APAs and CPAs.  Start at the far end away from the access hatch and work towards the hatch. 
   \item Perform electrical test on lower field cage panels and the entire loop around the TPC.
   \item Remove temporary floor sections as the TPC installation progresses.
   \item Install sections of argon-distribution piping as the TPC installation progresses.
   \item Install the final end wall of vertical field cage at the access end of the cryostat.  These will be installed from the floor of the cryostat.  Scaffolding will be needed to install the supporting structure and then attach the panels to the structure.
   \end{enumerate}
\item Remove movable scaffold and stair towers.
\item Temporarily seal the cryostat and test all channels for expected electronics noise.
\item Seal the access hatch.
\item Perform final test of all channels for expected electronics noise.
\end{enumerate}
 
In general, APA panels will be installed in order starting with the panel furthest from the hatch side of the cryostat and progressing back towards the hatch. The upper field cage will be installed in stages as the installation of APAs and CPA progresses.  After the APAs are attached to the support rods the electrical connections will be made to electrical cables that were already dressed to the support beams and electrical testing will begin. Periodic electrical testing will continue to assure that nothing gets  damaged during the additional work around the installed APAs.  

The TPC installation will be performed in three stages, each in a separate location. First, in the clean room vestibule, a crew will move the APA and CPA panels from storage racks, rotate to the vertical position and move them into the cryostat. Secondly, in the panel-staging area immediately below the equipment hatch of the cryostat, 
% CHECK THIS! I TRIED TO INTERPRET RUSS'S COMMENTS. -- ANNE
a second crew will transfer the panels from the crane to the staging platform, where the crew inside the cryostat will connect the panels to the rails within the cryostat. A third crew will reposition the movable scaffolding and use the scaffold to make the mechanical and electrical connections at the top for each APA and CPA as they are moved into position.  

The requirements for alignment and survey of the TPC are under development. Since there are many cosmic rays in the surface detector and beam events, significant corrections can be made for any misalignment of the TPC. The current plan includes using a laser guide or optical transit and the adjustment features of the support rods for the TPC to align the top edges of the APAs in the TPC to be straight, level and parallel within a few millimeters. The alignment of the TPC in other dimensions will depend on the internal connecting features of the TPC.  The timing of the survey will depend on understanding when during the installation process the hanging TPC elements are in a dimensionally stable state. The required accuracy of the survey is not expected to be more precise than a few millimeters.  




