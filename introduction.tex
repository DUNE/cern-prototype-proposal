
{\it 
This document describes the motivation and technical details of the proposed detector. 
In section~\ref{intro} we give a brief overview of the goals for the detector and beam test.
Section~\ref{detbeamtest} describes the scientific motivation and the planned program of measurements. The layout of the single phase liquid argon detector and technical details of the detector components are specified in section~\ref{singlephasedet}. In section~\ref{cryo} we provide information on the proposed cryostat which houses the detector as well as on the required cryogenics system. Section~\ref{calibration} introduces our approach to calibrating the detector and the necessary tools to do so. We describe the requirements for the charged particle beam and a preliminary run plan in section~\ref{testbeamreq}.
In section~\ref{computing} we estimate computing and data handling needs before we describe the installation procedure for the detector
and the interface to the CERN Neutrino Platform in section~\ref{nuplatform}. We present schedule and organization for the proposed activities in section~\ref{organ} before we conclude in section~\ref{summary}.}

\label{intro}


\subsection{Goals for the prototype detector and beam test}

The CERN single-phase prototype detector is a crucial milestone for the DUNE experiment that will inform the construction and operation of the first 10~kt DUNE far detector module.  The prototype detector and beam test serve two principal functions. 

The first goal is 
to measure and study the response of the detector to charged particles of different types and energies. 
Results from these measurements serve to assess systematic detector uncertainties and
validate MC simulations. They also enable validation and tuning of event reconstruction algorithms and particle identification tools.

The second goal is 
to serve as an engineering prototype to validate the performance of all detector components,
establish and commission detector production sites and associated quality assurance procedures and to validate the installation procedure. 
%
In order to mitigate the risks associated with extrapolating 
from a detector with a few TPC modules to one which contains several hundred 
 it is essential to benchmark the operation of full-scale detector elements
and perform measurements in a well characterized charged particle beam.  

Numerous basic detector performance parameters can be established with cosmic ray muons and these results are a critical input to finalizing the production procedures of the first 10~kt DUNE far detector components. Potentially problematic components can be identified and improvements and optimizations of the detector design for future far detector modules can be developed. 
%
In particular, the following checks are anticipated:
\begin{enumerate}
 \item characterizing the performance of full scale TPC module
 \item studying the performance of the photon detection system
 \item testing and evaluating the performance of detector calibration tools
  \item verifying the functionality of cold TPC electronics under LAr cryogenic conditions
  \item performing a full-scale structural test under LAr cryogenic conditions
  \item verifying argon contamination levels and associated mitigation procedures
  \item developing and testing installation procedures for full-scale detector components
  \item identifying flaws and inefficiencies in the manufacturing process
\end{enumerate}

Experience gained from construction, installation and commissioning of the CERN single-phase prototype detector 
as well as performance tests with cosmic ray data are expected to lead to a detector optimization of equivalent phases 
for the DUNE far detector. 

The use of a well-defined charged particle test beam will significantly enhance our understanding of the detector performance beyond the criteria already mentioned.
The beam measurements will serve as a calibration data set to tune the Monte Carlo simulations and serve as a reference data set for measurements of the future DUNE detector. 
%
In order to make such precise measurements, the detector will need to accurately identify and measure the energy of the particles produced in the neutrino interaction with Argon which will range from hundreds of MeV to several GeV.
%
More specifically, the goals of the prototype detector and beam test measurements include the use of a charged particle beam to:
\begin{enumerate}
\item measure the detector calorimetric response for
\begin{enumerate}
	\item hadronic particles
	\item electromagnetic showers
\end{enumerate}
\item study e/$\gamma$-separation capabilities
\item measure event reconstruction efficiencies as a function of energy and particle type
\item measure performance of particle identification algorithms as function of energy and for real detector conditions
\item assess single particle track calibration and reconstruction
%		-- characterize performance of algorithms
\item validate accuracy of Monte Carlo simulations for relevant energy ranges as well as directions with respect to the wire-plane geometry

%  \item secondary hadron interactions in detector
\item study other topics with the collected data sets
 \begin{enumerate}
    \item pion interaction kinematics and cross sections
    \item kaon interaction cross section to remove proton decay backgrounds
    \item muon capture for charge identification
 \end{enumerate}
\end{enumerate}
%
The CERN charged particle beam lines provide an opportunity to perform this crucial test of the 
proposed single-phase LAr TPC.
%
%We estimate the desired minimum integrated particle counts as a function of charged particle species and momentum
% is converted into a run plan based on realistic beam composition, particle energies and efficiency information. 
%
Additional follow-up measurements with modified detector components form a potential future extension 
to the proposed program.\\


\subsection{Single-phase LAr detector}

%DUNE plans to place modular LAr TPCs with a combined total fiducal mass of at least  40 kton in the underground facility at Homestake to 
%detect neutrinos and search for proton decay.
%The first 10~kton LAr TPC module is planned to be constructed underground on the time scale of 2022.

The basic components of the liquid argon detector include a time projection chamber (TPC) housed in a cryostat which is connected to  
a cryogenics system. 
%
The cryostat contains the liquid argon target material and the cryogenic system keeps the liquid argon at a temperature of 89K, and maintains the required purity through a pump and filter system. A uniform electric field is created within the TPC volume between cathode planes and anode wire planes. Charged particles passing through the TPC release ionization electrons that drift to the anode wires. The bias voltage is set on the anode plane wires so that ionization electrons drift between the first several (induction) planes and is collected on the last (collection) plane. Readout electronics amplify and continuously digitize the induced waveforms on the sense wires at several MHz, and transmit these data to the DAQ system for analysis. The wire planes are oriented at different angles allowing a 3D reconstruction of the particle trajectories. In addition to these basic components, a photon detection system
is included in the design to provide timing information for events not associated with the neutrino beam, enabling the study of proton decay and be more sensitive to galactic supernova neutrinos.

Our LAr detector design is characterized by a modular approach in which the LAr volume in the cryostat is instrumented with a number of identical anode wire plane assemblies (APA) and associated cathode plane assemblies (CPA). To a large extent, scaling from detector volumes containing a few of such modules  to several hundred should be feasible with low and predictable risk.\\









%






