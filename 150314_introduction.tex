This document is a technical report for the Expression of Interest $<$reference$>$ submitted to the CERN SPSC in October 2014.  In this report, we detail the 
proposal to test full-scale detector elements for a single-phase liquid argon (LAr) TPC based on the former LBNE design which is a potential viable technology 
for use as a far detector for the experiment that will be at the Long Baseline Neutrino Facility (LBNF).  To mitigate the risks associated with extrapolating 
small scale versions of the single-phase LAr TPC technology to a full-scale detector element, it is essential to benchmark the operation of a full-scale 
detector elements in a particle beam.  The beam facility at the CERN Neutrino Platform (cite) provides an opportunity to perform this crucial test of the 
proposed single-phase LAr TPC and inform the decision regarding phased implementation of the far detector for LBNF.


\subsection{Physics goals of LBNF}
An international collaboration is forming that will utilize a proposed 1.2 MW proton beam at FNAL to create a neutrino beam that will be directed at the 
Homestake Mine in the USA.  The proposal is to place by the year 2024 $<$mention 2021 date for 10kton$>$ a 40 kton LAr TPC in the underground facility at 
Homestake to observe the oscillation of $\nu_\mu \rightarrow \nu_e$ and $\bar{\nu}_\mu \rightarrow \bar{\nu}_e$ that occur near a neutrino energy of 3 GeV 
and thereby make precision measurements of $\delta_{CP}$, $\delta_{13}$, and $\delta_{23}$ and determining the neutrino mass hierarchy.  In order to make such 
precise measurements, the detector will need to accurately identity and measure the energy of the particles produced in the neutrino interaction with Argon 
which will range from a few GeV to hundreds of MeV.       


\subsection{Single-phase LAr TPC Prototype}
The LBNE-style single-phase LAr TPC is design to be scalable using module units that consist of a factory-built anode plane, a cathode a plane, and a field 
cage with the anode plane assemblies measuring 6.0m  (h)  $\times$  2.3m  (w) $\times$  0.09m (d) and a drift length of 2.5m.  The electronics will be mounted 
in the cryostat on the wire planes to reduce the number of cables that penetrate the cryostat and optimize the electronic signal to noise.  Event timing will 
be determined through the use of scintillation photon detectors that will use light collection paddles.  With this modular design as shown in $<$insert figure 
of LBNE detector to shown modular design$>$ , it will be possible to achieve the goal of 40 kton detector for the LBNF far site.

Some features of this single-phase design will be tested using smaller scale prototypes such as the ongoing tests with a 35ton unit at Fermilab and the 5ton 
CAPTAIN LAr TPC.  However, none of these smaller prototypes uses full scale detector elements or are large enough to fully contain particles.  Considering that 
the detector at the LBNF far site will be about a factor of 50 larger in scale than the ICARUS detector which is currently the largest LAr TPC detector built 
to date, it is essential to validate the full-scale detector elements in a particle beam at the CERN Neutrino Platform.    

In order to fully contain particles within the energy range of interest and provide space for detector services, the cryostat will need to have a minimum inner 
dimensions of 8.4m (h) $\times$ 7.3m (w) $\times$ 9.5m (d).  It is anticipated that the CERN Neutrino Platform will facilitate the design and construction of 
the cryostat and that this effort will common area that will motivate collaboration with the WA105 team.

\subsection{Goals for the prototype run}
The goals of the prototype run include:
\begin{itemize}
\item Use of particle beam to assess
 \subitem tracking and calorimetric response
 \subitem reconstruction algorithm
 \subitem Monte Carlo simulation
 \subitem Secondary hadron interactions in detector
\item Verifying argon contamination mitigation
\item Full-scale structural test under LAr cryogenic conditions
\item Verfication of TPC Electrics under LAr cryogenic conditions
\item Study light levels of the photon detection system
\item Developing installation procedures for full-scale electronics

\end{itemize}

