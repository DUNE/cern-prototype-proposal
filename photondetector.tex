



The photon detection system (PDS) of the DUNE far detector will utilize liquid argon scintillation light to trigger on non-beam events and to
determine the prompt event time primarily for non-beam events but also for beam events. 
Timing information will be useful in determining the t$_0$ of cosmic rays, including those which 
overlap with beam events and events from radiological decays. This timing information allows proper spatial event reconstruction, including 
the reconstruction and rejection of background events.\\
%
While the TPC will have spatial resolution that is far superior to a photon detection system, there is no intrinsic precise determination of the event time and the drift time for TPC events can be up to milliseconds. The photon detection system can determine the start of an event occurring in the TPC volume (or entering the volume) to about 6 ns. For beam-triggered events the performance of the photon detection timing information can be crosschecked and evaluated during the detector beam test. In addition to providing trigger functionality, the PDS may be able to improve the event energy reconstruction by providing drift length dependent corrections to energy loss of the drifting electrons.
In the absence of an external electric field, a charged particle passing through liquid argon will produce about 44,000 photons ($\lambda$ = 128~nm) per MeV of deposited energy. 
At higher electric drift fields the number of photons will be smaller due to reduced recombination, but at 500 V/cm the yield is about 20,000 photons per MeV. Roughly one-third of the photons are prompt, 2-6~ns, and two-thirds are generated after a delay of 1.1-1.6~$\mu$s. LAr is highly transparent to the 128~nm VUV photons with a Rayleigh scattering length and absorption length of 66~cm \cite{rayleigh} and $>$200~cm \cite{absorption} respectively. The relatively large light yield makes the scintillation process an excellent candidate for triggering and  determination of t$_0$ for non-beam related events. Detection of the scintillation light may also be helpful in background rejection and possibly
provide improvements for energy reconstruction.

Several prototypes of photon detection systems for single phase liquid argon detectors have been developed by the former LBNE photon detector group over the past few years. There are currently three prototypes under consideration for use in the first module of the DUNE far detector, a baseline design along with two alternate designs. A decision on the design to be deployed in the CERN test will be made in late 2015. The CERN single phase prototype detector provides the first full scale test of the photon detectors which will be fully integrated into a 
full scale TPC anode plane assembly. 

The present reference design for the photon detection system is based on acrylic bars that are 200~cm long and 7.6~cm wide, which are coated with a layer of tetraphenyl-butadiene (TPB). The wavelength shifter converts incoming VUV (128 nm) scintillation photons
%   with a conversion efficiency of ~50\% \cite{conversion-eff}
to longer wavelength photons which are characterized by an emission spectrum with a peak wavelength of 430~nm.
About 50\% of the converted photons will be emitted into the bar.
  A fraction of the wavelength-shifted optical photons are internally reflected to the bar's end where they are detected by SiPMs whose QE curve is well matched to the 430 nm wavelength-shifted photons. All PD prototypes are currently using SensL MicroFB-6K-35-SMT 6 mm $\times$ 6 mm devices \cite{sensl}. 

A full 6 m long APA will be divided into 5 bays with 2 PD modules (paddles) instrumenting each bay. The paddles will be inserted into the frames after the TPC wires have been strung allowing  final assembly at the CERN test location. Two alternative designs are also under consideration. 


One alternate design attempts to increase the geometrical acceptance of the photon detectors by using large acrylic TPB coated plates with imbedded WLS fibers for readout. In this design the number of required SiPMs and readout channels per unit detector area covered with photon detection panels would be significantly reduced to keep the overall cost for the photon detection system at or below the present design while increasing the geometrical acceptance at the same time. The prototype consists of a TPB-coated acrylic panel embedded with a S-shaped wavelength shifting (WLS) fiber. The fiber is read out by two SiPMs, coupled to either end of the fiber, and serves to transport the light over long distances with minimal attenuation. The double-ended fiber readout has the added benefit to provide some position dependence to the light generation along the panel by comparing relative signal sizes and arrival times in the two SiPMs. 



The third design under consideration was motivated by increasing the attenuation length of the PD paddles and allowing collection of 400 nm photons coming from anywhere in the active volume of the TPC.  The fiber-bundle design is based on a thin TPB coated acrylic radiator located in front of a close packed array of WLS fibers. This concept is designed so that roughly half of the photons converted in the radiator are incident on the bundle of fibers, the wavelength shifting fibers are Y11 UV/blue with a 4\% capture probability. The fibers are then read out using SiPMs at one end. The Y11  Kuraray fibers have mean absorption and emission wavelengths of about 440 nm and 480 nm respectively.  The attenuation length of the Y11 fibers is given to be greater than 3.5 m at the mean emission wavelength, which will allow production of full-scale (2 m length) photon detector paddles.


The PD system tested at the CERN neutrino platform will be based on technology selected later in 2015. The technology selection process will be based on a series of tests planned for 6 months utilizing large research cryostats at Fermilab and Colorado State University. The primary metric used for comparison between the three technologies will be photon yield per unit cost. In addition to this metric, PD threshold and reliability will also serve as inputs to the final decision. A technical panel will be assembled to make an unbiased decision. 

