



The ELBNF far detector will utilize liquid argon scintillation light to determine the prompt event time of beam-driven and non-beam events. While the TPC will have a far superior spatial resolution compared to a photon detection system, the drift time for TPC events is on the order of milliseconds. The beam clock will give much better timing resolution than this but a photon detection system can determine the start of an event occurring in the TPC volume (or entering the volume) to about 6~ns, which will be useful in determining the t$_0$ of cosmic ray events, events from radiological decays, and corrections to energy loss of the drifting electrons which in turn improves the particle identification capability of the TPC

A charged particle passing through liquid argon will produce about 40,000 128~nm photons per MeV of deposited energy in the absence of electric fields. At higher electric fields this will be reduced due to reduced recombination, but at 500 V/cm the yield is still about 20,000 photons per MeV. Roughly 1/3 of the photons are prompt 2-6~ns and 2/3 are generated with a delay of 1100-1600~ns. LAr is transparent to the 128~nm VUV photons. At this wavelength, however, the Rayleigh scattering cross-section is large leading to a relatively short Rayleigh scattering length of the order of 55 to 63 ~cm and an absorption length around 200 cm. The relatively large light yield makes the scintillation process an excellent candidate for determination of t$_0$ for non-beam related events. Detection of the scintillation light may also be helpful in background rejection and improving the particle identification.

Several prototypes of photon detection systems have been developed by the LBNE, now ELBNF, photon detector group over the past few years. There are currently three prototypes under consideration for use in the ELBNF far detector, a baseline design along with two alternate designs. A decision on the design to be deployed in the CERN test will be made in late 2015. The CERN neutrino platform ELBNF test would provide the first full scale test of the ELBNF photon detector completely integrated into a full scale TPC anode plane assembly. 

The present reference design for the photon detection system is based on acrylic bars that are 200 cm long and 7.63 cm wide, which are coated with a layer of of the wavelength shifter (WLS)tetraphenyl-butadiene (TPB). The wavelength shifter converts VUV (128 nm) scintillation photons striking it to 430 nm photons inside the bar, with an efficiency of ~50\% of converting a VUV to an optical photon.  A fraction of the wavelength-shifted optical photons are propagated through internal reflection to the bar's end where they are detected by SiPMs whose quantum efficiency (QE) is well matched to the 430 nm wavelength-shifted photons. All PD prototypes are currently using SensL MicroFB-6K-35-SMT 6 mm by 6 mm devices. 

A full 6 m long APA will be divided into 5 bays with 2 PD modules (paddles) instrumenting each bay. The paddles will be inserted into the frames after the TPC wires have been wrapped around the frames allowing  final assembly at the CERN test location. Two alternative designs are also under consideration. 


One alternate design targets increasing the geometrical acceptance of the photon detectors by using large acrylic TPB coated plates with imbedded WLS fibers for readout. In this design the number of required SiPMs and readout channels per unit detector area covered with photon detection panels would be significantly reduced to keep the overall cost for the photon detection system at or below the present design while at the same time increasing the geometrical acceptance. This prototype consists of a TPB-coated acrylic panel embedded with an S-shaped wavelength shifting fiber. The fiber is read out by two SiPMs, which are coupled to either end of the fiber and serves to transport the light over long distances with minimal attenuation. The double-ended fiber readout has the added benefit to provide some position dependence to the light generation along the panel by comparing relative signal sizes and arrival times in the two SiPMs. 



The third design under consideration was motivated by increasing the attenuation length of the PD paddles and allowing collection of 400 nm photons coming from anywhere in the active volume of the TPC.  The fiber-bundle design is based on a thin TPB coated acrylic radiator located in front of a close packed array of WLS fibers. This concept is designed so that roughly half of the photons converted in the radiator are incident on the bundle of fibers, the wavelength shifting fibers are Y11 UV/blue with a 4\% capture probability. The fibers are then read out using SiPMs at one end. The Y11  Kuraray fibers have mean absorption and emission wavelengths of about 440 nm and 480 nm respectively.  The attenuation length of the Y11 fibers is given to be greater than 3.5 m at the mean emission wavelength, which will allow production of full-scale (2 m length) photon detector paddles.

The PD system tested at the CERN neutrino platform will be based on the technology selected later this year. The technology selection process will be based on a series of tests planned for the next 6 months utilizing large research cryostats at Fermilab and Colorado State University. The primary metric used for comparison between the three technologies will be photon yield per unit cost. In addition to this metric photon detection threshold and reliability will also serve as inputs to the final decision. A technical panel will be assembled to make an unbiased decision. 

Once the technology has been chosen the PD group will focus on optimizing the selected design with the goal of procurement and assembly taking place in late FY 2016 and early FY 2016. The photon detector paddles will then be tested and shipped to CERN in early FY 2017 for installation into the APAs in late FY 2017 in preparation for installation into the test cryostat and operation in 2018. 
