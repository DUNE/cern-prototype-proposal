%%%%%%%%%%%%%%%%%%%%%%%%%%%%%%%%%%%%%%%%%%%%%%%%%%%%%%%%%%%%%%%%%%%%%%%%%%%%%%%
%%%%%%% 			CERN prototype proposal 2014/2015  					%%%%%%%%%%%%%%%%%%%%%
%%%%%%%%%%%%%%%%%%%%%%%%%%%%%%%%%%%%%%%%%%%%%%%%%%%%%%%%%%%%%%%%%%%%%%%%%%%%%%%

\documentclass[12pt]{article}

\topmargin=-0.5in
\oddsidemargin=0in
\evensidemargin=0in
\textwidth=6.2in
\textheight=9.25in

\usepackage[dvips]{graphics}
\usepackage{rotating}
\usepackage{amssymb,amsmath}
\usepackage{graphicx}
\usepackage{cite}
\usepackage{color}

%% draftwatermark causes infinite loop on SL6's ancient TeXLive
\usepackage[firstpage]{draftwatermark}
\usepackage{lipsum}
\usepackage{tikz}

%\usepackage[T1]{fontenc}
\usepackage[utf8]{inputenc}
\usepackage{authblk}

\RequirePackage{lineno}

\bibliographystyle{plain}
\bibliographystyle{unsrt}

\usepackage{graphicx, subfigure}
\usepackage[colorlinks=true,urlcolor=black,linkcolor=black,citecolor=black,bookmarks=true]{hyperref}

\setcounter{tocdepth}{2}


\begin{document}
\linenumbers

\title{  Proposal for a Full-Scale Detector Engineering Test and Test Beam Calibration of a Single-Phase LAr TPC}

\date{\today}
\date{April 15, 2015}
	\input{authors}

\maketitle


\SetWatermarkText{DRAFT}
\SetWatermarkLightness{0.8}
\SetWatermarkScale{5}


\begin{abstract}

After a short introduction to the DUNE physics program we motivate the proposed single phase liquid argon detector and charged particle beam measurement program. 
We describe the required beam line and beam monitoring instrumentation for the project.
The proposed single phase liquid argon detector presently described corresponds to the LBNE detector design. Discussions about alternate designs are in progress and this proposal will be updated according to a developing consensus on the detector design. The detector will be placed inside a membrane cryostat which will be connected to a cryogenics systems for which we provide engineering details. The proposal concludes with a description of data handling and analysis plans, as well as a schedule and an overview of the organizational structure put in place to execute the plan.



\end{abstract}

\newpage
\tableofcontents

\newpage

%\section{Introduction [$\sim$5 pages; {\color{red} Thomas/Greg/B.Wilson}]}
\section{Introduction [$\sim$5 pages; {\color{red} Thomas/Greg/B.Wilson}]}
	



The preponderance of matter over antimatter in the early Universe, the dynamics of the supernova bursts that produced the heavy elements necessary for life and whether protons eventually decay - these mysteries at the forefront of particle physics and astrophysics are key to understanding the early evolution of our Universe, its current state and its eventual fate. The Experiment at the Long-Baseline Neutrino Facility (ELBNF) represents an extensively developed plan for a world-class experiment dedicated to addressing these questions.

Experiments carried out over the past half century have revealed that neutrinos are found in three states, or flavors, and can transform from one flavor into another. These results indicate that each neutrino flavor state is a mixture of three different nonzero mass states, and to date offer the most compelling evidence for physics beyond the Standard Model. In a single experiment, ELBNF will enable a broad exploration of the three-flavor model of neutrino physics with unprecedented detail. Chief among its potential discoveries is that of matter-antimatter asymmetries (through the mechanism of charge-parity violation) in neutrino flavor mixing - a step toward unraveling the mystery of matter generation in the early Universe. Independently, determination of the unknown neutrino mass ordering and precise measurement of neutrino mixing parameters by ELBNF may reveal new fundamental symmetries of Nature.

Grand Unified Theories, which attempt to describe the unification of the known forces, predict rates for proton decay that cover a range directly accessible with the next generation of large underground detectors such as the ELBNF detector. The experiment's sensitivity to key proton decay channels will offer unique opportunities for the ground-breaking discovery of this phenomenon.

Neutrinos emitted in the first few seconds of a core-collapse supernova carry with them the potential for great insight into the evolution of the Universe. ELBNF's capability to collect and analyze this high-statistics neutrino signal from a supernova within our galaxy would provide a rare opportunity to peer inside a newly-formed neutron star and potentially witness the birth of a black hole.

To achieve its goals, ELBNF is centered around three central components: (1) a new, high-intensity neutrino source generated from a megawatt-class proton accelerator at Fermi National Accelerator Laboratory (Fermilab), (2) a fine-grained near neutrino detector installed just downstream of the source, and (3) a massive liquid argon (LAr) time-projection chamber (TPC) deployed as a far detector deep underground at the Sanford Underground Research Facility (SURF). This facility, located at the site of the former Homestake Mine in Lead, South Dakota, is $\sim$1,300 km from the neutrino source at Fermilab - a distance (baseline) that delivers optimal sensitivity to neutrino charge-parity symmetry violation and mass ordering effects. This ambitious yet cost-effective design incorporates scalability and flexibility and can accommodate a variety of upgrades and contributions.

ELBNF plans to place modular LAr TPCs with a combined total fiducal mass of at least  40 kton in the underground facility at Homestake and into the neutrino beam.
The first 10~kton LAr TPC module is planned to be constructed underground on the time scale of 2021.

With its exceptional combination of experimental configuration, technical capabilities, and potential for transformative discoveries, ELBNF promises to be a vital facility for the field of particle physics worldwide, providing physicists from institutions around the globe with opportunities to collaborate in a twenty to thirty year program of exciting science.



\subsection{Key physics goals of ELBNF}


The primary goal of ELBNF  is to measure the appearance of electron neutrinos in a beam of muon neutrinos and 
and the appearance of electron anti-neutrinos in a beam of muon anti-neutrinos, each
 over the 1300~km baseline of the experiment. Precise measurement of this
phenomenon would allow for determination of the relative masses and mass ordering of the three
known neutrinos. Measurement of these neutrino oscillation channels also allow to constrain or measure the
CP violation phase, $\delta_{CP}$ in the neutrino sector, which is possibly connected to the dominance of matter
over antimatter in the universe.

For a baseline of 1300~km the first maximum of the oscillation probability occurs in the 2 - 3~GeV energy range with additional  
oscillation maxima at lower energies. Hence the high intensity neutrino flux must be peaked in this energy range. Coverage of the 
sub GeV energy range is desirable to potentially map out the second maximum in the oscillation probability.
It is this key physics which dictates the neutrino energy range and thereby the energy range of charged particles which result from neutrino 
interactions in the ELBNF detectors. 


\subsection{Single-phase LAr detector}

The basic components of the liquid argon detector include a cryostat and associated cryogenic system. A time projection chamber (TPC) and readout electronics are housed in the cryostat.

The cryostat contains the liquid argon target material and the cryogenic system keeps the liquid argon at a cryogenic temperature of 89K, and maintains the required purity through pump and filter system. A uniform electric field is created within the TPC volume between cathode planes and anode wire planes. Charged particles passing through the TPC release ionization electrons that drift to the anode wires. The bias voltage is set on the anode plane wires so that ionization electrons drift between the first several (induction) planes and is collected on the last (collection) plane. Readout electronics amplify and continuously digitize the induced waveforms on the sensing wires at several MHz, and transmit these data to the DAQ system for analysis. The wire planes are oriented at different angles allowing a 3D reconstruction of the particle trajectories. In addition to these basic components, a photon detection system is also included in the design to enable the study of proton decay and be sensitive to galactic supernova neutrinos.

The LAr detector design is characterized by a modular approach in which the LAr volume in the cryostat is instrumented with a number of identical anode wire plane assemblies (APA) and associated cathode plane assemblies (CPA). To a large extent, scaling from detector volumes containing from a few to several hundred of such modules should be straightforward with small and predictable risk.


\subsection{Goals for the prototype detector and beam test}

The physics sensitivity of ELBNF has been estimated based on detector performance characteristics published in the literature, simulation based estimates
as well as a variety of assumptions about the anticipated performance of the future detector and event reconstruction and particle identification algorithms.
The proposed single phase LAr prototype detector and CERN beam test aim to replace these assumptions with measurements for the full scale ELBNF detector 
components and the presently available algorithms. Thereby the measurements will allow to enhance the accuracy and reliability of the ELBNF physics sensitivity 
projections. The beam measurements will serve as a calibration data set to tune the Monte Carlo simulations and serve as a reference data set for measurements of the 
future ELBNF detector. In addition, the measurement program aims to evaluate and benchmark the performance of the detector and its individual components.
This will allow to identify potentially problematic components and lead to future improvements and optimizations of the detector design.

In order to make such precise measurements, the detector will need to accurately identify and measure the energy of the particles produced in the neutrino interaction with Argon which will range from hundreds of MeV to several GeV.
To mitigate the risks associated with extrapolating small scale versions of the single-phase LAr TPC technology to a full-scale detector element, it is essential to benchmark the operation of a full-scale detector elements in a well characterized charged particle beam.  

More specifically, the goals of the prototype detector and beam test measurements include the
the use of a charged particle beam to:
\begin{enumerate}
\item measure the detector calorimetric response for
\begin{enumerate}
	\item hadronic showers
	\item electromagnetic showers
\end{enumerate}
\item study e/$\gamma$-separation capabilities
\item measure event reconstruction efficiencies as function of energy and particle type based on experimental data
\item measure performance of particle identification algorithms as function of energy and for realistic detector conditions
\item assess single particle track calibration and reconstruction
%		-- characterize performance of algorithms
\item validate accuracy of Monte Carlo simulations for relevant energy ranges as well as directions

%  \item secondary hadron interactions in detector
\item study other topics with the collected data sets
 \begin{enumerate}
    \item pion interaction kinematics and cross sections
    \item kaon interaction cross section to characterize proton decay backgrounds ...
    \item muon capture for charge identification
 \end{enumerate}
\end{enumerate}

For the detector performance characterization a well defined charged particle test beam will 
enable the following detector performance measurements:
\begin{enumerate}
 \item characterize performance of full scale TPC module
 \item verify functionality of cold TPC electronics under LAr cryogenic conditions
 \item perform full-scale structural test under LAr cryogenic conditions
 \item study performance of the photon detection system
 \item verify argon contamination levels and associated mitigation procedures
 \item develop and test installation procedures for full-scale detector components
\end{enumerate}


The CERN charged particle beam lines provide an opportunity to perform this crucial test of the 
proposed single-phase LAr TPC and thereby inform the decision regarding the far detector design and layout for ELBNF.
In order to be of greatest value to this decision making process results should be available as soon as possible.

This technical document describes the motivation and technical details for an initial  measurement program that we propose to be executed by mid 2018, 
that is {\it before} the anticipated LHC long shutdown. The estimated required beam time amounts to $\sim$ XX weeks of data collection. 
%
Additional follow-up measurements with potentially modified detector components form a potential extension to the proposed program.




%	\input{../150212_introduction_Greg}

%\section{CERN prototype detector and charged particle beam test [$\sim$10 pages; {\color{red} Donna/Jarek}]}
\section{Scientific Motivation and Measurement Program [$\sim$10 pages; {\color{red} Donna/Jarek}]}
	\label{detbeamtest}

The primary goals of DUNE are to 
constrain or discover CP violation in the lepton sector by determining 
the value of the
CP violating phase, $\delta_{CP}$, in the PMNS mixing matrix and
to determine the mass ordering of the three neutrino mass eignestates. 
This will be accomplished through measurement of 
appearance rate of electron neutrinos and electron anti-neutrinos 
as well as the corresponding disappearance rate of muon neutrinos 
and muon anti-neutrinos over the 1300~km baseline of the experiment. 
Other important physics goals include sensitive searches for proton decay and detection of supernova neutrinos.

The full power of the DUNE experiment to perform a careful test of the three-flavor paradigm will come from a measurement of the detected neutrino spectral shape over a broad energy range.  For a baseline of 1300 km the first maximum of the oscillation probability occurs in the 2 - 3 GeV energy range with additional oscillation maxima at lower energies so the high intensity neutrino flux must be maximized in this energy range. It is desirable to have sufficient flux in the sub-GeV energy range to enable a measurement of the rapidly changing spectral shape in the region of the second maximum in the oscillation probability. It is this requirement on the neutrino energy spectrum, and the subsequent energy range of charged particles that result from their interactions, that determines the performance requirements for the DUNE detectors. 


One of the main goals of the single-phase prototype test beam program is to perform measurements 
needed to control and understand systematic uncertainties in DUNE oscillation measurements.
%The program also includes measurements to support other important DUNE physics measurements as described below.
%detector is intended to provide input necessary to reduce systematic uncertainties for oscillation measurements 
%

%Current assumptions on detector related systematic uncertainties in DUNE to achieve
%projected sensitivities~\cite{DuneCDR} are shown in
%%Assumptions on current level of uncertainties are shown in 
%Table \ref{table:deterr}. These are compared with teh levels achieved in MINOS and T2K
%appearance measurements.
%\begin{table}[h]
%\centering
%\caption{Current estimated detector related  sources of uncertainty for oscillation 
%measurements.}
%\label{table:deterr}
%\begin{tabular}{|l|c|c|c|l|}
%\hline
%\textbf{Source of uncertainty } & \textbf{MINOS} & \textbf{T2K} & \textbf{DUNE} & \textbf{Comments}  \\ \hline
%$\nu_e$ energy scale  & 2.7\% & 2.5\% & 2\% & comment \\ \hline
%\end{tabular}
%\end{table}
%% THIS IS FROM TABLE 3.8 of CDR


As an example of the importance of controlling detector related uncertainties,
Fig.~\ref{fig:spectraleffect} shows the effect of a lepton energy scale 
shift of -5\% on the measured 
appearance signal %(for $\delta_{CP}=0$)
 and backgrounds in DUNE. 
The expected signal shape is shown for several values of $\delta_{CP}$. 
(solid curve: $\delta_{CP}=0$, lower crosses: $\delta_{CP}=+\pi/2$, and upper crosses: $\delta_{CP}=-\pi/2$). 
Energy scale uncertainties will distort and shift the $\nu_e$ appearance spectra and
can mimic a non-zero
$\delta_{CP}$ phase.
\begin{figure}[h!]
\centering
%\includegraphics[width=0.7\textwidth,height=7.7cm]{figures/lepbias10}
\includegraphics[width=0.7\textwidth,height=7.7cm]{figures/CSPP_LeptonBias_nue_app_FHC}
\label{fig:spectraleffect}
  \caption{DUNE $\nu_e$ appearance signal and background spectra. 
Solid curves show the effect of -5\% lepton energy scale shift on the 
measured appearance $\nu_e$ and $\overline{\nu}_e$ signals assuming $\delta_{CP}=0$.
The crosses show the expected signal shape for $\delta_{CP}=+\pi/2$ (lower crosses) and -$\pi/2$  (upper crosses). 
Ratios show the distortion on the $\nu_e$ and $\overline{\nu}_e$ spectra due to the 
5\% energy scale shift.
}
\end{figure}


 Fig.~\ref{fig:global_escale_sens} taken from Ref.~\cite{dunecdr} demonstrates that
energy scale uncertainties will impact DUNE 
%shows the effect of 
%three values of neutrino energy scale uncertainty on the DUNE 
mass hierarchy (left) and $\delta_{CP}$ (right) sensitivities.
The exposure to achieve this level of sensitivity corresponds to 
%The nominal sensitivity assumes a
 230-400 kt-MW-years for mass hierarchy and  850-1320 kt-MW-years for CPV depending on the 
details of the LBNF neutrino beam design.
%exposure with equal neutrino and antineutrino running. 
Work to evaluate the effect of all systematic uncertainties in DUNE sensitivities is still in progress.
This study already indicates a signficant reduction of the CPV peak sensitivity. 
This example assumes linear neutrino energy scale variations at the levels indicated.
More complex (and more likely) scenarios for energy scale uncertainties are under study and
will likely result in larger effects on the sensitivities.
%including separate particle
% correlated variations 
%It therefore likely underestimates the probable effects  of particle energy reconstruction 
%on DUNE sensitivities.
%The size of effects shown here do not account for correlated uncertainties in neutrino and 
%antineutrino running or effects on backgrounds and is therefore likely
%a best-case scenario. 
\begin{figure}[h!]
\centering
\includegraphics[width=0.49\textwidth,height=6.7cm]{figures/mh_230ktmwyear_varyesyst}
\includegraphics[width=0.49\textwidth,height=6.7cm]{figures/cpv_890ktmwyear_varyesyst}
%\includegraphics[width=0.49\textwidth,height=6.7cm]{figures/mh_escale_sens}
%\includegraphics[width=0.49\textwidth,height=6.7cm]{figures/deltacp_escale_sens}
\label{fig:global_escale_sens}
  \caption{One scenario for the effect of 1-sigma  neutrino energy scale uncertainties on
DUNE projected sensitivity for mass hierarchy (left) and $\delta_{CP}$ 
(right).
A 115 kt-MW-years exposure in each of neutrino and antineutrino modes is assumed.
% to one sigma neutrino energy scale uncertainties as indicated.
%If energy scale uncertainties can be controlled at the appropriate levels, DUNE can achieve 
%at least 5$\sigma$ sensitivity on mass hierarchy determination for 100\% of $\delta_{CP}$ values and
%for 3$\sigma$ sensitivity to  $\delta_{CP}$ for 75\% coverage of phase space.
}
\end{figure}

%Current levels of sensitivity in ~\cite{dunecdr} assumes
%$\nu_e$ energy scale is known at the level of 2\%. 
%More here...
%will require a dedicated test beam


%$\nu_e$ energy scale  & 2.7\% & 2.5\% & 2\% & comment \\ \hline
%\end{tabular}


\subsection{Summary of Detector and Beam Requirements }
\label{detbeam_main}

LAr TPC technology was first proposed for use in neutrino experiments by C. Rubbia in 1977
\cite{CRubbia} but extensive use in neutrino experiments is only now being realized. 
%CERN-EP-INT-77-8
%Title 	The liquid-argon time projection chamber : a new concept for neutrino detectors
%operating in the CNGS beamline  (with mean beam energy $\sim$17 GeV)~\cite{ICARUSmain}. 
The ICARUS T600 detector~\cite{icarus_mainref} pioneered the first large-scale detector when it operated in the CNGS 
neutrino beam at mean energy $\sim$17 GeV. ArgoNeuT~\cite{argoneut1}\cite{argoneut2} recently studied 
neutrino interactions in the NuMI beam down to sub-GeV energies with a small-scale (170~l fiducial volume) detector. 
While these samples are proving useful, they do not allow full isolation of
the low energy neutrino interaction processes
and final states from reconstruction and detector effects. 
The use of this technology in future precision neutrino experiments will require dedicated 
information on particle response
in the sub-GeV to few-GeV range provided by charged-particle test beams. 

%%%
%\subsubsection{Particles energy and direction}
%\label{detbeam_particles}
The DUNE experiment will run in both neutrino and anti-neutrino 
configurations. These beams will be composed  mainly of muon neutrinos (anti-neutrinos) as well as electron neutrinos (anti-neutrinos). In Fig.~\ref{fig:particle_momenta} the distributions of momenta and angles of particles created in neutrino interactions from 
simulated beam fluxes, including oscillation effects, are shown. The particle rates are calculated for a 34kt far detector and  are combined from both neutrino and anti-neutrino beams.
% In addition the electron rates from charged current interactions for muon neutrinos oscillated to electron neutrinos are shown. 
\begin{figure}[h!]
  \centering
\includegraphics[width=0.49\textwidth,height=6.0cm]{figures/True_Momenta_per_Particle_9_2_1_0_logy_logx} 
\includegraphics[width=0.49\textwidth,height=6.0cm]{figures/True_theta_per_Particle_9_2_1_0_lin}
  \caption{Particle momenta (left) and angular (right) distributions for particles produced in neutrino interactions 
from $\nu_e$, $\nu_\mu$, $\bar \nu_e$ and $\bar \nu_\mu$ at the far detector location.
%{\color{red}  combine e+e-, improve information content (y-axis, change colors, etc )
%}
}
\label{fig:particle_momenta}
\end{figure}


%\begin{figure}[h!]
%  \centering
%\includegraphics[scale=0.4]{figures/True_theta_per_Particle}
%\label{fig:particle_theta}
%  \caption{Particle angle wrt to the beam axis distributions for particles coming from all fluxes ($\nu_e$, $\nu_\mu$, $\bar \nu_e$ and $\bar \nu_\mu$) at both near and far detector locations.  }
%\end{figure}

%\newpage

The detector is designed from components that match exactly the current DUNE reference far detector design components. 
The test beam detector must be sufficiently large in both
longitudinal and transverse dimensions to contain showering particles up to the energy range of interest ($\sim$10~GeV).
Fig.~\ref{fig:containment} shows the simulated longitudinal and transverse 
energy containment for proton showers up to 10~GeV in energy.
For 10 GeV showers, more than 95\% of the energy is contained in a detector of longitudinal size of 6~m and 
radius of 2.5~m. Showers from pions, kaons, and electrons have also been studied and better containment
is achieved in those cases. 
\begin{figure}[htp]
  \centering
  \begin{tabular}{ccc}
%    \includegraphics[scale=0.15]{figures/electrons_density_overlay}&
%    \includegraphics[scale=0.15]{figures/electrons_lcont_overlay}&
%    \includegraphics[scale=0.15]{figures/electrons_wcont_overlay}\\
%  
%    \includegraphics[scale=0.15]{figures/photons_density_overlay}&
%    \includegraphics[scale=0.15]{figures/photons_lcont_overlay}&
%    \includegraphics[scale=0.15]{figures/photons_wcont_overlay}\\
%%
%   \includegraphics[width=0.31\textwidth,height=3.9cm]{figures/protons_density_overlay}&
   \includegraphics[width=0.49\textwidth,height=4.9cm]{figures/protons_lcont_overlay}&
   \includegraphics[width=0.49\textwidth,height=4.9cm]{figures/protons_wcont_overlay}\\
%    \includegraphics[scale=0.15]{figures/protons_lcont_overlay}&
%    \includegraphics[scale=0.15]{figures/protons_wcont_overlay}\\
% 
%    \includegraphics[scale=0.15]{figures/pions_density_overlay}&
%    \includegraphics[scale=0.15]{figures/pions_lcont_overlay}&
%    \includegraphics[scale=0.15]{figures/pions_wcont_overlay}\\
 
%   \includegraphics[width=0.31\textwidth,height=3.5cm]{figures/pions_density_overlay}&
%   \includegraphics[width=0.31\textwidth,height=3.5cm]{figures/pions_lcont_overlay}&
%   \includegraphics[width=0.31\textwidth,height=3.5cm]{figures/pions_wcont_overlay}\\
%    \includegraphics[scale=0.15]{figures/pions_density_overlay}&
%    \includegraphics[scale=0.15]{figures/pions_lcont_overlay}&
%    \includegraphics[scale=0.15]{figures/pions_wcont_overlay}\\
 
% 
%    \includegraphics[scale=0.15]{figures/kaons_density_overlay}&
%    \includegraphics[scale=0.15]{figures/kaons_lcont_overlay}&
%    \includegraphics[scale=0.15]{figures/kaons_wcont_overlay}\\
 
  \end{tabular}
  \caption{Simulated longitudinal and transverse containment for proton showers of 4 and 10~GeV/c momenta.
%{\color{red}
%Improve or perhaps remove far left plot (binning too fine).
%Add info on simulation details and containment defn (T. Junk).
%}
}
  \label{fig:containment}  
%For 10 GeV showers, more than 95\% of the energy is contained in a detector of longitudinal size of 550~cm and 
%radius of 200~cm.}
\end{figure}

%\clearpage
%\subsubsection{Particle rates}
%\label{detbeam_rates}
%Estimation of  beam particles rates  necessary to collect high enough statistics in a reasonable time to obtain goals of of the measurements.
% THis should be discussed in the beam section
\subsection {Summary of Beam Particle Requirements}

Table~\ref{tab:runsum} summarizes the requested particle types and momenta along with 
required exposures for the test beam program.
\begin{table}[h]
\centering
\begin{tabular}{|c|c|c|l|}
\hline
Particle & Momenta (GeV/c) & Exposure & Purpose \\ \hline
$\pi^+$       & 0.2, 0.3, 0.4, 0.5, 0.7, 1, 2, 3, 5, 7     &  10K  & hadronic cal, $\pi^0$ content \\ \hline
$\pi^-$       &  0.2, 0.3, 0.4, 0.5, 0.7, 1     &  10K  & hadronic cal, $\pi^0$ content \\ \hline
$\pi^+$   &  2  &  600K & $\pi^o$/$\gamma$ sample \\ \hline
%$\pi^+$ &   1 \& 2  &  10K  & vary angle ($\times$5), reco \\ \hline
proton &  0.7, 1, 2, 3   &  10K & response, PID \\ \hline
proton &  1   &  1M & mis-ID pdk, recombination \\ \hline
e$^+$ or e$^-$       &    0.2, 0.3, 0.4, 0.5, 1, 2, 3, 5, 7        &    10K   & e-$\gamma$ separation/EM shower     \\ \hline
% e$^+$ or e$^-$  &  1 \& 2  &  10K  & vary angle($\times$5), reco \\ \hline
%e$^+$ or e$^-$   (w/rad) &  3  &  20K  & tagged photons \\ \hline
$\mu^-$  &   (0.2), 0.5, 1, 2  &  10K & $E_\mu$, Michel el., charge sign \\ \hline
$\mu^+$ &   (0.2), 0.5, 1, 2   &  10K & $E_\mu$, Michel el.,charge sign  \\ \hline
$\mu^-$ or $\mu^+$ &   3, 5, 7  &  5K & $E_\mu$ MCS \\ \hline
%$\mu^-$ or $\mu^+$  &  1 \& 2  &  5K  & vary angle ($\times$5), reco \\ \hline
%proton &  1 \& 2 &  10K & vary angle ($\times$5), reco \\ \hline
antiproton &  low-energy tune  &  (100) & antiproton stars \\ \hline
K$^+$  & 1 & (13K)   &   response, PID, PDK  \\ \hline
K$^+$  & 0.5, 0.7 & (5K)   &   response, PID, PDK  \\ \hline \hline
$\mu$, e, proton  & 1 (vary angle $\times$5) & 10K  & reconstruction  \\ \hline
\end{tabular}
\caption{Requirements summary for particle types and momenta.
The Exposure column indicates the number of particles for each momentum point.
Items in parenthesis indicate lower priority (see text).
}
\label{tab:runsum}
\end{table}


Pions and protons spanning the energy range expected in DUNE beam neutrino interactions will be used 
primarily to study hadronic showering reconstruction and calibration as described in Sec.~\ref{sec:showers}
 as well as particle identification (PID) algorithms discussed in Sec.~\ref{detbeam_pid}. A sample of 600K
 2~GeV $\pi^+$ will be used to study  secondary $\pi^o$ over a large angular range 
to tune and calibrate electron/photon separation algorithms (see Sec.~\ref{sec_egam}).
A large sample (10$^{6}$) 1 GeV protons will produce a sample of low energy stopping protons over a 
wide angular range needed to study angular
dependent effects on collected charge as described in Sec.~\ref{sec_angle}.
Electrons will be used to benchmark and tune  electron/photon separation algorithms and to calibrate 
electromagnetic showers as discussed in \ref{sec_egam}.
Muon (and antimuon) samples are needed to study reconstruction and PID calibration as well as to obtain 
Michel electron events for calibration  
algorithms for charge-sign determination (see Secs.\ref{sec_reco} and \ref{sec_other}).

Charged-kaon samples will be useful to characterize kaon PID efficiency for proton decay sensitivity but
are a somewhat lower priority need. The antiproton sample
will be helpful for exotic physics sensitivity studies. Both of these lower priority requests are discussed
in Sec.~\ref{sec_other}.

%Charged pion samples will be used to characterize hadronic shower response and to measure
%absorption cross section parameters on argon.
%for all energies except 0.1 GeV point). 
%Muon samples will be used for calibration and 
%reconstruction tests. Electron samples will be used to measure EM shower response  
%and to tune PID algorithms (statistics shown will far exceed 1\% uncertainty on the mean for 
%EM showers with resolution on the order of 1\% as measured in DOCDB 9434 and 8835. 100k Electron event samples will allow
%high statistics studies of e-gamma separation).  
%Proton response will be studied for reconstruction and to tune PID 
%algorithms. Kaon data will be needed to tune proton decay backgrounds .
%Special runs at various angles with $\pi$, $\mu$, p and electrons will be performed to study reconstruction and tune PID algorithms. 

\subsection{Detector performance tests}

The prototype detector will allow to study the detector response to charge particles from the test beam. The measured energy deposition for various particles and its dependence on the direction of the particle will be used to tune
Monte Carlo simulations and allow more precise reconstruction of neutrino energy and interactions topologies with good particle identification.


\subsubsection{Shower calibration}
\label{sec:showers}

Accurate measurement of neutrino energy will require reconstruction of both electromagnetic and hadronic showers. Reconstruction of hadron energy 
in these energy ranges will require knowledge of the fate (interact, decay, or stop) of the
initiating hadron ($\pi^{+/-}$, $p$, or $K^{+/-}$).
%fate (interact, decay, or stop). 
For the case of  interacting hadrons the composition of secondaries
will need to be determined to characterize the response. 
These will include neutrals and particles which 
deposit energy electromagnetically ($\pi^o$, $\gamma$), as well as
secondary hadrons,
%and their energy responses 
The test beam with known incoming particle type and momentum will be used
to characterize interacting hadrons in this energy range.
%quantify responses to initiating hadrons
%($\pi^{+/-}$, $p$, or $K^{+/-}$)
%Hadronic showers initiated by protons in this energy range require

Fig.~\ref{fig:hadronshwr} shows the fraction of true energy deposited by interacting protons with 1~GeV/c (left) and
3~GeV/c (right) incident momenta simulated using FLUKA particle transport code~\cite{fluka05}. 
Interacting protons (65\% of the 1~GeV/c sample) are selected.
% The remaining 
%35\% which range out are be used to study particle identification algorithms (PID) (see Sec.~\ref{detbeam_pid}). 
%Reconstruction is performed with ICARUS spatial and calorimetric reconstruction algorithm~\cite{icarus_reco}.
For this study, visible energy is summed using hit information with corrections applied for the lifetime of 
the drift electrons. (No attempt is made here to correct for recombination effects or electromagnetic shower fractions). 
The resulting energy deposition in the two cases cannot be 
accurately characterized by an average shower calibration factor. Monte Carlo simulations of 
outgoing particles, especially at low energies, must be checked and bench-marked against calibration data to avoid
large uncertainties from shower modeling. 
\begin{figure}[h!]
  \centering
%\includegraphics[width=0.49\textwidth,height=5.0cm]{figures/protons_1gev_v0}
%\includegraphics[width=0.49\textwidth,height=5.0cm]{figures/protons_3gev_v0}
\includegraphics[width=0.49\textwidth,height=5.0cm]{figures/pr1GeV_1}
\includegraphics[width=0.49\textwidth,height=5.0cm]{figures/pr3GeV_1}
\label{fig:hadronshwr}
  \caption{Fraction of true energy deposited by interacting protons of 1~GeV/c (left) and
3~GeV/c (right) momenta simulated using FLUKA~\cite{fluka05}.
}
\label{fig:hadronshwr}
\end{figure}

Pion showers at low energies will also be important both for determining the interacted neutrino energy as well
as for modeling neutral current backgrounds resulting from $\pi^o$ content in showers. Significant
 differences in energy deposited in interactions initiated
by $\pi^+$ versus $\pi^-$  are present up to momenta on the order of 1~GeV/c due to different
final state particles and interaction cross sections. This is illustrated in 
Fig.~\ref{fig:pionshwr} which shows the differences in mean energy deposited (left) and width (right) 
for interacting pions ranging from 0.2~GeV/c up to 5~GeV/c momenta.
% simulated using FLUKA~\cite{fluka05}.
Resulting shower calibrations and reconstruction will differ and therefore each charge must be 
studied separately.

\begin{figure}[h!]
  \centering
%\includegraphics[width=0.49\textwidth,height=5.0cm]{figures/pi+pi-_means}
%\includegraphics[width=0.49\textwidth,height=5.0cm]{figures/pi+pi-_sig}
\includegraphics[width=0.49\textwidth,height=6.0cm]{figures/pipimean_ticks1}
\includegraphics[width=0.49\textwidth,height=6.0cm]{figures/pipisigma_ticks1}
  \caption{Differences in mean energy deposited (left) and width of visible energy (right) 
for interacting $\pi^+$ versus $\pi^-$ ranging from 0.2~GeV/c up to 5~GeV/c. 
}
\label{fig:pionshwr}
\end{figure}


% 
%\begin{table}[h]
%\centering
%\begin{tabular}{|c|c|c|}
%\hline
%Particle     & Momenta (GeV)                                                                                       & Exposure/bin (total)  \\ \hline
%$\pi ^+ $   & 0.2-1.0 (100MeV bins), 1.0-10.0 ( 200MeV bins)    &  1000 (48k)     \\ \hline
%$\pi ^- $    & 0.2-1.0 (100MeV bins), 1.0-10.0 ( 200MeV bins)    &  1000 (48k)     \\ \hline
%$e^+$       & 0.2-10  (100MeV bins), 1.0-10.0 ( 200MeV bins)    &  1000 (48k)       \\ \hline
%$e^- $       & 0.2-10  (100MeV bins), 1.0-10.0 ( 200MeV bins)    &  1000 (48k)       \\ \hline
%$\mu^+$   & 0.2-1.0 (100MeV bins), 1.0-10.0 ( 200MeV bins)    &  1000 (48k)     \\ \hline
%$\mu^-$    & 0.2-1.0 (100MeV bins), 1.0-10.0 ( 200MeV bins)    &  1000 (48k)     \\ \hline
%$p$          &  0.2-1.5 (100MeV bins), 1.5-10.0 ( 200MeV bins)    &  1000 (56k)     \\ \hline
%$\bar p$   &  0.2-1.5 (100MeV bins), 1.5-10.0 ( 200MeV bins)    &  1000 (56k)     \\ \hline
%$K^+$      &  0.2-1.5 (100MeV bins), 1.5-10.0 ( 200MeV bins)    &  1000 (56k)     \\ \hline
%$K^- $      &  0.2-1.5 (100MeV bins), 1.5-10.0 ( 200MeV bins)    &  1000 (56k)     \\ \hline
%\end{tabular}\caption{Data sample requirements for shower calibrations.  Currently about 1k particles is assumed per bin to include variations in the shower topologies. Details MC analysis is necessary. }
%\end{table}
%


\subsubsection{Cross section measurements}


Final state pions are produced copiously in neutrino interactions in our energy range 
of interest
and contribute substantially to total visible energy in the interaction.
These particles can re-interact or be absorbed in the nuclear medium
and substantially change the visible energy deposited in the event. 
Effects of modeling final state pion interactions on reconstructed neutrino energy have 
been recently demonstrated in the NuMI and Booster beam energy ranges~\cite{miniboonefsi, minervafsi}. 

Existing data used to tune the models
cover limited energies ($<$200~MeV) and are primarily on lighter target nuclei~\cite{fsirev}.
% need to check this reference from R. Ransom talk.
The requested $\pi^+$ and $\pi^-$ samples will allow new data samples for measuring
exclusive final state processes over the full relevant energy range and specifically on argon nuclei. 


%\item pion absorption on argon - Kotlinski, EPJ 9, 537 (2000)
%\item pion cross section as a function of A - Gianelli PRC 61, 054615 (2000)

%There is not currently a satisfactory theory describing absorption. The Valencia group (Vicente-Vacus NPA 568, 855 (1994)) developed model of    the pion-nucleus reaction with fairly good agreement, although not in detail. The actual  mechanism of multi-nucleon absorption
% is not well understood. 
 


\subsubsection{Angular Dependence}

\label{sec_angle}

%Ionization charge deposited by charged-particles in the LAr TPC must be calibrated to
%account for recombination effects. 

A track angular dependent correction must be
%Track angle must be accounted for and a correction
applied to deposited charge to accurately calibrate
dE/dx that is used for track momentum and particle ID (see Sec.~\ref{detbeam_pid}). 
An additional angular dependent effect could be present 
due to charge recombination. 
% effects could also be angular dependent. 
%It this is the case
%this will introduce a component that is currently not modeled. 
For example, Jaffe columnar recombination model~\cite{jaffe,argoneut_angle} predicts 
angular dependence given by 
$$Q \approx \frac{Q_o}{1+k_c (dE/dx) / {\cal E} \sin\phi}, $$ 
where
%Recombination modeling 
$Q_o$ and $Q$ are the ionization charge and the collected charge respectively, 
%$dE/dx$ is range based stopping power.
$k_c$ is a constant that depends on LAr diffusion and mobility coefficients, $\cal E$ 
is the electric field strength, and $\phi$ is the angle relative to the drift direction.
%
ICARUS~\cite{icarus_recombination} and LArSoft default recombination models do not currently incorporate angular dependence. 
%(ArgoNeuT~\cite{arogneut_angle} suggests a 
%modified Box model with explicit angular dependence can be incorporated 
%into Birk's law  by replacing ${\cal E}$  with ${\cal E} \sin\phi$).
Recent results from ArgoNeuT~\cite{argoneut_angle} find a small unmodeled
angular dependence in a data sample obtained using stopping protons from 
neutrino interactions in the range 50-300 MeV with angles in the range 40-90$^{\circ}$.

Additional data covering a large angular range would be useful to further investigate this possibility.
The large sample of 1~GeV protons requested will also be used to study angular dependence using 
samples of secondary stopping protons produced in showers at a range of angles with
respect to the field direction.

\subsubsection{Bethe-Bloch parametrization of charged particles and PID}

%
\label{detbeam_pid}

%The reconstruction of events in the LAr TPC is still a challenge but rapid progress has been achieved in recent years (cite pandora and other reconstruction algorithms). Despite the progress reconstruction algorithms have to rely Monte Carlo predictions which don't simulate liquid argon detectors responses correctly. Reconstruction algorithms will benefit greatly from test beam data particularly from the full scale prototype. The reconstruction algorithms will be trained to correctly reconstruct track, electromagnetic and hadronic showers.
%The data of tracks and showers can be used to create a library of reference events with which to tune algorithms.
%which can be sed for matching with he neutrino data, similar to the  LEM (library event matching).

Information on range and charge deposition for stopping charged-particles can be used to 
accurately identify particle type as well as measure kinetic energy. 
Fig.~\ref{fig:resrange}  (left) shows track energy loss per unit length\footnote{Signal attenuation due to recombination effect is not corrected for the PID purpose since it is a deterministic transformation which does not add measurement information.}
(dQ/dx) as a function of residual 
track range for simulated muon, pion, proton and kaon particle tracks. 
A neural-net-based algorithm~\cite{nn_pid,rd_pid}
was used to determine PID functions for each particle hypothesis.
% at a bin centered at 6$\pm$1~cm.
%  for each particle hypothesis.
%functions for each particle hypothesis. 
Fig.~\ref{fig:resrange} (right) 
shows the distribution of reconstructed dQ/dx in a narrow bin of residual range 6($\pm$1)cm to illustrate the scale of difference between particle types. Proton and kaon bands are well separated from each other and from pion and muon bands, while pion and muon bands are almost overlapping which is challenging for the efficient separation. 
Simulation is performed with
FLUKA~\cite{fluka05} and ICARUS geometry (3~mm wire pitch) and reconstruction code~\cite{icarus_reco}.
%Stopping particles were selected using truth information.

%shows the result for events in the residual range bin centered at 6($\pm$1)~cm.
The PID efficiency and purity will  depend on the detector configuration, geometry and 
reconstruction algorithm. The DUNE APA configuration and reconstruction algorithms
%Fig.~\ref{fig:resrange} uses ICARUS geometry with 3~mm pitch, APA configuration 
may give significantly different results, especially for pion/muon separation. 
%Plots were made for 3mm pitch, we don't have such simulation for APA configuration, it may differ a bit (which does not change much for proton/kaon, but can be significant for pion/muon).
It is therefore important to test PID in a realistic prototype detector
data in the presence of all detector effects.
Event samples which include an adequate number of stopping particles for each species
(muon, pions, protons and kaons) are included in the particle summary request in order to 
perform tests of PID algorithms and Bethe-Bloch calibration measurements.


%The tail is used to calibration the MIP response, the region near the end of the track is used
%to give PID information. (WIP)
%Test beam measurements of stopping particles of each type will be used
%to calibrate the energy deposition functions and to determine particle 
%mis-ID probabilities. 
\begin{figure}[h!]
  \centering
\includegraphics[width=0.53\textwidth,height=6.0cm]{figures/pids_new}
\includegraphics[width=0.46\textwidth,height=6.0cm]{figures/prkpimu_new}
%\includegraphics[width=\textwidth,height=5.0cm]{figures/pid_curves}
  \caption{(left) ICARUS simulated track dQ/dx as a function of residual range for muons, pions, protons and kaons, used as training data in a neutral-net-based PID algorithm. (right) Distribution of dQ/dx for each particle type for a bin with residual range 
6($\pm$1)~cm. The small difference between muon and pion PID is illustrated.
}
\label{fig:resrange}
\end{figure}




%collected particle charge, track angle, as well as 
%which depends strongly 

\subsubsection{Reconstruction Effects}
\label{sec_reco}

Reconstruction algorithms use all three signal planes for 3D track and event reconstruction. The quality of reconstruction is affected by two main factors that can be quantified with test beam data:
\begin{itemize}
\item The complexity of the event topology, which includes the number of objects overlapping in 2D projections and the number of possible object associations between 2D projections. The topology complexity depends on the energy of incident particle. Hadronic showers collected with the test beam can provide data sample to assess the reconstruction algorithm performance and test its dependence on the incident particle energy.
\item Orientation of the reconstructed object w.r.t. the readout wires and electron drift direction:
\begin{itemize}
\item Tracks and cascades in the plane parallel or almost parallel to the signal planes have strongly limited variation of the hit drift time, required for the 3D reconstruction; various reconstruction algorithms can show different performance in real detector conditions for such inclined objects, especially in the presence of noise affecting hit time reconstruction.
\item A 2D projection of objects aligned with the wires of one of the planes is strongly shortened, which limits the amount of information available to the reconstruction algorithm; the two other signal planes can be used for spatial reconstruction, however, the validation of the correctness of reconstruction, calculated from the 3D object projected to the third plane, is less efficient.
The most important aspect of this issue is the use of all planes for the charge measurement. If the reconstructed object is aligned with collection wires then the calorimetric measurement has to be carried out with the signal from the induction planes. The additional shielding wire plane in the DUNE design will improve the quality of the bipolar induction plane signals.  The test beam 
data will help with the calibration of these signals.
\item Objects aligned with the drift field are projected onto a low
number of wires and have a large span of drift time. The wire signals in these
cases are significantly different than those from tracks at higher angles with
respect to the drift field, and therefore require a dedicated signal processing
algorithm to do hit reconstruction. Corrections for angular dependence of
recombination will also need to be included in the calorimetric reconstruction
and calibrated with real data.
%Objects aligned with the drift field are projected to a low number of wires and have a large span of drift time. Wire signals of such objects are significantly different than those of objects at higher angles w.r.t. the drift field, requiring a dedicated signal processing for the hit reconstruction. In such orientation also the correction of eventual angular dependence of the recombination effect should be included in the calorimetric reconstruction and calibrated with real data.
\end{itemize}
\end{itemize}

%Main issues for the reconstruction algorithms:
%\begin{itemize}
%\item Reconstruction algorithms use all three signal planes for 3D track and event reconstruction. 
%If the orientation of the track/shower is such that it is aligned with wires on one of the planes, it significantly reduces quality of reconstructed objects. 
%\item Calorimetry with collection and induction planes. In the ICARUS experiment the deposited energy was reconstructed from the signal on the collection plane. The induction planes bipolar signal wasn't "stable" enough to use it for calorimetric measurement. In the ELBNF design there is additional shielding  wire plane which will improve the quality of the bipolar signal and the  test beam experiment will help with its calibration.
%\item   Vertexing.
%\item Reconstruction efficiency for low energy particles. The reconstruction algorithm suffer from the lose of efficiency for low energy particle or particles which leave less than 200-300 hits. Training the algorithms on a low energy particles from the test beam will improve the quality and efficiency of the reconstructed objects.
%\end{itemize}

Any reconstruction algorithm will depend on the particular choices in the TPC design. 
Therefore bench-marking our reconstruction algorithms with a prototype of the final design will
be invaluable to understand the performance of reconstruction.

%
%\begin{table}[h]
%\centering
%\begin{tabular}{|c|c|c|}
%\hline
%Particle     & Momenta (GeV)                                                                                       & Exposure/bin (total)  \\ \hline
%$\pi ^+ $   & 0.2-1.0 (100MeV bins), 1.0-10.0 ( 200MeV bins)    &  500 (26.5k)     \\ \hline
%$\pi ^- $    & 0.2-1.0 (100MeV bins), 1.0-10.0 ( 200MeV bins)    &  500 (26.5k)     \\ \hline
%$e^+$       & 0.2-10  (100MeV bins), 1.0-10.0 ( 200MeV bins)    &  500 (26.5k)       \\ \hline
%$e^- $       & 0.2-10  (100MeV bins), 1.0-10.0 ( 200MeV bins)    &  500 (26.5k)       \\ \hline
%$\mu^+$   & 0.2-1.0 (100MeV bins), 1.0-10.0 ( 200MeV bins)    &  500 (26.5k)     \\ \hline
%$\mu^-$    & 0.2-1.0 (100MeV bins), 1.0-10.0 ( 200MeV bins)    &  500 (26.5k)     \\ \hline
%$p$          &  0.2-1.5 (100MeV bins), 1.5-10.0 ( 200MeV bins)    &  500 (27.8k)     \\ \hline
%$\bar p$   &  0.2-1.5 (100MeV bins), 1.5-10.0 ( 200MeV bins)    &  500 (27.8k)     \\ \hline
%$K^+$      &  0.2-1.5 (100MeV bins), 1.5-10.0 ( 200MeV bins)    &  500 (27.8k)     \\ \hline
%$K^- $      &  0.2-1.5 (100MeV bins), 1.5-10.0 ( 200MeV bins)    &  500 (27.8k)     \\ \hline
%\end{tabular}\caption{Data sample requirements for the development of the reconstruction algorithms. The most important are  the low momenta particles where the showers are more likely to have different topologies. }
%\end{table}


\subsubsection{e/$\gamma$ separation}
\label{sec_egam}

The search for a CP violation phase using $\nu_e$ appearance 
in a $\nu_\mu$ beam requires good electron/photon separation.
Backgrounds originating from photons produced primarily from 
final state $\pi^0$'s must be identified and removed from the signal
electron sample. 

High energy photons can undergo two process: pair production and Compton scattering. 
The dominant process for photons with energies of several hundred MeV is 
e$^+$ e$^-$ pair production.
%energies. 
%For pair production the 
e/$\gamma$ discrimination
for this process can be achieved using the beginning of the electromagnetic shower, where 
a single MIP is characteristic of electron energy deposition while two MIPs is consistent 
with a photon hypothesis.
In the case where the photon Compton scatters, the two particles cannot easily be distinguished
from the energy deposition pattern.


\begin{figure}[h!]
  \centering
%\includegraphics[width=0.49\textwidth,height=5.0cm]{figures/pid}
%\includegraphics[width=0.49\textwidth,height=5.0cm]{figures/pid_probabilities}
\includegraphics[width=0.49\textwidth,height=6.0cm]{figures/eff-bgdrej-diffen}
\includegraphics[width=0.49\textwidth,height=6.0cm]{figures/APAT600_all_norm}
  \caption{
 (left) Background rejection versus signal efficiency for various energies, calculated for sample which pass the reconstruction. Simulation performed in APA design configuration (4.67~mm wire pitch, Induction wires at $\pm$35.7$^{\circ}$ w.r.t. Collection wires). (right) Background rejection versus signal efficiency
integrated over the energy range 0.2-1.0 GeV for two different readout designs: APA parameters as on the left plot, ICARUS parameters: 3~mm wire pitch, Induction wires at $\pm$60$^{\circ}$ w.r.t. Collection wires. 
Background rejection and signal selection efficiency values normalized to the full data sample size, including events which do not pass reconstruction.
%(Left)Background rejection versus signal efficiency for various energies.
%{\color{red}
%}
}
\label{fig:egam}
\end{figure}
Electron-photon separation has been studied in LAr TPCs
(ICARUS~\cite{icarus_eg} and ArgoNeuT~\cite{argoneut_eg}).
% numbers come from Dorotas plot docdb 10660 
%as shown in Fig.~\ref{fig:egam1}.
%Currently the 
%separation efficiency is estimated to be at the level of of 95 \% (?) 
Fig.~\ref{fig:egam} (from ~\cite{dunecdr})
 shows estimated background rejection versus efficiency using 
simulated isotropic electron and photon event samples  with the 
APA geometry configuration.
% and propagated thorough the 
%event reconstruction chain 
Background rejection as a function of signal
selection efficiency depends mildly
on incoming particle energy in the range of interest. 
Greater than 90\% background rejection can be 
achieved with $>$70\% efficiency for energies above 300~MeV. (Above 1~GeV 
rejection is even greater at high efficiency).
Our studies indicate that rejection and efficiency depend 
on particular features of the geometry including wire pitch and plane 
orientation which affect the reconstruction. 
For example, Fig.~\ref{fig:egam} (right) shows a comparison of background
rejection versus signal efficiency for simulated
DUNE APA configuration compared with with ICARUS T600 geometry.
%integrated over the energy range 0.2-1.0 GeV. 
Pure samples of electrons and photons in the sub-GeV energy range will be
needed to tune the separation algorithms and to measure 
detector-dependent electron-photon separation efficiency and purity.

%Therefore, it is critically important 
%to study e/$\gamma$ separation in a prototype LAr TPC detector.

%More discussion of specific samples to be used here.
% move to table section.





\subsection{Other measurements} 
\label{sec_other}

The DUNE experiment will be capable of addressing other physics beyond the flagship long-baseline neutrino measurements with the unique massive underground LAr TPC. Event samples discussed here include those that support this rich array of additional physics goals.
%The DUNE experiment will be capable of addressing other new physics possibilities with its one-of-a-kind
%massive underground LAr TPC. Event samples discussed here include those that support DUNE
%rich array of other physics goals. 


\subsubsection{Supernova and Michel electrons}

Neutrinos produced in supernova explosions since the first stars formed in the universe constitute the
diffuse relic supernova neutrino background. The DUNE Far detector could have unique sensitivity to these
few-to-30-MeV range $\nu_e$'s. 
The test beam cannot offer a sample of  such low energy electrons, but Michel electrons  produced by 
decaying stopped muons is ideal to calibrate electron response in the appropriate 10-50~MeV energy range. 
The requested sample of low energy muons will supply the 1400 Michel electrons required for a 1\% 
calibration of electrons in this energy range. 

%\begin{table}[h]
%\centering
%\begin{tabular}{|c|c|c|}
%\hline
%Particle     & Momenta (GeV/c)    & Exposure/bin  \\ \hline
%$\mu^+$   & (0.2), 0.5, 1      &  10K    \\ \hline
%\end{tabular}\caption{Stopping . }
%\end{table}

\subsubsection{Charge sign determination}

It is not possible to determine the charge of the particle on an event by event basis with non-magnetized LAr TPC detectors. A statistical separation will be studied which will make use of differences in muon versus anti-muon capture cross sections and lifetime.
%However, the statistical analyst will be possible. We will fit the muon's half time which is different for muons and antimony due to different muon capture cross sections. 
For the $\mu^-$ we expect about 99.9\% to be captured on argon whereas essentially all $\mu^+$ decay \cite{stopmu}.
Charge-sign tagged $\nu_\mu$ samples may be useful to constrain exotic possibilities such as
non-standard neutrino interactions which predict particle-antiparticle survival probability differences. 


\subsubsection{Proton decay sensitivity and background samples}


The DUNE experiment in the deep underground location will seek to detect several modes of proton decay.
In particular, a first ever LAr detector of this scale underground will primarily improve sensitivity to 
proton decays with final state kaons such as  $p \rightarrow K^+ \overline{\nu}$. 
Sensitivity to this process is studied in \cite{bueno}. $K^+$ detection efficiency is estimated to be $>$97\% in the
appropriate momentum range (500-800 MeV/c). The kaon samples requested in Table~\ref{tab:runsum} are needed to directly measure 
$K^+$ PID and detection efficiency. 

Obtaining low energy kaons will likely be difficult in this beamline.
A sample of 13K beam kaons with 1~GeV/c momentum are requested to provide 2K stopping $K^+$ track samples for PID studies.
(only 15\% of $K^+$ stop at 1~GeV/c).


%\begin{table}[h]
%\centering
%\begin{tabular}{|c|c|c|}
%\hline
%Particle     & Momenta (GeV/c) & Exposure/bin  \\ \hline
%\hline
%K$^+$  &  1 & (13k)    \\ \hline
%K$^+$  & 0.5, 0.7 & (5k)   \\ \hline
%proton &  1  &  (1M)  \\ \hline
%\end{tabular}\caption{Samples related to proton decay physics requirements.}
%\label{pdktable}
%\end{table}

A sizable sample of protons ($\sim 10^6$)
are requested to study the possible background contributions to  $p \rightarrow K^+ \overline{\nu}$.
This sample of  protons are needed to quantify the possibility that an interacting proton 
is  {\em mis-IDed} as stopping kaon. A proton interaction which produces neutrals and one charged pion 
(which is mis-IDed or subsequently decays to $\mu$) can fake the final state kaon signal.


\subsubsection{Anti-proton annihilation }

A sample of antiprotons would be useful to calibrate the $p$-$\overline{p}$ annihilation process. 
This would provide input to exotic baryon number-violating process in which neutron-antineutron oscillation
occurs and results in a
% (reference) modeling of 
subsequent  $n$-$\overline{n}$ annihilation. The modeling of $p\overline{p}$ annihilation could be
studied and tuned on a low-energy $\overline{p}$ beam sample and used to constrain the related modes 
expected for  $n$-$\overline{n}$ annihilation. These events would be tagged in the 
mixed-mode beam. Events in the sub-GeV range would be the most useful for this purpose.





	

\section{Single Phase LAr Detector [$\sim$10 pages; {\color{red} J. Stewart et al.}]}
\rm

	\label{singlephasedet}

\subsection{DUNE detector plans}

The far detector for the DUNE collaboration will be a series of four liquid argon time projection chambers (TPC), each in a cryostat that holds a fiducial/active/total LAr mass of 10.0/13.3/16.9~kt. The TPCs will be instrumented with photon detection. It is planned that the first 10~kt detector will be ready for installation in the 2022 timeframe. 
The design for the first 10~kt detector is a submerged wire plane-based TPC with electronic readout also in the liquid argon.  Designs of this style are referred to as single-phase detectors as the charge generation, drift, and detection all occurs in the liquid argon phase.  This style TPC features no charge amplification before collection, thereby making a very precise charge measurement possible. 


To achieve DUNE's goals, a detector is needed that is much larger than ICARUS, the largest LAr TPC detector built to date. The former long-baseline neutrino experiment (LBNE) developed a scalable far detector design shown in Fig.~\ref{fig:fardet-overview} that would scale-up LAr TPC technology by roughly a factor of 40 compared to the ICARUS T600 detector. To achieve this scale-up, a number of novel design elements need to be employed. A membrane cryostat typical for the liquefied natural gas industry will be used instead of a conventional evacuated cryostat.  The wire planes or anode plane assemblies (APAs) will be factory-built as planar modules that are then installed into the cryostat. The modular nature of the APAs allow the size of the detector to be scaled up to at least 40~kt fiducial mass. Both the analog and digital electronics will be mounted on the wire planes inside the cryostat in order to reduce the electronic noise, to avoid transporting analog signals large distances, and to reduce the number of cables that penetrate the cryostat. 

The scintillation photon detectors will employ light collection paddles to reduce the required photo-cathode area and thereby cost.  Designs being considered are also more compact than the photomuliplier tubes solution used elsewhere.

Many of the aspects of the design are being tested in a small scale prototype at Fermilab but given the very large scale of the detector elements a full-scale test is critical. 
Since the recent formation of the new DUNE collaboration a combined detector design team is emerging. 
Ideas from this new collaboration have the potential to modify the detector design for the additional three far detector modules
which are foreseen for DUNE.
The detector design described here is the LBNE detector design chosen by the DUNE collaboration as the reference design for the first 10~kt 
detector module and also adopted as the basis for the DUNE-PT.


\begin{figure}[!htb]
\centering
\begin{minipage}[b]{1.0\textwidth}
\begin{center}
\includegraphics[width=.75\textwidth]{figures/fardet-3D.png}
%\includegraphics[width=0.7\textwidth]{EndView-sketch.png}
\end{center}
\end{minipage}
\caption{\small 3D model of the design for the first DUNE single-phase detector. Shown is a 5~kt fiducial volume detector which would need to be lengthened for the 10~kt design. The present DUNE plan calls for the construction of four 10~kt detectors. }
\label{fig:fardet-overview} 
\end{figure}

The engineering goals of the single-phase APA/CPA detector test can be broken into five broad categories: 
\begin{enumerate}
	\item TPC performance, mechanical and electrical verification, 
	\item photon detection light yield verification,
	\item calibration strategy verification,
	\item argon contamination mitigation verification, 
	\item production and installation procedure verification
\end{enumerate}

	 The goals related to mechanical testing are to test the integrity of the detector. In the current design, each APA measures 2.3 m by 6.0 m and includes 2560 wires and associated readout channels. Given the complexity of these assemblies, a test where the detector can be thermally cycled and tested under operating conditions is highly advisable prior to mass production. The mechanical support of the APAs can be tested to verify that the mechanical design is reliable and will accommodate any necessary motion between the large wire planes. The impact of vibration isolation between the cryostat roof and the detector can also be tested. Finally, an improvement over existing cryostat designs is the possibility to move the pumps external to the main cryostat. This will reduce any mechanical coupling to the detector and also greatly improve both reliability and ease of repair.

	 
	 The electrical testing goals are to insure that the high voltage design is robust and that the required low electronic noise level can be achieved. As the detector scale increases so does the capacitance and the stored energy in the device. The design of the field cage and high voltage cathode planes needs to be such that HV discharge is unlikely and that if the event occurs no damage to the detector or cryostat results. The grounding and shielding of large detectors is also critical for low noise operation. By testing the full scale elements one insures that the grounding plan is fully developed and effective. Large scale tests of the resulting design will verify the electrical model of the detector. 

	 Research at Fermilab utilizing the Materials Test Stand~\cite{mat-test-stand} has shown that electronegative contamination to the ultra-pure argon from all materials tested is negligible if the material is immersed in the liquid argon. This implies that the dominant source of contamination originates from the gas ullage region and the room temperature connections to the detector. Careful design of the ullage region to insure that all surfaces and feedthroughs are cold is expected to greatly reduce the sources of contamination over what exists in present detectors. 
	 
\subsection{DUNE-PT}

%\subsubsection{Overview of the CERN Single-Phase test Detector}

This section presents the design details of a single-phase prototype detector based on the design by the former LBNE collaboration. 
The DUNE detector design is very modular and the DUNE-PT will be constructed from modular components of exactly the same design.

\begin{figure}[htb]
\centering
\begin{minipage}[b]{1.0\textwidth}
\begin{center}
\includegraphics[width=0.40\textwidth]{figures/CERN_single_TPC}
\includegraphics[width=.59\textwidth]{figures/TPC-3D-section.jpg}
\end{center}
\end{minipage}
\caption{\small 3D model of the CERN single-phase detector TPC (left) and inserted in the cryostat (right).}
\label{fig:CERNdet-overview}
\end{figure}

The TPC consists of alternating anode plane assemblies (APAs) and cathode plane assemblies (CPAs), with field-cage panels enclosing the four open sides between the anode and cathode planes.  Fig.~\ref{fig:CERNdet-overview} shows a sectioned view for the planned TPC 
by itself and inside the cryostat at CERN.  A uniform electric field is created in the volume between the anode and cathode planes. A charged particle traversing this volume leaves a trail of ionization. The electrons drift toward the anode plane, which is constructed from multiple layers of sense wires, inducing electric current signals in the front-end electronic circuits connected to the wires.

The TPC will be assembled from elements that are of the same size and materials as those planned for the first DUNE far detector module.  

The overall size of the TPC has been determined based on the desired particle containment in order to address the required physics measurements (see Section \ref{detbeamtest}). The TPC will have a 3-APA wide active volume and consists of two drift volumes with a drift length of 3.6~m each (see Fig.~\ref{fig:CERNdet-overview}).  
The APAs have an active (total) area measuring 2.29 m (2.32 m) wide and 5.9 m (6.2 m) high. The combination of the three APAs determines the overall TPC length to be 7.0~m. There will be a cathode plane (CPA) in the center between the two rows of APAs.  
The overall width of the TPC will be determined by a combination of the drift distances and the thicknesses of the two APA planes and the 
CPAs and amounts to 7.4~m.  
The overall height of the TPC is determined by the height of the APA which is 6.2~m.  The TPC dimensions are summarized in 
Table~\ref{table:TPC-dim}.
%
The minimum internal size of the cryostat is also indicated in Table \ref{table:TPC-dim} and was determined by adding the necessary mechanical and electrical clearances to the computed size of the TPC.  
 
\begin{table}[h]
\centering
\begin{tabular}{|c|c|}
\hline
\textbf{ Component } & dimensions [m]  \\ \hline \hline
APA  (active) &  $2.29 (wide) \times 5.9 (high)$ \\ \hline
APA  (external) &  $2.32 (wide) \times 6.2 (high)$ \\ \hline
TPC (active)       & $7.0 (long) \times 7.2 (wide) \times 5.9 (high)$  \\ \hline
TPC (external)       & $7.3 (long) \times 7.4 (wide) \times 6.2 (high)$  \\ \hline
cryostat (internal) &  $8.9 (long) \times 7.8 (wide) \times 8.1 (high)$  \\ \hline
\end{tabular}
\caption{Dimensions of DUNE-PT.}
\label{table:TPC-dim}
\end{table} 

Along with the APAs and CPAs, the TPC will include a field cage that surrounds the entire assembly to ensure a uniform drift field in the TPC's active volume. -

%This is a series of fiberglass I-beams for the structural elements.  These I-beams will be tiled with large copper sided FR4 panels to create the field cage.  Each panel will be connected with a series of resistors.  The field cage will also be connected to the CPAs through a capacitor assembly.

All of this will be supported by rows of I-beams supported from a mechanical structure above the cryostat.  The hangers for these I-beams will pass through the insulated top cap.  There will be a series of feedthrough flanges in the top cap of the cryostat to bring in and take out services for the TPC.  One HV feed-through is foreseen for the CPA row and one signal feed-through for each of the APAs.

The design also foresees the option to have the two APA rows mounted at 2.5~m from the central CPA each.
A reduced drift distance between the APA and CPA represents a deviation from the DUNE far detector design but is potentially very
useful in order to lessen the impact of space charge effects.
Due to the operation of DUNE-PT on the surface space charge 
 effects are expected to be larger than for underground operation at SURF.
We foresee to calibrate out any space charge effects for a 3.6~m drift distance using laser beam calibration and comics. 
At the same time we maintain the  possibility for a second test with reduced drift length if the uncertainties associated 
with our calibration limit the precision of measurements.
The cryostat would have to be emptied and the planes shifted to the 2.5~m drift distance.
 





%The plan is to have the CPA located in the center of the cryostat with APAs on each side near the walls of the cryostat membrane.  The above dimensions preserve the ability to reverse the order of the TPC rows by placing the CPA next to the wall of the cryostat and the APAs in the center.  However, this can only be done with the shorter 2.5m drift distance.  This reversed configuration at the 3.6m drift distance would place the CPAs too close to the membrane and risk high voltage discharge and and thereby possible damage to the membrane.  


% File from Bo Yu on the TPC component design
\input{APA-CP-FieldCage}

\subsubsection{Photon detection system}




The photon detection system (PDS) of the DUNE far detector will utilize liquid argon scintillation light to trigger on non-beam events and to
determine the prompt event time primarily for non-beam events but also for beam events. 
Timing information will be useful in determining the t$_0$ of cosmic rays, including those which 
overlap with beam events and events from radiological decays. This timing information allows proper spatial event reconstruction, including 
the reconstruction and rejection of background events.\\
%
While the TPC will have spatial resolution that is far superior to a photon detection system, there is no intrinsic precise determination of the event time and the drift time for TPC events can be up to milliseconds. The photon detection system can determine the start of an event occurring in the TPC volume (or entering the volume) to about 6 ns. For beam-triggered events the performance of the photon detection timing information can be crosschecked and evaluated during the detector beam test. In addition to providing trigger functionality, the PDS may be able to improve the event energy reconstruction by providing drift length dependent corrections to energy loss of the drifting electrons.
In the absence of an external electric field, a charged particle passing through liquid argon will produce about 44,000 photons ($\lambda$ = 128~nm) per MeV of deposited energy. 
At higher electric drift fields the number of photons will be smaller due to reduced recombination, but at 500 V/cm the yield is about 20,000 photons per MeV. Roughly one-third of the photons are prompt, 2-6~ns, and two-thirds are generated after a delay of 1.1-1.6~$\mu$s. LAr is highly transparent to the 128~nm VUV photons with a Rayleigh scattering length and absorption length of 55$\pm$5~cm \cite{rayleigh} and $>$200~cm \cite{absorption} respectively. The relatively large light yield makes the scintillation process an excellent candidate for triggering and  determination of t$_0$ for non-beam related events. Detection of the scintillation light may also be helpful in background rejection and possibly
provide improvements for energy reconstruction.

Several prototypes of photon detection systems for single phase liquid argon detectors have been developed by the former LBNE photon detector group over the past few years. There are currently three prototypes under consideration for use in the first module of the DUNE far detector, a baseline design along with two alternate designs. A decision on the design to be deployed in the CERN test will be made in late 2015. 
DUNE-PT provides the first full scale test of the photon detectors which will be fully integrated into a 
full scale TPC anode plane assembly. 

The present reference design for the photon detection system is based on acrylic bars that are 200~cm long and 7.6~cm wide, which are coated with a layer of tetraphenyl-butadiene (TPB). The wavelength shifter converts incoming VUV (128 nm) scintillation photons
%   with a conversion efficiency of ~50\% \cite{conversion-eff}
to longer wavelength photons which are characterized by an emission spectrum with a peak wavelength of 430~nm.
About 50\% of the converted photons will be emitted into the bar.
  A fraction of the wavelength-shifted optical photons are internally reflected to the bar's end where they are detected by SiPMs whose QE curve is well matched to the 430 nm wavelength-shifted photons. All PD prototypes are currently using SensL MicroFC-6K-35-SMT 6 mm $\times$ 6 mm devices \cite{sensl}. 

A full 6 m long APA will be divided into 5 bays with 2 PD modules (paddles) instrumenting each bay. The paddles will be inserted into the frames after the TPC wires have been strung allowing  final assembly at the CERN test location. Two alternative designs are also under consideration. 


One alternate design attempts to increase the geometrical acceptance of the photon detectors by using large acrylic TPB coated plates with imbedded WLS fibers for readout. In this design the number of required SiPMs and readout channels per unit detector area covered with photon detection panels would be significantly reduced to keep the overall cost for the photon detection system at or below the present design while increasing the geometrical acceptance at the same time. The prototype consists of a TPB-coated acrylic panel embedded with a S-shaped wavelength shifting (WLS) fiber. The fiber is read out by two SiPMs, coupled to either end of the fiber, and serves to transport the light over long distances with minimal attenuation. The double-ended fiber readout has the added benefit to provide some position dependence to the light generation along the panel by comparing relative signal sizes and arrival times in the two SiPMs. 



The third design under consideration was motivated by increasing the attenuation length of the PD paddles and allowing collection of 400 nm photons coming from anywhere in the active volume of the TPC.  The fiber-bundle design is based on a thin TPB coated acrylic radiator located in front of a close packed array of WLS fibers. This concept is designed so that roughly half of the photons converted in the radiator are incident on the bundle of fibers, the wavelength shifting fibers are Y11 UV/blue with a 4\% capture probability. The fibers are then read out using SiPMs at one end. The Y11  Kuraray fibers have mean absorption and emission wavelengths of about 440 nm and 480 nm respectively.  The attenuation length of the Y11 fibers is given to be greater than 3.5 m at the mean emission wavelength, which will allow production of full-scale (2 m length) photon detector paddles.


The PD system tested at the CERN neutrino platform will be based on technology selected later in 2015. The technology selection process will be based on a series of tests planned for 6 months utilizing large research cryostats at Fermilab and Colorado State University. The primary metric used for comparison between the three technologies will be photon yield per unit cost. In addition to this metric, PD threshold and reliability will also serve as inputs to the final decision. A technical panel will be assembled to make an unbiased decision. 



\subsubsection{TPC and PDS readout}
\input{readout}

\subsubsection{DAQ, Slow control and monitoring}
\input{daq}







\section{Cryostat and cryogenics system [$\sim$5 pages; {\color{red} David/Barry/Jack}]}
	\label{cryo}
\subsection{Cryostat}

%{\color{red} BALANCE BETWEEN CRYOSTAT AND CRYOGENICS SECTIONS MAY HAVE TO BE ADJUSTED ... ?!}

%\subsection{Cryostat size from TPC dimensions }

The single-phase detector test at CERN will use a membrane tank technology supported by an outer steel structure.

The minimum internal size of the cryostat is determined from size of the TPC.  At the bottom of the 
cryostat there needs to be a minimum of 0.36 m between the frame of the CPA and closest point on the stainless steel 
membrane.  This is to prevent high voltage discharge between the CPA and the electrically grounded 
membrane. It is foreseen that there would be some cryogenic piping and instrumentation under the TPC.  
There is a height allowance of 0.165 m for this.  There will be access and egress space around the outside 
of the TPC and the membrane walls.  On two sides, 0.15 m of space is reserved for this. The front will have 0.36 mm and the back, where piping and instrumentation for the cryogenic system will be located, 1.20 m.

The support system for the TPC will be located outside the cryostat roof, with a bridge connected to the floor. 
It is the same design solution currently foreseen for the DUNE single phase TPC in the LBNF cryostat.  
The plan is to model this space similar to what is planned for the far site TPC.  There 
will be 0.82 m of ullage space.  In order to prevent high voltage discharge, the upper most part of the CPA 
needs to be submerged a minimum of 0.5 m below the liquid argon surface.  The top of the TPC will be 
separated from the membrane by a minimum of 1.1 m.  

Adding all of these to the size of the TPC yields the minimum inner dimensions of the cryostat.  A 
minimally-sized cryostat would be 8.9~m long, 7.8~m wide and 8.1~m high.  This assumes the TPC will be 
positioned inside the cryostat with the CPAs and end field cages parallel to the walls of the cryostat. 
Figure~\ref{fig:fardet-overview} shows a 3D view of the detector inside the cryostat and Figure~\ref{fig:cryostat-views} shows side and end views of the cryostat, respectively. 
%
\begin{figure}
\begin{center}
\includegraphics[width=.53\textwidth]{figures/cryostat-side-view} 
\includegraphics[width=.455\textwidth]{figures/cryostat-westend-view}  
\caption[Views of cryostat]{Side (left) and end (right) views of cryostat}
\label{fig:cryostat-views} 
\end{center}
\end{figure}
%
The beam window design is still being worked on and may lead to minor modifications if
any of the current boundaries conditions listed above is violated.
  
%%%%%%%%%%%%%%%%%

The cryostat has an inner volume of 483 m$^3$ and can contain  673 tons of LAr.
%
%The design is based on a scaled up version of the LBNE 35-ton Prototype\cite{montanari_35ton} 
%% (this is a comment) add \cite{montanari_sbn_perf} if public access in docdb 9270 (AH)
%and the Fermilab Short-Baseline Near Detector\cite{acciarri_sbn_proposal}.  
%
The cryostat will use a steel outer supporting structure with a metal liner inside to isolate the insulation volume, similar to the one of the dual phase WA105 detector, the $1\times1\times3$~m$^3$ prototype and to the Fermilab Short-Baseline Near Detector. 
The support structure will rest on I-beams to allow for air circulation underneath in order to maintain the temperature within the allowable limits. The proposed design encompasses the following components:
%
\begin{itemize}
\item steel outer supporting structure,
\item main body of the membrane cryostat (sides and floor), 
\item top cap of the membrane cryostat.
\end{itemize}
%
A membrane cryostat design commonly used for liquefied natural gas (LNG) storage and transportation will be used. In this vessel a stainless steel membrane contains the liquid cryogen. The pressure loading of the liquid cryogen is transmitted through rigid foam insulation to the surrounding outer support structure, which provides external support. The membrane is corrugated to provide strain relief resulting from temperature related expansion and contraction. The vessel is completed with a top cap that uses the same technology.

Two membrane cryostat vendors are known: GTT (Gaztransport \& Technigaz) from France and IHI (Ishikawajima-Harima Heavy Industries) from Japan. Each one is technically capable of delivering a membrane cryostat that meets the design requirements for this detector. To provide clarity, only one vendor is represented in this document, GTT; this is for informational purposes only. Figure~\ref{fig:lar-org} shows a 3D model of the GTT membrane and insulation design.

\begin{figure}[htbp]
\begin{center}
\includegraphics[width=.9\textwidth]{figures/membrane-exploded-view}
\caption[Exploded view of the membrane cryostat technology]{ Exploded view of the membrane cryostat technology}
\label{fig:lar-org}
\end{center}
\end{figure}

To minimize the contamination from warm surfaces, during operation the temperature of all surfaces in the ullage shall be lower than 100~K. 
It has been observed in the Materials Test Stand (MTS) and the Liquid Argon Purity Demonstrator (LAPD) at Fermilab that the outgassing is significantly reduced below 100~K \cite{outgassing}. A possible way to achieve this requirement is to spray a mist of clean liquid and gaseous argon to the metal surfaces in the ullage and keep them cold, similar to the strategy that was developed for the cool down of the LBNE 35 Ton prototype.

The top plate will contain two hatches for the installation of the TPCs; it will also contain a manhole to enter the tank after closing the hatches, and several penetrations for the cryogenic system and the detector. 

%\begin{figure}[htb]
%\begin{center}
%\includegraphics[width=.75\textwidth]{figures/cryostat-isometric-view} 
%\caption[Isometric view of cryostat]{\label{fig:cryostat-views} Isometric view of the membrane cryostat}
%\end{center}
%\end{figure}

%\textbf{Design Parameters:}
%
The cryostat design for the CERN prototypes includes technical solutions that are of interested for the future needs of the DUNE program. For example the use of a cold ullage (\textless  100~K) to lower the impurities in the gas region, and of a LAr pump outside the cryostat to minimize the effect of noise, vibration and microphonics to the TPC inside the liquid argon volume.
%
The design parameters for the TPC Test at CERN cryostat are listed in Table~\ref{tbl:cryogenics-design-parameters}.

\begin{table}[htpb]
\centering
\begin{tabular}{|p{.4\textwidth}|p{.5\textwidth}|}
\hline
\textbf{Design Parameter} & \textbf{Value} \\ \hline
Type of structure & Membrane cryostat \\ \hline
Membrane material    &  SS 304/304L, 316/316L or equivalent. \\ \hline
Fluid & Liquid argon (LAr)  \\ \hline
Other materials upon approval.\\ \hline
 Outside reinforcement (support structure)  &  Steel enclosure with metal liner to isolate the outside from the insulation space, standing on legs to allow for air circulation underneath. \\ \hline
 Total cryostat volume  &  538 m$^3$ \\ \hline
 Total LAr volume  &  483 m$^3$ \\ \hline
LAr total mass   & 673,000 kg  \\ \hline
Minimum inner dimensions (flat plate to flat plate).   &  7.8 m (W) x 8.9 m (L) x 8.1 0 (H) \\ \hline
Depth of LAr   &  7.2 m (0.82 m ullage, same as LBNF) \\ \hline
Primary membrane   &   1.2 mm thick SS 304L corrugated stainless steel\\ \hline
Secondary barrier system   &  GTT design; 0.07 mm thick aluminum between fiberglass cloth. Overall thickness 1 mm located between insulation layers.  \\ \hline
 Insulation  &  Polyurethane foam (0.9 m thick from preliminary calculations) \\ \hline
Maximum static heat leak   &  10 W/m$^2$ \\ \hline
LAr temperature   & 88 +/- 1K  \\ \hline
Operating gas pressure   &  Positive pressure. Nominally 70 mbarg ($\sim$1 psig) \\ \hline
 Vaccuum  &  No vacuum \\ \hline
 Design pressure  &  350 mbarg ($\sim$5 psig) + LAr head (1,025 mbarg) \\ \hline
Design temperature   &  77 K (liquid nitrogen temperature for flexibility) \\ \hline
Temperature of all surfaces in the ullage during operation   & \textless 100 K  \\ \hline
Leak tightness   & $10^{-6}$ mbar*l/sec   \\ \hline
Maximum noise/vibration/microphonics inside the cryostat   & LAr pump outside the cryostat  \\ \hline
Beam window   & Precise location TBD. Figure~XXX shows the location where the beam enters the cryostat.  \\ \hline
 Accessibility after operations  & Capability to empty the cryostat in 30 days and access it in 60 days after the end of operations. \\ \hline
  Lifetime / Thermal cycles &  Consistent with liquid argon program. TBD. \\ \hline
 \end{tabular}
\caption{Design requirements for the membrane cryostat}
\label{tbl:cryogenics-design-parameters}
\end{table}

\textbf{Insulation system and secondary membrane: }
%
The membrane cryostat requires insulation applied to all internal surfaces of the outer support structure 
and roof in order to control the heat ingress and hence required refrigeration heat load. 
To avoid bubbling of the liquid argon inside the tank, the maximum static heat leak is 10 W/m$^2$ for the floor and the sides and 15 W/m$^2$ for the roof, higher to account for the penetrations that increase the heat budget. Preliminary calculations show that these values can be obtained using 0.9 m thick insulation panels of polyurethane foam.
Given an 
average thermal conductivity coefficient for the insulation material of 0.0283 W/(m$\cdot$K), the heat input 
from the surrounding steel is expected to be about 3.2~kW total. It assumes that the hatches are foam 
insulated as well. This is shown in Table~\ref{tbl:heat-load-calc}.

The insulation material is a solid reinforced polyurethane foam manufactured as composite panels. The 
panels get laid out in a grid with 3 cm gaps between them (that will be filled with fiberglass) and fixed 
onto anchor bolts anchored to the support structure. The composite panels contain the two layers of 
insulation with the secondary barrier in between. After positioning adjacent composite panels and filling 
the 3-cm gap, the secondary membrane is spliced together by epoxying an additional overlapping layer 
of secondary membrane over the joint. All seams are covered so that the secondary membrane is a 
continuous liner.

In the GTT design, the secondary membrane is comprised of a thin aluminum sheet and 
fiberglass cloth. The fiberglass-aluminum-fiberglass composite is very durable and flexible with an 
overall thickness of about 1 mm. The secondary membrane is placed within the insulation space. It 
surrounds the bottom and sides. In the unlikely event of an internal leak from the primary membrane of 
the cryostat into the insulation space, it will prevent the liquid cryogen from migrating all the way 
through to the steel support structure where it would degrade the insulation thermal performance and 
could possibly cause excessive thermal stress in the support structure. The liquid cryogen, in case of 
leakage through the inner (primary) membrane will escape to the insulation volume, which is purged with 
GAr at the rate of one volume exchange per day.

\begin{table}[htpb]
\centering
\begin{tabular}{|p{.15\textwidth}|p{.15\textwidth}|p{.15\textwidth}|p{.15\textwidth}|p{.15\textwidth}|}
\hline
 \textbf{Element} & \textbf{Area ($m^2$)}  &  \textbf{K ($W/mK$)} & \textbf{$\Delta$ T ($K$)}
 & \textbf{Heat Input ($W$)}\\ \hline
Base   & 83  & 0.0283   &205   & 534 \\ \hline
End walls  &  153 & 0.0283  &  205 &  986 \\ \hline
Side walls   & 172  & 0.0283  &  205 & 1,108 \\ \hline
Roof  &  83 & 0.0283  & 205  &  550\\ \hline
   &   &   &   &  \\ \hline
Total   &   &   &   & 3,162 \\ \hline
\end{tabular}
\caption{Heat load calculation for the membrane cryostat (insulation thickness=0.9~m). }
\label{tbl:heat-load-calc}
\end{table}


\textbf{Outer Support Structure:}
%
The proposed design is a steel support structure with a metal liner on the inside to isolate the insulation region and keep the moisture out. This choice allows natural and forced ventilation to maintain the temperature of the steel within its limit, without the need of heating elements and temperature sensors. 
The main body of the membrane cryostat does not have side openings for construction. The access is only from the top. There is a side penetration for the liquid argon pump for the purification of the cryogen.

%\textbf{Main body of the membrane cryostat:}
%
%The sides and bottom of the vessel constitute the main body of the membrane cryostat. They consist of several layers. From the inside to the outside the layers are stainless steel primary membrane, insulation, thin aluminum secondary membrane, more insulation, and steel outer support structure with meal panels acting as vapor barier. The secondary membrane contains the LAr in case of any primary membrane leaks and the vapor barrier prevents water ingress into the insulation. 


\textbf{Top cap:}
%
Several steel reinforced plates welded together constitute the top cap. The stainless steel primary 
membrane, intermediate insulation layers and vapor barrier continue across the top of the detector, 
providing a leak tight seal. The secondary barrier is not used nor required at the top. The cryostat roof is 
a removable steel truss structure that also supports the detector. Stiffened steel plates are welded to the 
underside of the truss to form a flat vapor barrier surface onto which the roof insulation attaches directly. 
The penetrations will be clustered in one region. The top cap will have two large openings for TPC 
installation, and a manhole to enter the tank  after the 
hatches have been closed.

The truss structure rests on top of the supporting structure where a positive structural connection 
between the two is made to resist the upward force caused by the slightly pressurized argon in the ullage 
space. The hydrostatic load of the LAr in the cryostat is carried by the floor and the sidewalls. In order to meet the maximum deflection of 3~mm between APA and CPA 
and to decouple the detector from possible sources of vibrations, the TPCs will be connected to an external bridge over the top of the plate supported on the floor of the building. Everything else within the cryostat %(TPC planes, 
(electronics, sensors, cryogenic and gas plumbing connections) is 
supported by the steel plates under the truss structure. All piping and electrical penetration into the 
interior of the cryostat are made through this top plate, primarily in the region of the penetrations to 
minimize the potential for leaks. Studs are welded to the underside of the top plate to bolt the insulation 
panels. Insulation plugs are inserted into the bolt-access holes after panels are mounted. The primary 
membrane panels are first tack-welded then fully welded to complete the inner cryostat volume.
%
Table~\ref{tbl:cryostat-top-parameters} presents the list of the design parameters for the top of the cryostat.
%
\begin{table}[htpb]
\centering
\begin{tabular}{|p{.3\textwidth}|p{.7\textwidth}|} % AH {|p{.4\textwidth}|p{.5\textwidth}|}
\hline
 \textbf{Design Parameter} & \textbf{Value} \\ \hline
 Configuration &  Removable metal plate reinforced with trusses/I-beams anchored to the membrane cryostat support structure. Contains multiple penetrations of various sizes and a manhole. Number, location and size of the penetrations TBD. The hatches shall be designed to be removable. If welded, provisions shall be made to allow for removal and re-welding six (6) times.\\ \hline
Plate/Trusses non-wet material  &  Steel if room temperature.
SS 304/304 or equivalent if at cryogenic temperature
\\ \hline
Wet material  & SS 304/304L, 316/316L or equivalent. 
Other materials upon approval.
 \\ \hline
 Fluid & Liquid argon (LAr) \\ \hline
Design pressure  & 350 mbarg (~5 psig) \\ \hline
Design temperature  & 77 K (liquid nitrogen temperature for flexibility) \\ \hline
Inner dimensions  & To match the cryostat \\ \hline
Maximum allowable roof deflection  & 0.003 m (differential between APA and CPA) \\ \hline
Maximum static heat leak  & \textless 15 W/m$^2$  \\ \hline
 Temperatures of all surfaces in the ullage during operation & \textless 100 K \\ \hline
Additional design loads  &  -	Top self-weight \\
 & -	Live load (488 kg/m$^2$)\\
& -	Electronics racks (400 kg in the vicinity of the feed through)\\
& -	Services (150 kg on every feed through)
\\ \hline
TPC anchors  & %Capacity: AH
Number and location TBD. Minimum 6.
 \\ \hline
 Hatch opening for TPCs installation &  3.550 m x 2.000 m (location TBD)\\ \hline % , to . in numbers
Grounding plate  &  1.6 mm thick copper sheet brazed to the bottom of the top plate\\ \hline
Lifting fixtures  & Appropriate for positioning the top at the different parts that constitute it. \\ \hline
Penetrations  &  1 LAr In, 1 Purge GAr In, 1 Vent GAr In \\ 
& 2 Pressure Safety Valves, 2 Vacuum Safety Valves \\ 
& 1 GAr boil off to condenser \\ 
& 1-2 Liquid level sensors \\ 
& 1-2 Instrumentation \\ 
& 1 Temperature sensors feedthroughs? \\ 
& 1 LAr for cool down, 1 GAr for cool down \\ 
& 1 TPC signal, 3 TPC feedthroughs \\
& 1 Photon Detector for APA (Cold) \\
& Calibration dc\\ \hline
Lifetime / Thermal cycles  & Consistent with the liquid argon program TBD. \\ \hline
\end{tabular}
\caption{Design parameters for the top of the cryostat}
\label{tbl:cryostat-top-parameters}
\end{table}

\textbf{Cryostat grounding and isolation requirements:}
%
The cryostat has to be grounded and electrically isolated from the building. 
Figure~\ref{fig:top-plate-gnd} shows the layout of the top plate grounding.
We list the grounding and isolation requirements for the cryostat. 

\begin{enumerate}
\item \textbf{Isolation}
	\begin{enumerate}
	\item The cryostat membrane and any supporting structure, whether it is a steel structure or a concrete and rebar pour, shall be isolated from any building metal or building rebar with a DC impedance greater than 300 k$\Omega$.
	\item All conductive piping penetrations through the cryostat shall have dielectric breaks prior to entering the cryostat and the top plate.
	\end{enumerate}

\item \textbf{Grounding}
	\begin{enumerate}
	\item The cryostat, or ``detector'' ground, shall be separated from the ``building'' ground.
	\item A safety ground network consisting of saturated inductors shall be used between detector ground and building ground.
	\end{enumerate}

\item \textbf{Top plate grounding}
	\begin{enumerate}
	\item The top grounding plate shall be electrically connected to the cryostat membrane by means of copper braid connections.
	   \begin{enumerate}
	   \item Each connection shall be at least 1.6 mm thick and 63.5 mm wide.
	   \item The length of each connection is required to be as short as possible.
	   \item The distance between one connection and the next one shall be no more than 1.25 m.
	   \item The layout can follow the profile of several pieces of insulation, but it shall be continuous.	
	   \item The DC impedance of the membrane to the top plate shall be less than 1 ohm.
	   \end{enumerate}
	\end{enumerate}
\end{enumerate}


\begin{figure}
\begin{center}
\includegraphics[width=1.0\textwidth]{figures/cryostat-top-plate-gnd} 
\caption[Top plate grounding layout]{\label{fig:top-plate-gnd}Top plate grounding layout}
\end{center}
\end{figure}

\textbf{Leakage quality assurance: }
%
The primary membrane will be subjected to several leak tests and weld remediation, as necessary. All 
(100\%) of the welds will be tested by an Ammonia colorimetric leak test (ASTM E1066-95) in which 
welds are painted with a reactive yellow paint before injecting a Nitrogen-Ammonia mixture into the 
insulation space of the tank. Wherever the paint turns purple or blue, a leak is present. The developer is 
removed, the weld fixed and the test is performed another time. Any and all leaks will be repaired. The 
test lasts a minimum of 20 hours and is sensitive enough to detect defects down to 0.003 mm in size 
and to a 10$^{-7}$ std-cm$^3/s$ leak rate (equivalent leak rate at standard pressure and temperature, 1 bar and 
273 K). To prevent infiltration of water vapor or oxygen through microscopic membrane leaks (below 
detection level) the insulation spaces will be continuously purged with gaseous argon to provide one 
volume exchange per day. The insulation space will be maintained at 70 mbar, slightly above 
atmospheric pressure. This space will be monitored for changes that might indicate a leak from the 
primary membrane. Pressure control devices and safety relief valves will be installed on the insulation 
space to ensure that the pressure does not exceed the operating pressure inside the tank. The purge gas 
will be recirculated by a blower, purified, and reused as purge gas. The purge system is not safety-
critical; an outage of the purge blower would have negligible impact on LAr purity.


%%%%%%%%%%%%%%%%%
\subsection{Cryogenic System}

Figure~\ref{fig:proposed-LN2-system} outlines the basic scheme of the LN2 supply system, which was 
proposed by CERN for the Short Baseline Program and found to be an appropriate solution for this 
detector as well. 
%
\begin{figure}[htb]
\begin{center}
\includegraphics[width=.95\textwidth]{figures/proposed-LN2-system} 
\caption[Schematic diagram for the proposed LN2 system.]{\label{fig:proposed-LN2-system}Schematic diagram for the proposed LN2 system}
\end{center}
\end{figure}
%
The experiment will rely on LN2 tankers for regular deliveries to a local dewar storage, 
which will be sized to provide several days of cooling capacity in the event of a delivery interruption. 
From the dewar storage the LN2 is then transferred to a distribution facility located in the experimental 
hall. It includes a small buffer volume and an LN2 pumping station that transfers the LN2 to the argon 
condenser and other services as needed. The low estimated heat leak of the vessel ($\sim$3.2~kW) and the 
location inside an above ground building allow for use of an open loop system typical of other 
installations operated at Fermilab (LAPD, LBNE 35 ton prototype, MicroBooNE) and at CERN. 
The main goal of the LN2 system is to provide cooling power for the argon condenser, the initial cool down of 
the vessel and the detector, and all other services as needed.
%
 Table~\ref{tbl:cryo-design-parameters} presents the list of 
requirements for the cryogenic system for the single phase detector test at CERN.
%
\begin{table}[htbp]
\centering
\begin{tabular}{|p{.45\textwidth}|p{.45\textwidth}|}
\hline
 \textbf{ Parameter} & \textbf{Value} \\ \hline
 Location & Preferably not in front of the cryostat (on the beam) \\ \hline
 Cooling Power & TBD based on the heat leak of the cryostat (estimated 3.4~kW), the cryo-piping and all other contributions (cryogenic pumps, etc.) \\ \hline
 Liquid argon purity in cryostat & 10 ms electron lifetime (30 ppt O2 equivalent) \\  \hline
 Gaseous argon piston purge rate of rise & 1.2 m/hr \\ \hline
 Membrane cool-down rate & From manufacturer \\  \hline
 TPCs cool-down rate & \textless40 K/hr,\textless10 K/m (vertically)
 \\ \hline
Mechanical load on TPC & The LAr or the gas pressure shall not apply a mechanical load to the TPC greater than 200 Pascal. \\ \hline
Nominal LAr purification flow rate (filling/ops) & 5.5 day/volume exchange \\ \hline
 Temperature of all surfaces in the ullage during operations & \textless100 K \\  \hline
 Gaseous argon purge within insulation & 1 volume change /day of the open space between insulation panels. \\ \hline
 Lifetime of the cryogenic system & Consistent with the LAr program. TBD. \\ \hline
\end{tabular}
\caption{Design requirements for the cryogenic system}
\label{tbl:cryo-design-parameters}
\end{table}
%


Figure~\ref{fig:proposed-LAr-system} shows a schematic diagram of the proposed liquid argon system. It is based on the design of the 
LBNE 35 ton prototype, the MicroBooNE detector systems and the plans for the DUNE Far Detector.
\begin{figure}
\begin{center}
\includegraphics[width=1.0\textwidth]{figures/proposed-LAr-system} 
\caption[Schematic diagram for the proposed LAr system]{\label{fig:proposed-LAr-system} Schematic diagram for the proposed LAr system}
\end{center}
\end{figure}
The main goal of the LAr system is to purge the cryostat prior to the start of the operations (with GAr in open 
and closed loop), cool down the cryostat and fill it with LAr. Then continuously purify the LAr and the boil 
off GAr to maintain the required purity which is measured by the detector in the form of an electron drift time measurement.

The LAr receiving facility includes a storage dewar and an ambient vaporizer to deliver LAr and GAr to the 
cryostat. The LAr goes through the liquid argon handling and purification system, whereas the GAr 
through the gaseous argon purification before entering the vessel.

The LAr purification system is currently equipped with a filter containing mol sieve and copper beads, and 
a regeneration loop to regenerate the filter itself. Filters containing Oxysorb and Hydrosorb rather than 
mol sieve and copper beads represent a proven alternate solutions.
Studies are ongoing to standardize the filtration scheme and select the optimal filter medium for all 
future generation detectors, including the CERN prototypes. 

During operation, an external LAr pump circulates the bulk of the cryogen through the LAr purification 
system. The boil off gas is first recondensed and then is sent to the LAr purification system before 
re-entering the vessel.





	
\section{Calibration}
	\label{calibration}
%\subsection{Calibration}
The design of the calibration system for DUNE-PT pursues two principal goals. The first is to calibrate the detector itself to
provide high quality data for successive data analysis and the second is to test and optimize the calibration tools themselves for future
use in the DUNE far detectors.

The accuracy of track and kinematics reconstruction and particle identification largely relies on the knowledge of the electric field map, purity map, and temperature inside the active detector volume. The electric field in the drift region, designed to be a uniform 500 V/cm, could vary at different locations due to sagged wires, misalignment and imperfections in the field cage. 


The electric field could also be distorted by the space charge.  Electrons and ions which are generated by the passing of high energy particles drift in opposite directions to anode and cathode planes, respectively. The electrons drift at about 1.6 mm/$\mu s$ and take 2.25 ms to travel the maximum distance between APAs and CPAs. The ions, drifting at $\sim$ 8$\times 10^{-6}$~mm/$\mu s$, can take up to 7.5 minutes to travel the same distance. \\
The surface cosmic ray flux entering the detector imposes a large space-charge effect to the detector. 
%The magnitude of the distortion of the electric field is expected to be on the same order as the microBooNE detector which is also on the surface.  An example of the electric field distortion simulated in the microBooNE detector is shown in Fig.~\ref{fig:space_charge}.
%\begin{figure}[h!]
%  \centering
%\includegraphics[width=0.49\textwidth,height=6.0cm]{figures/spacechargeEvsx} 
%\includegraphics[width=0.49\textwidth,height=6.0cm]{figures/spacechargeEvsy} 
%  \caption{Fractional deviation of the microBooNE drift field $E_{x}$ as a function of position due to space charge effects.  The figure on the left shows the variation as a function of the distance between the cathode (x = 0) and the anode (x = 2.5 m).  The figure on the right shows the variation as a function of the vertical position.  The plots are from \cite{ref:space_charge}.
%{\color{red}  combine e+e-, improve information content (y-axis, change colors, etc )
%}
%}
%\label{fig:space_charge}
%\end{figure}
Distortion of the electric field could result in centimeter levels of uncertainty in position reconstruction and can also make a difference in electron-ion recombination and light output. 

The temperature of liquid argon affects the electron drift velocity and has a gradient of about -1.9\%/K. Precision measurement of temperature could be easily achieved with commercial silicon diode sensors down to tens of milliKelvin. 

Purity of argon directly affects the amount of surviving electrons, which would be used to estimate the amount of energy deposited in that wire space. This energy deposition, dE/dx, is a key quantity for particle identification. Monitoring argon purity is also essential for the operation of the experiment.

Multiple means should be employed to measure the electric field, the purity of liquid argon and its temperature. 
The presently foreseen calibration equipment includes:
\begin{itemize}
\item Gas purity analyzers
\item Liquid purity monitors
\item Temperature sensors
\item Laser calibration system
\item Muon detector system
\end{itemize}

The gas purity analyzers for oxygen and H$_{2}$O are commercially available. These analyzers measure purity down to a few parts per billion (ppb). They are not directly measuring the liquid argon purity in the detector but can be installed outside of the detector and take samples from various points of the cryogenic system and thereby provide an overview of the detector system.

Liquid purity monitors are small size TPCs with light sources that generate electrons via photoelectric effect and electronics to read out the amount of electrons in the cathode plane and anode plane. The electron lifetime can be derived by comparing the number of surviving electrons with those generated. The electron lifetime is a direct estimator of the argon purity.
Multiple liquid purity monitors will be installed at a variety of locations (close to and far from the recirculation inlet) and at different heights.

The temperature of liquid argon varies as a function of the pressure of the detector by a little less than 1~K/psi. It therefore also varies 
as function of height in the liquid. Silicon diode sensors with accuracy as little as $\sim$ 20 mK will be installed at multiple locations in the detector.

To measure and calibrate electric field, purity, electron lifetime and drift speed in-situ, a laser system and a muon detector are foreseen. 
Both systems, each with particular pros and cons, use ionization-particle paths to measure the above quantities.\\
%
The {\bf laser calibration system} employs a high power ultra-violet (UV) laser to ionize the liquid argon. The 266 nm UV photons have energy of 4.66 eV. Three photons could ionize one argon molecule (ionization potential for liquid argon is 13.78 eV). 
A laser beam is directed into the TPC region via a steerable feedthrough, which allows reflection of the laser beam to various regions of the 
active detector volume. 
The laser energy is about 10 mJ, which corresponds to 10$^{16}$ UV photons, and has sufficient energy to produce 
a straight (long Rayleigh scattering length) and uniform ionization path. Due to the size of the laser beam, about 1 cm in diameter, 
the electrons are generated in a relatively large space, and the electron-ion recombination effect is negligible compared to the cosmic ray 
induced ionization. In the proposed TPC configuration with a CPA in the center and two APAs on the outsides, 
the laser feedthrough will be installed outside of the TPC region and in the same plane as the central CPA.
Therefore the laser beam can be directed to both CPA-APA regions. The field cage is designed with multiple slits to allow the laser 
beam to pass to the inside of the TPC. Position detection systems such as SiPMs will be installed on the other side of the field cage 
to measure the position of the laser beam.

The {\bf cosmic muon detector} system serves as an alternative tool to the laser calibration system. Cosmic rays with energies of a few GeV 
have nearly uniform dE/dx in liquid argon with about 2~MeV/cm and generate about 18,000 electrons. % in the 3 mm wire space. 
Given standard values quoted for the cosmic ray muon flux at sea level, we arrive at a number of roughly 200 incident muons per square meters per second.  Taking into account the dimensions of the TPC, we estimate the area of the top face of its rectangular volume to be just over 50~m$^{2}$, which means that the detector will be subject to $\sim10^{4}$ particles per second.  The full electron drift time in liquid argon for a 3.6~m drift length is 2.25~ms, and each readout window for an event will contain three drift time windows which will include data from drift windows before and after the event.  Based on the cosmic rate and the size of the readout window, there is expected to be on average $\sim$68 track segments on top of the actual beam event.  This estimate takes into account the readout of charge that may still be drifting from the window just prior to the triggered one and the loss of charge that is still drifting after the triggered window ends.    Muon detectors are planned to be installed to preferentially measure cosmic rays passing nearly horizontally through the cryostat.
Additional muon counters on top of the cryostat will allow tagging of highly inclined muons and veto of vertical muons. 
  A disadvantage of using muons as a calibration tool compared to a laser beam is that muons could scatter multiple times in passing 
  through more than 7 meters of liquid argon and will therefore not be perfectly straight anymore.
  


\newpage
\section{Charged Particle Test Beam Requirements [$\sim$10 pages; {\color{red} Cheng-Ju}]}
	\label{testbeamreq}
\subsection{Particle Beam Requirements}
The requested beam parameters are driven by the requirement that the results from the CERN test beam should be directly applicable to the future large underground single-phase LAr detector with minimal extrapolation. The CERN test beam data will be used to evaluate the detector performance, to understand the various physics systematic effects, and to provide ``neutrino-like'' data for event reconstruction studies. To satisfy the requirement, the beam parameters must span a broad range of particle spectrum that are expected in the future neutrino experiment. The particle beam composition should consist of electrons, muons, and hadron beams that are charge-selected. The expected momentum distributions for secondary particles from neutrino interactions are shown in Figure~\ref{fig:particle_momenta}. There is a large spread in the momentum distribution with most particles peaked near 200 MeV/c. To cover the momentum range of interest, the momentum of the test beam should step from 0.2 GeV/c up to 10 GeV/c. 

The maximum electron drift time in the TPC is about 2.2 ms. So, to minimize pile-up in the TPC, the planned beam rate should be around 200 Hz.  Since the single-phase TPC consists of two drift volumes, it is desirable to aim the particle beam so that the hadronic showers are mostly contained in the same drift volume.  The nominal plan is to have the beam enter the cryostat slightly downward at an angle of about 6 degrees. This angle will roughly match the angle at which the neutrino beam enters the liquid argon cryostat at the far detector for the DUNE experiment. Along the horizontal plane, the beam should enter the cryostat with an angle of about 10 degrees to avoid pointing the beam perpendicular to the electron drift in the TPC.  We also plan to take some data with the beam entering a different region of the TPC, and may include some data with particles crossing one drift volume to the next.  The summary of the beam requirements are shown in Table~\ref{table:beamspecs}.
%The two beam entry angles and positions with respect to the LAr cryostat are shown in Figures \ref{fig:BP_SideView} and \ref{fig:BP_TopView}. 

%\begin{figure}[h!]
%  \centering
%\includegraphics[scale=0.5]{figures/True_Momenta_per_Particle.png}
%  \caption{Particle momenta distributions for particles coming from all fluxes ($\nu_e$, $\nu_\mu$, $\bar \nu_e$ and $\bar \nu_\mu$) at both near and far detector locations.  }
%  \label{fig:particle_momentav2}
%\end{figure}

%\begin{figure}[h]
%  \centering
%\includegraphics[scale=0.6]{figures/BeamPos_SideView.pdf}
%  \caption{Side view: beam enters the cryostat slightly downward with a dip angle of 6 degrees.  }
%  \label{fig:BP_SideView}
%\end{figure}
%
%\begin{figure}[h]
%  \centering
%\includegraphics[scale=0.6]{figures/BeamPos_TopView.pdf}
%  \caption{Top view: beam enters the cryostat with an entry angle of about 10 degrees along the horizontal plane. The primary orientation sends the particle beam into one TPC drift volume. The secondary orientation sends the particle beam across the APA. }
%  \label{fig:BP_TopView}
%\end{figure}

\begin{table}[h]
\centering
\begin{tabular}{|c|c|}
\hline
\textbf{Parameter } & \textbf{Requirements}  \\ \hline
  Particle Types        & $e^\pm,\mu^\pm,\pi^\pm$,$K$,$p$  \\ \hline
  Momentum Range   & 0.2 - 10 GeV/$c$ \\ \hline
  Momentum Spread   & $\Delta p/p  < $5 \% \\
  & (limited by the aperture of the magnets)  \\ \hline
  Transverse Beam Size   & RMS(x,y) $\approx$ 10 cm  \\
  & (At the entrance face of the LAr cryostat) \\ \hline
  Beam Divergence & tbd   \\ \hline
  Beam Angle &  $\approx$10$^{\circ}$ \\
  (horizontal plane) &  (w.r.t. the long axis of the cryostat)\\ \hline
  Beam Dip Angle &  $\approx$6$^\circ$ (downward from horizontal)   \\ 
  (vertical plane) &  \\ \hline
  Beam Entrance Position & Multiple beam windows    \\ \hline
  Rates & 200 Hz (maximum)    \\ \hline
\end{tabular}
\caption{Particle beam requirements.}
\label{table:beamspecs}
\end{table}

\subsection{EHN1 H4ext Beamline and Beam Instrumentation}
The H4ext is an extension of the existing H4 beamline in Experimental Hall North 1 (EHN1).  To produce particles in the momentum range of interest, a 60-80 GeV/c pion beam from the T2 target is used to generate tertiary beams. The tertiary particles are momentum and charge-selected and transported down H4ext beamline to the experimental area. A conceptual layout of the H4ext beamline is shown in Figure~\ref{fig:H4extPrelim}.  The top plot in Figure~\ref{fig:H4extPrelim} is the bird's eye view and the bottom plot is the side view of the beam line layout.
%In the Figure, the cryostat for this proposal is located at the lower right hand corner of the plot (in the H4 beam line). The cryostat in the H2 beam line is for the WA105 Collaboration.

\begin{figure}[h]
  \centering
\includegraphics[scale=0.62]{figures/EHN1_H4ext.pdf}
  \caption{A conceptual layout of the H4ext beamline. The top plot is the bird's eye view and the bottom plot is the side view of the H4ext beam line in EHN1. The liquid argon cryostat for this proposal is the rectangular box in the right hand corner of both plots.  The beam nominally enters the TPC at an angle of about 10$^\circ$ horizontally (top plot). On the vertical plane (bottom plot), the beam enters the cryostat downward with an angle of about 6$^\circ$. Multiple beam windows will be installed on the upstream wall of the cryostat to allow the beam to enter the TPC from different positions and angles.}
  \label{fig:H4extPrelim}
\end{figure}

%\subsubsection{Beam Optics}
%[Waiting for inputs from Ilias]

%\subsection{Beam Instrumentation}
\label{beaminstrument}
Beam instrumentation provides important information about the characteristics of the beam. It is expected that a series of detectors will be installed along the beam line to measure the particle momentum, identify particle type, and track the particle trajectory.

\subsubsection{Beam Position Detector}
The beam position detector measures the positions of the particle as it traverses the detector. Two detector technologies are under considerations: wire chambers and scintillating fiber trackers. For the nominal setup, one beam position detector is installed upstream and another one downstream of the last bending magnet. This pair provides additional momentum information about the particles as well as the first set of position measurements. Without tracking information, the momentum spread of the beam is expected to be at around 5\% based on the acceptance of the dipole magnets. With additional tracking information, we are likely to be able to measure the momentum of the individual particles to close to one percent level. A third beam position detector is placed right in front of the beam window on the cryostat wall to provide the last position information before the beam enters the cryostat.

\subsubsection{Particle Identification}
In order to have good particle identification over large momentum range, two independent particle identification systems are needed in the beamline. The Time-of-Flight system will be used to cover the lower momentum range while a Threshold Cherenkov detector will be tuned for higher momentum particles. We will require $\geq$ 3$\sigma$ $K$/$\pi$ separation for momentum range from 0.2 to a few GeV/c. Work is in progress to better define the beamline layout to meet the requirements.

\subsubsection{Muon Beam Halo Counters}
The muon beam halo counters is a set of detectors (e.g. plastic scintillator paddles) surrounding the beamline. The main purpose is to tag particles (primarily muons from the upstream production target) that are outside of the beam axis, but may potentially enter the TPC volume. The counter information is used to either veto or simply flag these class of events. The Muon Beam Halo counter can be a subset of the cosmic muon veto system. 

%\subsection{Beam Window on LAr Cryostat}
%This section could be absorbed into the cryostat chapter.


\subsection{Beam Rates and Run Plan}
At the time of this proposal, the beamline design has not been finalized. To estimate the beam rates, we use inputs from a generic target simulation based on 100k 80 GeV $\pi^+$ beam on 15 cm copper target. This $\pi^+$ rate is roughly equivalent to 10\% of a typical SPS spill. The distributions of the tertiary particles from the copper target are shown in Figure~\ref{fig:PionOnCuTarget}. The figure on the left is for postively charged and the figure on the right is for negatively charged tertiary particles. 

\begin{figure}[tbh]
  \centering
\includegraphics[scale=0.47]{figures/80GeVPion-15cmCuTarget.jpg}
  \caption{Simulation of 100k 80 GeV $\pi^+$ on 15 cm copper target. The figure on the left is for positively charged and the figure on the right is for negatively charged secondary particles from the target. }
\label{fig:PionOnCuTarget}
\end{figure}

To formulate a preliminary beam time request, we assume the hadron beam rates and spectrum as given in Figure~\ref{fig:PionOnCuTarget}. For the beam rate estimates, we account for particle decays assuming the distance from the secondary target to the cryostat to be 30 m. A significant fraction of pions and kaons below 1 GeV/c decay before reaching the liquid argon cryostat. In addition, we also assume the data taking efficiency to be about 50\%. For the electron sample, we expect a different optimized beamline setup to produce a pure electron beam with a flux of about 200 Hz. A preliminary run plan for one configuration is shown in Table~\ref{tab:RunPlan}. 
%
\begin{table}[p]
\centering
\rowcolors{0}{gray!30}{gray!30}
\begin{tabular}{|c|c|c|c|c|c|c|}
\hline
\multicolumn{7}{|c|}{\bf Positive Sample} \\ \hline
\bf $P$ & \bf $\#$ of Spills & \bf Time & \bf $\#$ of $\pi^+$ & \bf$\#$ of $\mu^+$ & \bf$\#$ of $K^+$ & \bf$\#$ of p \\ 
\bf (GeV)& & \bf (hours) & & & & \\ \hline
\hiderowcolors
0.2&900 &11&\textcolor{red}{\bf 15k} &180k&$\approx$0&160k\\ 
0.3&200 &3 &\textcolor{red}{\bf 15k} &30k &$\approx$0&50k \\
0.4&150 &\textcolor{red}{\bf 2} &22k &18k &$\approx$0&32k \\ 
0.5&150 &\textcolor{red}{\bf 2} &26k &12k &$\approx$0&38k \\
0.7&150 &\textcolor{red}{\bf 2} &40k &10k &$\approx$0&45k \\
1  &350 &4 &120k&\textcolor{red}{\bf 10k} &$\approx$0&65k \\
2  &600 &8 &320k&\textcolor{red}{\bf 10k} &3k        &130k\\
3  &500 &6 &290k &\textcolor{red}{\bf 5k} &7k        &70k \\
5  &1800&23& 1M &\textcolor{red}{\bf 5k}  &5k        &270k\\
7  &1200&15&660k&\textcolor{red}{\bf 6k}  &3k        &120k\\ \hline
Total &6000&76&2.5M  & 286k  &18k   & 1M\\
\hline \hline
\multicolumn{7}{|c|}{\bf Negative Sample} \\ \hline
\showrowcolors 
\bf $P$ & \bf $\#$ of Spills & \bf Time & 
\multicolumn{2}{|>{\columncolor[gray]{0.83}}c|}{\bf $\#$ of $\pi^-$ }& 
\multicolumn{2}{|>{\columncolor[gray]{0.83}}c|}{\bf$\#$ of $\mu^-$ } \\ 
\bf (GeV)& & \bf (hours) & 
\multicolumn{2}{|>{\columncolor[gray]{0.83}}c|}{}& 
\multicolumn{2}{|>{\columncolor[gray]{0.83}}c|}{} \\ \hline  
\hiderowcolors
0.2&600&8&\multicolumn{2}{|c|}{\textcolor{red}{\bf 15k}} &\multicolumn{2}{|c|}{88k}\\
0.3&200&3&\multicolumn{2}{|c|}{\textcolor{red}{\bf 15k}} &\multicolumn{2}{|c|}{30k}\\
0.4&150&\textcolor{red}{\bf 2}&\multicolumn{2}{|c|}{30k} &\multicolumn{2}{|c|}{18k}\\
0.5&150&\textcolor{red}{\bf 2}&\multicolumn{2}{|c|}{40k} &\multicolumn{2}{|c|}{13k}\\
0.7&150&\textcolor{red}{\bf 2}&\multicolumn{2}{|c|}{50k} &\multicolumn{2}{|c|}{12k}\\
1  &150&\textcolor{red}{\bf 2}&\multicolumn{2}{|c|}{70k} &\multicolumn{2}{|c|}{12k}\\
2  &200&3&\multicolumn{2}{|c|}{135k}&\multicolumn{2}{|c|}{\textcolor{red}{\bf 6k}}\\ \hline
Total  &1600&22&\multicolumn{2}{|c|}{350k}&\multicolumn{2}{|c|}{180k}\\ 
\hline 
\hline
\multicolumn{7}{|c|}{\bf Electron Sample} \\ \hline
\showrowcolors 
\multicolumn{3}{|>{\columncolor[gray]{0.83}}c|}{\bf $P$} &\bf $\#$ of Spills&\multicolumn{2}{|>{\columncolor[gray]{0.83}}c|}{\bf Time}&{\bf $\#$ of electron }\\
\multicolumn{3}{|>{\columncolor[gray]{0.83}}c|}{\bf (GeV)} & &\multicolumn{2}{|>{\columncolor[gray]{0.83}}c|}{\bf (hours)}&\\
\hline
\hiderowcolors
\multicolumn{3}{|c|}{0.2,0.3,0.4,0.5,0.7,1,2,3,5,7}  & 150 per bin & \multicolumn{2}{|c|}{2 hours per bin} &{140k per bin} \\ \hline
\multicolumn{3}{|c|}{Total}  & 1500 & \multicolumn{2}{|c|}{20} &{1.4M} \\ \hline
\end{tabular}
\caption{A preliminary run plan for one beam angle and position. The number of spills needed for a given momentum bin is driven by the samples highlighted in red or by the requirement of at least 150 spills per momentum bin.}
\label{tab:RunPlan}
\end{table}
The number of spills needed for each momentum bin is driven by the samples highlighted in red. The minimum beam time requirement is 150 spills ($\approx$ 2 hours of beam time) per momentum bin to ensure we have sufficient data taken with stable beam running.  This proposed plan satisfies the requested samples as listed in Table~\ref{tab:runsum}, except for kaons less than 1 GeV. Most low momentum kaons produced from the secondary target decay before reaching the liquid argon cryostat. To obtain those samples, we will need to carry out extended runs and trigger exclusively on particles tagged as kaons by either the time-of-flight or the threshold Cherenkov counters.

In addition to the above samples with beam at the nominal position, we expect to take some additional data with the beam entering the TPC at different position and angles. Without a detail beamline design, there are still some uncertainties on the actual beam rates. We are working on estimating the amount of beam time required. Based on the current information that we have, the total estimated beam time needed to carry out the physics program (physics samples as listed in Table~\ref{tab:runsum}, multiple angles and beam positions, dedicated kaon runs, etc) in this proposal is in the range of 4 to 6 weeks.
 



\section{Computing requirements, data handling and software  [$\sim$3 pages; {\color{red} Maxim/Craig}]}
	\label{computing}
% moved to main doc \section{Computing requirements, data handling and software}

The CERN single phase prototype detector builds upon the technology and expertise developed in the
process of design and operation of its smaller predecessor, the 35t detector at Fermilab.
This includes elements of front-end electronics, data acquisition, run controls and related systems. We also expect that for the most part,
Monte Carlo studies necessary to support this program will be conducted utilizing software evolved from tools currently used (2015). Likewise,
event reconstruction software and analysis tools will rely on the evolving software tools developed for DUNE.

The volume of the recorded data  will depend on the number of events to be collected in each measurement,
as specified in the run plan (see table \ref{tab:RunPlan}).
Cosmic ray muons have a very large impact on the data volume due to the large detector dimensions and surface operation.

It is optimal to first stage the data collected from the prototype on disk at CERN and then save it to tape also at CERN,
while simultaneously performing replication to data centers in the US. For the latter, Fermilab will be the primary site, with additional data centers at Brookhaven National Laboratory and the NERSC facility as useful additional locations for better redundancy and more efficient access to the data from a greater number of locations.



\subsubsection{Cosmic ray muons and readout window}
\label{readout_windows}

Given standard values quoted for the cosmic ray muon flux at sea level, we arrive at a number of roughly 200 incident muons
per square meters per second.
Taking into account the dimensions of the TPC, we estimate the area of the top face of its rectangular volume to be just over 50m$^{2}$,
which means that the detector will be subject to $\sim10^{4}$ particles per second.
Since the full electron drift time in liquid argon for a 3.6~m drift length is 2.25~ms, each readout window will contain 
$\sim$45 track segments on top of the actual beam event. These track segments may be from tracks that were initiated in the triggered
readout window or the be left drifting charges from the window just prior to the triggered one.


Background tracks due to cosmic ray particles must be properly identified and accounted for, in order to ensure high quality of the measurements and subsequent detector characterization. Since overlay of cosmic
ray muons over beam events is stochastic in nature, the optimal way to achieve this is by recording signals which were 
produced ``just before'' and ``just after'' the arrival of the test particle from the beam line. 
It will be possible because the design of the DAQ contains buffer memory that
can be accessed after the trigger decision is made. 
This technique will enable us to record and reconstruct either partial or complete background
tracks present in the ``main'' event.


\subsection{Event size estimate and data volume}

The data volume will dominated by the TPC data. Even though the photon detector as well as  other elements of the experimental apparatus 
(muon counters, trigger systems) contribute to the data stream their contributions to the data volume are expected to be sufficiently small 
such that realistic data volume estimates can be obtained from the TPC event data sizes alone.

Event sizes  can be estimated from "first principles" under the assumption of some event topology.
Based on the track range and the number of needed samples to capture the track, a given drift velocity and sample rate
a generous over-estimate is that a 1 GeV MIP needs some 20~kbyte. Then 5GeV MIP needs no more than 100~kbyte.
Showering events will require less data for the same energy deposition
due to some portion of the activity overlapping in the same voxels.
The estimate assumes that all particles are minimum ionizing so this is
another source of over-estimation.
The estimate is based on signals above the threshold of zero-suppression and neglects
radioactivity or assumes that signals from radioactivity (predominantly from $^{39}Ar$) are below the zero-suppression threshold.
This basic estimate serves to set the scale but does neglect channel overhead information that may be useful in 
interpreting the saved data.\\

In the LArSoft framework which is presently being used for the 35t detector
data are zero suppressed (ZS). 
Our assumption is that the nominal readout policy for the bulk of the data
will be to use ZS in order to follow the plan for the DUNE FD.
However, even for 
ZS data LArSoft saves 2 bytes for every channel even if that channel is
ZS'ed away. 

The data event size can be calculated as
\begin{equation}
  Event size = (\#channels) \times (clock\, rate) \times (readout \, time) \times (sample \, size)
\end{equation}
where the $\# channels$ equals 15,360 channels, corresponding to the proposed TPC which consists of 6 APAs with 2560 channels each, 
the $clock \, rate$ is 2~MHz, the $readout \, time$ is assumed to be 3 $ \times $ 2.25 ms = 6.75 ms corresponding to three drift times (see below) and the $sample \, size$ is 2~bytes.

{\color{red} This results in an event size of $\sim$ 400~Mbyte.\\
HOW WERE THE 2.5 MB CALCULATED ??! EMAIL FROM BRETT IN RESPONSE TO QIUGUANG AND DATED MAY 5TH INDICATED THAT 2 BYTES PER CHANNEL WERE ASSUMED ?!\\
FOR NOW WILL MOVE FORWARD WITH 2.5MB\\}
%

Based on these assumptions which leave room for optimization
we arrive at an event size of 2.5MB/event which is ZS but uncompressed.
With compression event sizes are expected to reduce to 0.1MB/event (ZS + compressed).
%
%Since each sample is 16 bit (or 12 bit in more recent design), we arrive to the limit of approximately 20MB per single charged track.
%For this class of events, the amount of data will scale roughly linearly with the length of the track, i.e. in cases when a track is
%stopped or leaves the sensitive volume there will be less data. 
%
%Further, in most cases the data will be zero-suppressed by the front-end
%electronics (e.g signals below a certain threshold
%will not be included into the outgoing data stream). 
%The exact data reduction factor will depend on a variety of factors (cf. threshold, which is yet to be chosen), but as a rule of
%thumb it's an order of magnitude. \textit{We conclude therefore the events will typically be a few megabytes in size}. 
The estimate is supported by Monte Carlo studies.


In view of the factors due to the cosmic ray muons presented above, the actual beam particle event data will represent but a fraction of
the total volume being read out.  
%As a concrete example, for an incident
%electron of 4GeV/c momentum MC calculations indicate an average event size of $\sim$2MB, after zero-suppression.
%This is less than 1\% of the estimate quoted above. 

Total statistics resulting from the run plan presented in section~\ref{tab:RunPlan} is approximately 25M events total, 
split across different particle species and incident momentum bins. Taking into account the 2.5~MB data load per event, 
this leads to the estimate of YYY~PB of nominal data volume to be collected in this experiment. 

In summary, we expect that tape storage of $O(PB)$ size will be required
and a somewhat more modest disk space for raw data staging at CERN, for replication purposes. 
We envisage storing the primary copy of raw data at CERN, with replicas at additional locations. 

Processed and Monte Carlo data placement will require additional resources that are addressed in section \ref{data process}.


\subsubsection{Data Transmission and Distribution}
Moving data to remote locations outside of CERN is subject to a number of requirements that include
automation, monitoring and error checking and recovery. 

A number of candidate systems satisfy these requirements, and one of them where we possess sufficient expertise and experience 
is Spade, first used in IceCube~\cite{spade_icecube} and then enhanced and successfully utilized in the Daya Bay experiment~\cite{spade_dayabay}.


\subsection{Databases}
Databases will be required to store Run Logs, Slow Controls records and detector conditions, as well as (offline) calibration information.

Most database servers will need to be local to the experiment (i.e. at CERN) in order to reduce latency, guarantee reliability, minimize
downtime due to network outages etc. A replication mechanism is foreseen to make data readily available at the US and other sites.
The volume of data stored in these databases is expected to be modest and of the order of YYY B.

\subsection{Computing and Software}

%\subsubsection{Distributed Processing}
%\label{distr_proc}

Fermilab provides the bulk of computational power to DUNE via Fermigrid and other facilities. 
We plan to leverage these resources to process the data coming from the test.

One of the principal goals will be quick validation of the data collected in each measurement, in
order to be able to make adjustments during the run as necessary. 
This is common practice in other experiment which have "express streams" to assess data quality~\cite{atlas_express}.


Given that tracking, reconstruction and other algorithms are in a stage of development with significant improvements
and optimizations expected, the required scale of CPU power needed to process the data are rough estimates.
The estimates we have at this point range from 10 to 100 seconds required by a typical
CPU to reconstruct a single event. 
This means that utilizing a few thousand cores through Grid facilities, it will be possible to ensure timely processing of these data.

To ensure adequate capacity, we envisage a distributed computing model where Grid resources are utilized in addition to Fermilab
As an example, we have had good experience working with the Open Science Grid Consortium.


\subsubsection{Data processing}
\label{dataprocess}

In addition to the raw data preparations are being made for offline data handling, processing and storage.
The offline data can be classified as follows:
\begin{itemize}
\item Monte Carlo data, which will contain multiple event samples to cover various event types and other conditions during the measurements with the prototype detector
\item Data derived from Monte Carlo events, and produced with a variety of tracking and pattern recognition algorithms in order to create a basis for the detector characterization
\item Intermediate calibration files, derived from calibration data
\item Processed experimental data, which will likely exist in several parallel branches corresponding to a few reconstruction algorithms being applied, with the purpose of their evaluation.
\end{itemize}

In the latter, there will likely be more than one processing step, thus multiplying data volume. 

The derived data will at most contain a small fraction of the raw data in order to keep the data manageable.
Hence the size of the processed data will likely by significantly smaller than the input (the raw data). 
Given consideration presented above, we will plan for
$\sim$$ O($1PB$)$ of tape storage to keep the processed data. 
For efficient processing, disk storage will be necessary
to stage a considerable portion of both raw data (inputs) and one or a few steps in processing (outputs).

Extrapolating from our previous experience running Monte Carlo for the former LBNE Far Detector, we estimate that we'll need a few hundred TB of continuously available
disk space. In summary, we expect the need for 5~PB of disk storage at Fermilab to ensure optimal data availability and 
processing efficiency. 

\subsubsection{Data distribution}
We foresee that data analysis (both experimental data and Monte Carlo) will be performed by collaborators residing in many 
institutions and geographically dispersed. In our
estimated above, we mostly outlined storage space requirements for major data centers like CERN and FNAL. When it comes to making these data available to the researchers,
we will utilize a combination of the following:
\begin{itemize}
\item Managed replication of data in bulk, performed with tools like Spade discussed above. Copies will be made according to wishes and capabilities of participating institutions.
\item Network-centric federated storage, based on XRootD. This allows for agile, just-in-time delivery of data to worker nodes and workstations over the network. This
technology has been evolving rapidly in the past few years, and solutions have been found to mitigate performance penalty due to remote data access, by implementing caching
and other techniques.
\end{itemize}

In order to act on the latter item, we plan to implement a global XRootD redirector, which will make it possible to transparently access data from anywhere.
A concrete technical feature of storage at FNAL is that there is a dCache network running at this facility, with substantial capacity which can be leveraged
for the needs of the CERN prototype analysis. This dCache instance is equipped with a XRootD ``door'' which makes it accessible to outside world, subject
to proper configuration, authentication and authorization.


Copies for a significant portion of raw and derived data are planed to be hosted at NERSC and also at Brookhaven National Laboratory.
These two institutions have substantial expertise  in the field of data handling and processing at scale and will serve as ``hubs'' for data archival and distribution.


\subsubsection{Software Infrastructure}

The CERN prototype effort will benefit from utilizing simulation toolkits, tracking and other reconstruction
that have and continue to be developed for DUNE, the 35t detector test and the short baseline program at Fermilab as well as the 
neutrino platform development efforts and in particular the WA105 experiment.

The software tools will need to be portable, well maintained and validated. To ensure that this happens,
we plan to establish close cooperation among participating laboratories and other research institutions.



%\end{document}



\section{CERN neutrino platform test environment [5 pages; {\color{red} David/Jack/Cheng-Ju/Thomas}]}
	\label{nuplatform}

%A description of construction of components, facility requirements, layout and constraints follows:

%We propose that the cryostat be housed in the extension of the EHN1 Bat 887 at CERN, where the cryogenic system components will also be located. (moved to sec 7)

%\begin{itemize}

%short description of%location and orientation of cryostat + cryogenics system in EHN1 (David)
%\item {\bf Cryostat construction:}
%The outer steel support structure reduces the time needed for the construction: the structure will be prefabricated in pieces of dimensions appropriate for transportation, shipped to the destination and only assembled in place. Fabrication will take place at the vendor's facility for the most part. This shortens the construction of the outer structure on the detector site, leaving more time for completion of the building infrastructure. If properly designed, a steel structure may allow the cryostat to be moved, should that be desired in the future.

%\item {\bf Cryogenics construction:}
%The outer steel support structure for the cryostat will be prefabricated in pieces of dimensions appropriate for transportation, shipped to the destination and only assembled in place.  Fabrication will take place at the vendor's facility for the most part. This shortens the construction of the outer structure on the detector site, leaving more time for completion of the building infrastructure. If properly designed, a steel structure may allow the cryostat to be moved, should that be desired in the future.

%\item {\bf Detector component construction:} 
%TPC and PDS detector components will be manufactured and submitted to quality assurance procedures at one or more of the potential DUNE detector component production sites. Successively, components will be shipped to CERN for further testing and final assembly into the cryostat. This approach will begin the preparations for the various production sites for the first 10~kt module of the DUNE far detector.

%\item {\bf PDS schedule items:}
%Once the technology has been chosen the PD group will focus on optimizing the selected design with the goal of procurement and assembly taking place in late FY 2016 and early FY 2016. The photon detector paddles will then be tested and shipped to CERN in early FY 2017 for installation into the APAs in late FY 2017 in preparation for installation into the test cryostat and operation in 2018. 

%\item Requirements for staging space, control room, electronics racks, clean room, scaffolding, etc. (Jack)

%\item power requirements and cooling 

%\item ...

%\end{itemize}

%\subsection{Installation}

The interior of the cryostat will be prepared prior to the installation of the TPC.  A series of I-beam support rails will be suspended below the top surface of the cryostat membrane by a series of hangers.  These hangers will be structurally supported by an independent structure above the cryostat.  Decoupling the TPC support from the cryostat structure eliminates the movement of the TPC with the flexure of the cryostat structure from the filling and internal pressure changes of the Argon inside.  The hangers will pass through the top of the cryostat to the independent structure inside a bellows type feedthrough.  These feedthroughs need to be designed to minimize the heat flow into the cryogenic volume.  For the CPAs, the support rails and hangers need to be electrically isolated due to high voltage concerns.  To preserve the ability to reverse the order of the TPC components, all of the support points will be designed to the maximum set of requirements regarding loads and clearances.  

There will be a series of feedthrough flanges located along each of the support rails.  These will be cryogenic flanges where the services for the TPC components can pass through the top of the cryostat.  It is foreseen that each CPA row will require one feedthrough for the high voltage probe to bring in the drift voltage.  The drift voltage is 500 V/cm.  For a drift distance of 3.6~m and 2.5~m, the probe voltages will be 180~kV and 125~kV respectively.  There will be one service feedthrough for each of the APAs.  These feedthroughs will include high speed data connections, bias voltages for the wire planes, control and power for the cold electronics.  

The main TPC components will be installed through large hatches in the top of the cryostat.  This is 
similar to the installation method intended for the detector at the DUNE far site.  These hatches will have an 
aperture approximately 2.0 m wide and 3.5 m long.  Each APA and CPA panel will be carefully tested after transport into the clean area and before installation into the cryostat. Immediately after a panel is installed it will be rechecked. The serial installation of the APAs along the rails means that removing and replacing one of the early panels in the row after others are installed would be very costly in effort and time. Therefore, to minimize the risk of damage, as much work around already installed panels as possible will be completed before proceeding with further panels.
The installation sequence is planned to proceed as follows:

\begin{enumerate}
\item Install the monorail or crane in the staging area outside the cryostat, near the equipment hatch.
\item Install the relay racks on the top of the cryostat and load with the DAQ and power supply crates.
\item Dress cables from the DAQ on the top of the cryostat to remote racks.
\item Construct the clean-room enclosure outside the cryostat hatch.
\item Install the raised-panel floor inside the cryostat. 
\item Insert and assemble the stair tower and scaffolding in the cryostat.
\item Install the staging platform at the hatch entrance into the cryostat.
\item Install protection on (or remove) existing cryogenics instrumentation in the cryostat.
\item Install the cryostat feedthroughs and dress cables inside the cryostat along the support beams.
\item Install TPC panels:
   \begin{enumerate}
   \item Install the CPA panels supported from the center support rail.  These will be installed from the floor of the cryostat.  Access to the top edge will be required by scaffolding.  
   \item Install and connect HV probe for the CPAs.
   \item Perform electrical tests on the connectivity of the probe to the CPAs.
   \item Install first end wall of vertical field cage at the non-access end of the cryostat.  These will be installed from the floor of the cryostat.  Scaffolding will be needed to install the supporting structure and then attach the panels to the structure.  
   \item Test the inner connections of the field cage panels.
   \item Install the first APA and connect to the far end field cage support.
   \item Connect power and signal cables.  This will require scaffolding to access the top edge of the APA.
   \item Test each APA wire for expected electronics noise. Spot-check electronics noise while cryogenics equipment is operating.
   \item Install the upper field cage panels for the first APA between the APA and CPAs  This will require scaffolding to access the upper edge of the APA, CPA and field cage structure.
   \item Perform electrical tests on upper field cage panels.
   \item Repeat steps (f) through (j) for the next five APAs.
   \item Install the lower field cage panels between the APAs and CPAs.  Start at the far end away from the access hatch and work towards the hatch. 
   \item Perform electrical test on lower field cage panels and the entire loop around the TPC.
   \item Remove temporary floor sections as the TPC installation progresses.
   \item Install sections of argon-distribution piping as the TPC installation progresses.
   \item Install the final end wall of vertical field cage at the access end of the cryostat.  These will be installed from the floor of the cryostat.  Scaffolding will be needed to install the supporting structure and then attach the panels to the structure.
   \end{enumerate}
\item Remove movable scaffold and stair towers.
\item Temporarily seal the cryostat and test all channels for expected electronics noise.
\item Seal the access hatch.
\item Perform final test of all channels for expected electronics noise.
\end{enumerate}
 
In general, APA panels will be installed in order starting with the panel furthest from the hatch side of the cryostat and progressing back towards the hatch. The upper field cage will be installed in stages as the installation of APAs and CPA progresses.  After the APAs are attached to the support rods the electrical connections will be made to electrical cables that were already dressed to the support beams and electrical testing will begin. Periodic electrical testing will continue to assure that nothing gets  damaged during the additional work around the installed APAs.  

The TPC installation will be performed in three stages, each in a separate location. First, in the clean room vestibule, a crew will move the APA and CPA panels from storage racks, rotate to the vertical position and move them into the cryostat. Secondly, in the panel-staging area immediately below the equipment hatch of the cryostat, 
% CHECK THIS! I TRIED TO INTERPRET RUSS'S COMMENTS. -- ANNE
a second crew will transfer the panels from the crane to the staging platform, where the crew inside the cryostat will connect the panels to the rails within the cryostat. A third crew will reposition the movable scaffolding and use the scaffold to make the mechanical and electrical connections at the top for each APA and CPA as they are moved into position.  

The requirements for alignment and survey of the TPC are under development. Since there are many cosmic rays in the surface detector and beam events, significant corrections can be made for any misalignment of the TPC. The current plan includes using a laser guide or optical transit and the adjustment features of the support rods for the TPC to align the top edges of the APAs in the TPC to be straight, level and parallel within a few millimeters. The alignment of the TPC in other dimensions will depend on the internal connecting features of the TPC.  The timing of the survey will depend on understanding when during the installation process the hanging TPC elements are in a dimensionally stable state. The required accuracy of the survey is not expected to be more precise than a few millimeters.  






\subsection{Installation}

The outer steel support structure for the cryostat will be prefabricated in pieces of dimensions appropriate for transportation, shipped to the destination and only assembled in place.  Fabrication will take place at the vendor's facility for the most part. This shortens the construction of the outer structure on the detector site, leaving more time for completion of the building infrastructure. If properly designed, a steel structure may allow the cryostat to be moved, should that be desired in the future.

The TPC and PDS detector components will be manufactured and submitted to quality assurance procedures at one or more of the potential DUNE detector component production sites. Successively, components will be shipped to CERN for further testing and final assembly into the cryostat. This approach will begin the preparations for the various production sites for the first 10~kt module of the DUNE far detector.

The interior of the cryostat will be prepared prior to the installation of the TPC.  A series of I-beam support rails will be suspended below the top surface of the cryostat membrane by a series of hangers.  These hangers will be structurally supported by an independent structure above the cryostat.  Decoupling the TPC support from the cryostat structure eliminates the movement of the TPC with the flexure of the cryostat structure from the filling and internal pressure changes of the Argon inside.  The hangers will pass through the top of the cryostat to the independent structure inside a bellows type feedthrough.  These feedthroughs need to be designed to minimize the heat flow into the cryogenic volume.  For the CPAs, the support rails and hangers need to be electrically isolated due to high voltage concerns.  To preserve the ability to reverse the order of the TPC components, all of the support points will be designed to the maximum set of requirements regarding loads and clearances.  

There will be a series of feedthrough flanges located along each of the support rails.  These will be cryogenic flanges where the services for the TPC components can pass through the top of the cryostat.  It is foreseen that each CPA row will require one feedthrough for the high voltage probe to bring in the drift voltage.  The drift voltage is 500 V/cm.  For a drift distance of 3.6~m and 2.5~m, the probe voltages will be 180~kV and 125~kV respectively.  There will be one service feedthrough for each of the APAs.  These feedthroughs will include high speed data connections, bias voltages for the wire planes, control and power for the cold electronics.  

The main TPC components will be installed through large hatches in the top of the cryostat.  This is 
similar to the installation method intended for the detector at the DUNE far site.  These hatches will have an 
aperture approximately 2.0 m wide and 3.5 m long.  Each APA and CPA panel will be carefully tested after transport into the clean area and before installation into the cryostat. Immediately after a panel is installed it will be rechecked. The serial installation of the APAs along the rails means that removing and replacing one of the early panels in the row after others are installed would be very costly in effort and time. Therefore, to minimize the risk of damage, as much work around already installed panels as possible will be completed before proceeding with further panels.
 
In general, APA panels will be installed in order starting with the panel furthest from the hatch side of the cryostat and progressing back towards the hatch. The upper field cage will be installed in stages as the installation of APAs and CPA progresses.  After the APAs are attached to the support rods the electrical connections will be made to electrical cables that were already dressed to the support beams and electrical testing will begin. Periodic electrical testing will continue to assure that nothing gets  damaged during the additional work around the installed APAs.  

The TPC installation will be performed in three stages, each in a separate location. First, in the clean room vestibule, a crew will move the APA and CPA panels from storage racks, rotate to the vertical position and move them into the cryostat. Secondly, in the panel-staging area immediately below the equipment hatch of the cryostat, 
% CHECK THIS! I TRIED TO INTERPRET RUSS'S COMMENTS. -- ANNE
a second crew will transfer the panels from the crane to the staging platform, where the crew inside the cryostat will connect the panels to the rails within the cryostat. A third crew will reposition the movable scaffolding and use the scaffold to make the mechanical and electrical connections at the top for each APA and CPA as they are moved into position.  

The requirements for alignment and survey of the TPC are under development. Since there are many cosmic rays in the surface detector and beam events, significant corrections can be made for any misalignment of the TPC. The current plan includes using a laser guide or optical transit and the adjustment features of the support rods for the TPC to align the top edges of the APAs in the TPC to be straight, level and parallel within a few millimeters. The alignment of the TPC in other dimensions will depend on the internal connecting features of the TPC.  The timing of the survey will depend on understanding when during the installation process the hanging TPC elements are in a dimensionally stable state. The required accuracy of the survey is not expected to be more precise than a few millimeters.  


\subsection{Infrastructure}

The inner detector and surrounding cryostat of the prototype detector will be located in a recessed pit in the floor of the extension of the EHN1 experimental hall.  However, additional space will be required for the installation and operation of the prototype.  This includes an unloading area, a clean room for assembly, and a control room to host computers to operate the detector components, readout the detector, and provide local storage of detector data.  Crane and forklift access will be required in the unloading zone and crane transport of detector components will be required between the clean room and recessed pit.  The control room will have to be accessible 24 hours a day and 7 days a week.

The experiment will rely on liquid nitrogen to provide cooling power for the argon condenser and the initial cool down of the vessel and the detector.  The area will have to be setup to receive regular tanker deliveries to a local dewar storage.  A distribution facility located in the experimental hall will be used to transfer the liquid nitrogen from the dewar system.  In addition, a liquid argon receiving facility which includes a storage dewar and an ambient vaporizer will be used to deliver liquid argon and gaseous argon to the cryostat.  Because of the risk of argon collecting in the pit, an exhaust pipe out of the pit will be required to vent the area.  The humidity and temperature in the experimental area will need to be controlled by air conditioning.   

The experimental hall will need to supply dedicated electrical power for the various components. These include:
the cryogenic system ($e.g.$ pumps), front-end electronics, high voltage systems, monitor systems, computers, $etc.$



	
	
\section{Schedule, organization and cost estimate [$\sim$5 pages; {\color{red} Thomas/Greg}]}
	\label{organ}
%insert organization, schedule and cost estimates here

\subsection{Schedule}
The major milestones for the proposed CERN prototype detector construction and beam test is dictated by the DUNE overall schedule which foresees to place the first 10~kton detector module underground as early as calendar year 2021. Additional detector modules, possibly of different design, are expected to follow at intervals of 1-2 years in a technically driven schedule.
Information and experience gained from manufacturing, installing and operating the CERN detector test will inform decision regarding future DUNE far detectors.
The LHC long shutdown, which is presently scheduled for mid-2018 represents a significant constraint on the beam data run schedule.
Cosmic muon data taking and ideally also beam data taking for the proposed measurement program 
should be acquired prior to  the long LHC shutdown in mid-2018. 
While it is desirable to have results from the charged particle beam run available as soon as possible the value of these results is not diminished if they become available only after installation of the DUNE far detector.
Figure \ref{fig:schedule} shows a technically limited schedule which meets the above requirements.
%is based on experiences from the production and installation schedule for the 35~t detector.
This schedule is based on experience of designing and manufacturing components for the 35t detector which will be commissioned starting in August 2015 at  Fermilab.
%
\begin{figure}[htb]
  \centering
\includegraphics[scale=0.34]{figures/150514_CERN_test_sched.png}
  \caption{Rolled up version of a draft schedule for manufacturing, installing and commissioning the CERN prototype detector. A 2 - 3 months data taking period is included in the schedule. }
  \label{fig:schedule}
\end{figure}
%
Pending an approval of the effort proposed in this document we foresee the following milestones
\begin{itemize}
\item 2016: Cryostat constructed
\item early 2016: TPC production readiness review
\item spring 2017: Engineering Trial Detector Assembly
\item early 2018: Detector Installation
\item spring 2018: Detector commissioning 
\end{itemize}


\subsection{Organization}

The CERN single phase detector effort is an integral part of the DUNE collaboration and the DUNE project structure. 
A working group has been set up within the DUNE collaboration to coordinate all aspects relevant for the CERN single phase prototype detector and beam test. 

The CERN prototype effort is coordinated by the "CERN Prototype Technical Coordinator (CTC)" \cite{LBNEorg}.
%The charge of the CERN prototype Collaboration Technical Coordinator is defined as follows.
The CERN prototype CTC is responsible for the following:
\begin{enumerate}[i]
	\item Prioritize and maintain a schedule of collaboration activities regarding simulations and physics analysis. 
	\item Prioritize and maintain a schedule of other collaboration service activities on the CERN prototype.  These activities may include collaboration participation in assembly, commissioning, calibration, debugging, data-taking, shifts, etc. 
	\item The responsibility of the CERN prototype CTC shall include maintaining a database of contributions from the collaboration to the CERN detector and beam test prototyping.
	\item The CERN prototype CTC shall recruit collaboration personnel and resources.
	\item The CERN prototype CTC shall coordinate with the DUNE far detector L-2 project manager and the DUNE technical coordinator
\end{enumerate}
There are currently one CTC,Thomas Kutter (LSU)  and a deputy, Greg Pawloski (Minnesota) sharing the responsibilities.\\
 

In order to effectively execute the tasks required for the CERN prototype detector and beam test several sub-groups have been created 
and sub-group leaders have been appointed.

{\bf Measurement Program + Analysis :}   Their charge is to develop a comprehensive and prioritized list of measurements required to evaluate detector performance and provide detector charged particle response as input into DUNE physics sensitivity studies (beam physics, nucleon decay, supernovae, atmospheric neutrinos).  And to perform simulation studies to quantitatively compare the relevance of various measurements.
The leaders are Donna Naples (Pittsburgh) and Jaroslaw Nowak (Lancaster).
 
 
{\bf Beam :} Their charge is to work closely with the measurement group to identify ideal beam requirements, to work closely with the CERN beam group to develop realistic beam design to make relevant beam measurements, to evaluate and optimize possible beam injection points and beam orientations, to perform beam simulations and provide simulated beam spectra as input for detector response simulations, to identify required beam instrumentation for beam characterization, and to develop beam run plans.
The  beam subgroup leader is Cheng-Ju Lin (LBNL).


{\bf Calibration :}  Their charge is to develop tools to calibrate the performance of the detector, to interface with the physics measurement group to prioritize different calibration measurements, and to interface with detector subcomponent working groups to identify all required calibration tools and their integration into the detector /cryostat design.
The calibration sub-group leaders are Qiuguang Liu (LANL) [interim] and Michele Weber (Bern).\\


The CERN prototype coordinators work closely with the DUNE technical coordinator and the DUNE far detector manager.
%Direct communication with  DUNE spokespeople is important for any strategic issues.
The CTC also coordinates closely with the CERN Neutrino Platform leader.
The leaders of the "measurement program + analysis" subgroup work closely with the DUNE physics and software tools working group leaders. The leader of the "beam" subgroup works closely with the relevant CERN beam-line group leader.

The project work break down structure (WBS) for the CERN single phase prototype detector down to level 3 and along with contributing institutions is shown in table~\ref{tab:wbs}.
%
\begin{table}[h]
\centering
\begin{tabular}{|l l c|}
\hline
\textbf{WBS no. } & \textbf{Description}  & \textbf{Contributing Institutes}  \\ \hline

X.1 & Single Phase LAr detector & \\
X.1.1 & TPC & Wisconsin \\
X.1.2 & Photon Detection System  &  ANL, CSU, IU, LSU \\
X.1.3 & Electronics  &   \\
X.1.4 &  DAQ & Oxford, SLAC  \\
X.1.5 & Installation  & Duke  \\
X.1.6 & ...  & ...  \\ \hline

X.2 & Experiment Infrastructure  &   \\
X.2.1 & Cryostat + Top Cap &  CERN, Fermilab \\
X.2.2 & Cryogenics System  &  CERN, Fermilab \\
X.2.3 &  Slow control ? &  CERN \\
X.2.4 &  Clean Room & CERN   \\ 
X.2.5 &  Logistics \& Integration & CERN  \\ 
X.2.6 &  Detector Commissioning \& operations &  all  \\ 
X.2.7 &  Beam \& Run Coordination &  LBNL, CERN \\ \hline

X.3 &  Safety ?? &   \\ \hline

X.4 & Scientific Effort &    \\ 
X.4.1 & Software \& Simulations &  all  \\
X.4.2 & Data analysis &  all  \\ \hline

\end{tabular}
\caption{Work break down structure down to level 3.}
\label{tab:wbs}
\end{table}
%
Work on a MOU between the CERN nu-platform and DUNE describing responsibilities, listing institutions involved and defining deliverables has started. 
%

\subsection{Cost estimate}
The total estimated cost of the CERN single phase LAr prototype detector is presented in table~\ref{tab:cost}.
Cost figures are informed by component production for the 35~t detector.

\begin{table}[h!]
\centering
\begin{tabular}{| l| l| c |}
\hline
\textbf{no. } & \textbf{Item}  & \textbf{Cost [kUSD]}  \\ \hline
1 & APAs & \\
2 & CPA  & \\
3 & PDS  & \\
4 & field cage  & \\
5 & TPC support & \\
6 & power supplies & \\
7 & Electronics & \\
8 & DAQ & \\
9 & computing & \\
10 & cryostat & 2300 \\
11 & cryogenics system & 1400 \\
12 & clean room & \\
13 & infrastructure & \\
14 & installation \& logistics & \\
15 & ... & \\ \hline
  & \textbf{Total } & \\ \hline
\end{tabular}
\caption{Estimated cost for the single phase LAr detector.}
\label{tab:cost}
\end{table}


%\subsection{Division of Responsibilities}

%\subsubsection{Shared responsibilities}

%The engineering design of the cryostat and the cryogenics system is considered to be a shared responsibility between DUNE/LBNF and CERN.


%\begin{itemize}

%\item plans for data analysis and publications:\\
%include: description of overlap/commonalities with WA105 data analysis and joined efforts
%\end{itemize}




%\subsubsection{CERN responsibilities}

%\paragraph{The beam line:} design, setup of the beam line and beam monitoring instrumentation are expected to be provided by CERN.

%\paragraph{The cryostat and cryogenics system:} are expected to be organized and paid for by the CERN nu-platform.
%The scope of the EHN1 cryostat subsystem includes the design, procurement, fabrication, testing, delivery and oversight of a cryostat to contain the liquid argon and the TPC.\\

%\paragraph{DAQ requests:}  Data links of sufficient bandwidth to transfer the data files from the CENF to the CERN data center, and from there to locations worldwide for analysis. \\

%\paragraph{Computing/Software support:} In order to leverage existing software and expertise, appropriate manpower will need to be allocated in order to create and maintain the computing infrastructure necessary for effective use of the reconstruction and physics analysis tools.\\





\section{Summary [$\sim$2 pages; {\color{red} Thomas/Greg}]}
	\label{summary}
The single-phase detector design that is proposed in this document has been chosen as the reference design for the first 10 kt far detector module of the DUNE experiment.  Although the single phase liquid argon technology has been successfully operated on a smaller scale, a liquid argon detector on the scale of the DUNE experiment has never been built before.  In order to mitigate risks, a CERN test of the single-phase prototype detector is a crucial milestone that will inform the construction and operation of the first 10 kt module.

A primary goal of the prototype is to perform an engineering test to validate the mechanical and electrical performance of the full-scale detector components to inform decisions that need to be made well in advance of the 2022 installation date of the first 10 kt module.  It would be of great value to also take test beam data before the long accelerator shutdown in 2018, to validate the physics performance of the detector to further inform these decisions.  However, data taken after the shutdown can still be instrumental in improving the final sensitivity of the DUNE experiment by understanding reconstruction parameters that affect the particle identification and energy reconstruction.   The experiment will have unprecedented sensitivity to CP violation in the lepton sector, the neutrino mass hierarchy, and proton decay.   

The CERN prototype represents the cumulation of years of R&D effort.  The technology, data acquisition, computing, and installation procedures are mature and have been tested on smaller scales.  There is a well established organization to implement the prototype program.  The SPSC had favorable comments on the initial proposal for the prototyping of a single-phase liquid argon detector for the
North Area facility, resulting in the request for the technical report that is presented in this document.


\begin{thebibliography}{99}
\bibitem{spade_icecube} IceCube Data Movement \url{https://icecube.wisc.edu/science/data/datamovement}.
\bibitem{spade_dayabay}Data processing and storage in the Daya Bay Reactor Antineutrino Experiment \url{http://arxiv.org/pdf/1501.06969.pdf}.
\bibitem{atlas_express} Prompt reconstruction of LHC collision data with the ATLAS reconstruction software, N.Barlow et al, \textit{Journal of Physics: Conference Series 331 (2011) 032004}

\bibitem{montanari_35ton} First scientific application of the membrane cryostat technology, D.Montanari et al, \textit{AIP Proceedings 1573, 1664 (2014)} \url{http://scitation.aip.org/content/aip/proceeding/aipcp/10.1063/1.4860907}

 \bibitem{acciarri_sbn_proposal} A Proposal for a Three Detector Short-Baseline Neutrino Oscillation Program in the Fermilab Booster Neutrino Beam \textit{R.Acciarri et al} \url{http://arxiv.org/abs/1503.01520}
 
% Add if doc can be made public in docdb -- AH 
% \bibitem{montanari_35ton_perf} Performance And Results of the LBNE 35 ton Membrane Cryostat Prototype
% \textit{D.Montanari et al, 25th International Cryogenic Engineering Conference and the International Cryogenic Materials Conference in 2014, ICEC 25–ICMC 2014, LBNE Docid 9270} 
 
\end{thebibliography}

{\color{red}
$\rightarrow$ total estimated page count: $\sim$60 pages}
\end{document}
