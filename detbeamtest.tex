
\subsection*{TODO list}
{\color{red}
\begin{description}
\item[Sensitivity plots.] 	
 Create plots for the CPV and HM sensitivities for 40kton detector and 1.2 MW beam. Plots should include various combinations of the assumed uncertainties: with current best measurements,  with best guesses about uncertainties we can archive and with  uncertainties which will be obtain using beam test experiment at CERN.
\end{description}

}

\subsection{Requirements for the detector, beam and commissioning}
\label{detbeam_main}
The Single-Phase Cern Prototype detector is intended to provide input necessary to reduce systematic uncertainties for oscillation measurements 
and additional measurements that support the physics program in DUNE. 
The LAr TPC technology is not new but wasn't extensively used in the 1-10 GeV neutrino energy range.  
The main source of uncertainties due to detector with the current values are shown in table \ref{table:deterr}


\begin{table}[h]
\centering
\caption{Current known sources of detector uncertainties for liquid argon or TPC.}
\label{table:deterr}
\begin{tabular}{|c|c|c|}
\hline
\textbf{source of uncertainty } & \textbf{value} & \textbf{reference}  \\ \hline
  e/$\gamma$ separation        &           &                   \\ \hline
  e-m shower calibration        &           &            \\ \hline
   hadronic shower calibration       &           &        \\ \hline
low energy acceptance electron identification &   &  \\ \hline
 .....   &   &   \\ \hline
\end{tabular}
\end{table}



\begin{table}[h]
\centering
\caption{Current known sources of  uncertainties due to interaction of charged particle with argon.}
\label{table:physicserr}
\begin{tabular}{|c|c|c|}
\hline
\textbf{source of uncertainty } & \textbf{value} & \textbf{reference}  \\ \hline
 pion(Kaon) absorbtion       &           &                   \\ \hline
 pion(Kaon) charge exchange       &           &            \\ \hline
pion (Kaon) production in secondary interactions  &   &  \\ \hline
 muon capture       &           &  Phys. Rev. C 35, 2212      \\ \hline
energy scale  &   &   \\ \hline
Michel electron tagging  &  &  \\ \hline

 .....   &   &   \\ \hline
\end{tabular}
\end{table}



With current detector uncertainties from table \ref{table:deterr} the sensitivities for the CP violation phase measurement is shown in Fig. \ref{fig:cpsensitivity}  {\bf Task: make this plot} . The  proposed test beam detector will reduce uncertainties to XX\%  and improve our sensitivity to $\delta_{CP}$ as shown in Fig. \ref{fig:cpsensitivity} {\bf Task: make this plot.}



\begin{figure}[h!]
  \centering
%\includegraphics[scale=0.4]{}
\label{fig:cpsensitivity}
  \caption{Sensitivites for the $\delta_{CP}$ measurement  for using current knowledge of the single-phase LAr-TPC detector technology and for reduced detector uncertainties from SPCP beamtest data.  The plots prepared for 40 kton fiducial mass and $xx\times 10^{21}$POT.}
\end{figure}

\newpage

\subsubsection{Particles energy and direction}
\label{detbeam_particles}
Plans for running beam for the the ELBNF include both neutrino and anti-neutrino configurations. These beams will be composed  mainly of muon neutrinos (anti-neutrinos) as well as electron neutrinos (anti-neutrinos). In figures \ref{fig:particle_momenta} and \ref{fig:particle_theta} the distributions on momenta and angles of particles created in neutrino interaction are shown. 

\begin{figure}[h!]
  \centering
\includegraphics[scale=0.4]{figures/True_Momenta_per_Particle}
\label{fig:particle_momenta}
  \caption{Particle momenta distributions for particles coming from all fluxes ($\nu_e$, $\nu_\mu$, $\bar \nu_e$ and $\bar \nu_\mu$) at both near and far detector locations.  }
\end{figure}


\begin{figure}[h!]
  \centering
%\includegraphics[scale=0.4]{figures/True_Theta_per_Particle}
\label{fig:particle_theta}
  \caption{Particle angle wrt to the beam axis distributions for particles coming from all fluxes ($\nu_e$, $\nu_\mu$, $\bar \nu_e$ and $\bar \nu_\mu$) at both near and far detector locations.  }
\end{figure}

\newpage


\begin{figure}[htp]
  \centering
  \label{fig:containment}
  
  \begin{tabular}{ccc}
%    \includegraphics[scale=0.15]{figures/electrons_density_overlay}&
%    \includegraphics[scale=0.15]{figures/electrons_lcont_overlay}&
%    \includegraphics[scale=0.15]{figures/electrons_wcont_overlay}\\
%  
%    \includegraphics[scale=0.15]{figures/photons_density_overlay}&
%    \includegraphics[scale=0.15]{figures/photons_lcont_overlay}&
%    \includegraphics[scale=0.15]{figures/photons_wcont_overlay}\\
%%
   \includegraphics[width=0.31\textwidth,height=3.5cm]{figures/protons_density_overlay}&
   \includegraphics[width=0.31\textwidth,height=3.5cm]{figures/protons_lcont_overlay}&
   \includegraphics[width=0.31\textwidth,height=3.5cm]{figures/protons_wcont_overlay}\\
%    \includegraphics[scale=0.15]{figures/protons_lcont_overlay}&
%    \includegraphics[scale=0.15]{figures/protons_wcont_overlay}\\
% 
%    \includegraphics[scale=0.15]{figures/pions_density_overlay}&
%    \includegraphics[scale=0.15]{figures/pions_lcont_overlay}&
%    \includegraphics[scale=0.15]{figures/pions_wcont_overlay}\\
 
   \includegraphics[width=0.31\textwidth,height=3.5cm]{figures/pions_density_overlay}&
   \includegraphics[width=0.31\textwidth,height=3.5cm]{figures/pions_lcont_overlay}&
   \includegraphics[width=0.31\textwidth,height=3.5cm]{figures/pions_wcont_overlay}\\
%    \includegraphics[scale=0.15]{figures/pions_density_overlay}&
%    \includegraphics[scale=0.15]{figures/pions_lcont_overlay}&
%    \includegraphics[scale=0.15]{figures/pions_wcont_overlay}\\
 
% 
%    \includegraphics[scale=0.15]{figures/kaons_density_overlay}&
%    \includegraphics[scale=0.15]{figures/kaons_lcont_overlay}&
%    \includegraphics[scale=0.15]{figures/kaons_wcont_overlay}\\
 
  \end{tabular}
  \caption{Particle containment plots.}
\end{figure}

\clearpage
%\subsubsection{Particle rates}
%\label{detbeam_rates}
%Estimation of  beam particles rates  necessary to collect high enough statistics in a reasonable time to obtain goals of of the measurements.
% THis should be discussed in the beam section
\subsubsection {Run plan}

Table~\ref{tab:runplan} summarizes the required exposures to beam particles.
\begin{table}[h]
\centering
\begin{tabular}{|c|c|c|l|}
\hline
Particle & Momenta (GeV/c) & Exposure & Purpose \\ \hline
$\pi^+$       & 0.2, 0.3, 0.4, 0.5, 0.7, 1, 2, 3, 5, 7     &  10K  & hadronic cal, $\pi^0$ content \\ \hline
$\pi^-$       &  0.2, 0.3, 0.4, 0.5, 0.7, 1     &  10K  & hadronic cal, $\pi^0$ content \\ \hline
$\pi^+$   &  2  &  600K & $\pi^o$/$\gamma$ sample \\ \hline
%$\pi^+$ &   1 \& 2  &  10K  & vary angle ($\times$5), reco \\ \hline
e$^+$ or e$^-$       &    0.2, 0.3, 0.4, 0.5, 1, 2, 3, 5, 7        &    10K   & e-$\gamma$ separation/EM shower     \\ \hline
% e$^+$ or e$^-$  &  1 \& 2  &  10K  & vary angle($\times$5), reco \\ \hline
%e$^+$ or e$^-$   (w/rad) &  3  &  20K  & tagged photons \\ \hline
$\mu^-$  &   (0.2), 0.5, 1, 2  &  10K & $E_\mu$, Michel el., charge sign \\ \hline
$\mu^+$ &   (0.2), 0.5, 1, 2   &  10K & $E_\mu$, Michel el.,charge sign  \\ \hline
$\mu^-$ or $\mu^+$ &   3, 5, 7  &  5K & $E_\mu$ MCS \\ \hline
%$\mu^-$ or $\mu^+$  &  1 \& 2  &  5K  & vary angle ($\times$5), reco \\ \hline
proton &  0.7, 1, 2, 3   &  10K & response, PID \\ \hline
proton &  1   &  1M & mis-ID pdk, recombination \\ \hline
%proton &  1 \& 2 &  10K & vary angle ($\times$5), reco \\ \hline
antiproton &  low-energy tune  &  (100) & antiproton stars \\ \hline
K$^+$  & 1 & (13K)   &   response, PID, PDK  \\ \hline
K$^+$  & 0.5, 0.7 & (5K)   &   response, PID, PDK  \\ \hline \hline
$\mu$, e, proton  & 1 (vary angle $\times$5 & 10K  & reconstruction  \\ \hline
\end{tabular}
\caption{Requirements summary for particle types and momenta. Items in parenthesis indicate lower priority (see text).
}
\label{tab:summaryphysics}
\end{table}



%Charged pion samples will be used to characterize hadronic shower response and to measure
%absorption cross section parameters on argon.
%for all energies except 0.1 GeV point). 
%Muon samples will be used for calibration and 
%reconstruction tests. Electron samples will be used to measure EM shower response  
%and to tune PID algorithms (statistics shown will far exceed 1\% uncertainty on the mean for 
%EM showers with resolution on the order of 1\% as measured in DOCDB 9434 and 8835. 100k Electron event samples will allow
%high statistics studies of e-gamma separation).  
%Proton response will be studied for reconstruction and to tune PID 
%algorithms. Kaon data will be needed to tune proton decay backgrounds .
%Special runs at various angles with $\pi$, $\mu$, p and electrons will be performed to study reconstruction and tune PID algorithms. 

\subsection{Detector performance tests}

\subsubsection{Bethe-Bloch parametrisation of charged particles}

The prototype detector will allow to study the detector response to charge particles from the test beam and will serve as a calibration detector. The measured energy deposition for various particles and its dependence on the direction of the particle will feed into our Monte Carlo generator and allow more precise reconstruction of neutrino energy and interactions topologies with good particle identifications.
 
{\bf How we compare with Lariat?} 
{\bf Multiple scattering}  

The set of single-phase prototype detector helped to understand the detector response to cosmic muons. But there is still lots to learn with additional studies. 
The charge  particle identification efficiencies  has been mapped for only limited range of the particle energies.  

%\begin{table}[h]
%\centering
%\begin{tabular}{|c|c|c|}
%\hline
%Particle     & Momenta (GeV)                                                                                       & Exposure/bin (total)  \\ \hline
%$\pi ^+ $   & 0.2-1.0 (50MeV bins), 1.0-3.0 ( 100MeV bins), 3.0-10 ( 200MeV bins)    &  200 (16.6k)     \\ \hline
%$\pi ^- $    & 0.2-1.0 (50MeV bins), 1.0-3.0 ( 100MeV bins), 3.0-10 ( 200MeV bins)    &  200 (16.6k)     \\ \hline
%$e^+$       & 0.2-10  (200MeV bins)                                                                              &  200 (9.8k)        \\ \hline
%$e^- $       & 0.2-10  (200MeV bins)                                                                              &  200 (9.8k)        \\ \hline
%$\mu^+$   & 0.2-1.0 (50MeV bins), 1.0-3.0 ( 100MeV bins), 3.0-10 ( 200MeV bins)    &  200 (16.6k)    \\ \hline
%$\mu^-$    & 0.2-1.0 (50MeV bins), 1.0-3.0 ( 100MeV bins), 3.0-10 ( 200MeV bins)    &  200 (16.6k)     \\ \hline
%$p$          &  0.2-1.5 (50MeV bins), 1.5-3.0 ( 100MeV bins), 3.0-10 ( 200MeV bins)    &  200 (17.2k)     \\ \hline
%$\bar p$   &  0.2-1.5 (50MeV bins), 1.5-3.0 ( 100MeV bins), 3.0-10 ( 200MeV bins)    &  200 (17.2k)    \\ \hline
%$K^+$      &  0.2-1.5 (50MeV bins), 1.5-3.0 ( 100MeV bins), 3.0-10 ( 200MeV bins)    &  200 (17.2k)      \\ \hline
%$K^- $      &  0.2-1.5 (50MeV bins), 1.5-3.0 ( 100MeV bins), 3.0-10 ( 200MeV bins)    &  200 (17.2k)    \\ \hline
%\end{tabular}\caption{Data sample requirements for the $dE/dx$ Bethe-Bloch parametrisation. The lower range starts with the nominal 200 MeV which is not achievable for p and difficult for K. The rough estimate is that we need about 200 particles per bin assuming that about half of them will be actually measured it gives  10\% statistical error for each bin but lower for the fit to the Bethe-Bloch formula. About half of the data sample is in the high momenta region, which can be reduce by decreasing number of particles per bin. }
%\end{table}
%
\subsubsection{e/$\gamma$ separation}

The search for a CP violation phase using $\nu_e$ appearance 
in a $\nu_\mu$ beam requires good electron/photon separation.
Backgrounds originating from photons produced primarily from 
final state $\pi^0$'s must be identified and removed from the signal
electron sample. 


The photons can undergo two process: pair production and Compton scattering. 
The dominant process for photons with energies of several hundreds MeV  is 
the e$^+$ e$^-$ pair production, but Compton scattering also occur at this 
energies. For pair production the e/$\gamma$ separation is achieved by looking 
at the beginning of the electromagnetic shower, where for election we see energy 
deposition typical for single MIP and for photon we see energy deposition consisted 
with two MIPs. In case of Compton scattering off of atomic electrons the 
signal is much more difficult to distinguish from the CC $\nu_e$ scattering signal.

Electron-photon separation has been studied in LAr TPCs
(Icarus and Argoneut) as shown in Fig.~\ref{fig:egam1}.
Currently the 
separation efficiency is estimated to be at the level of of 94 \% (? cite and 
check the number). 
This may depend on particular features of the geometry including wire pitch, etc.
Therefore, it is critically important 
to study e/$\gamma$ separation in a prototype LAr TPC detector.
{\bf we need someone to look into this}



\subsubsection{Reconstruction efficiencies and particle identification}
\label{detbeam_pid}

The reconstruction of events in the LAr TPC is still a challenge but rapid progress has been achieved in recent years (cite pandora and other reconstruction algorithms). Despite the progress reconstruction algorithms have to rely Monte Carlo predictions which don't simulate liquid argon detectors responses correctly. Reconstruction algorithms will benefit greatly from test beam data particularly from the full scale prototype. The reconstruction algorithms will be trained to correctly reconstruct track, electromagnetic and hadronic showers.
The data of tracks and showers can be used to create a library of reference events with which to tune algorithms.
%which can be used for matching with he neutrino data, similar to the  LEM (library event matching).

Main issues for the reconstruction algorithms:
\begin{itemize}
\item The reconstruction algorithms try to use all three planes on the signal readout. if the orientation of the track/shower is such that it is aligned with wires on one of the plans it significantly reduces quality of reconstructed objects. 
\item Calorimetry with collection and induction planes. In the ICARUS experiment the deposited energy was reconstructed from the signal on the collection plane. The induction planes bipolar signal wasn't "stable" enough to use it for calorimetric measurement. In the ELBNF design there is additional shielding  wire plane which will improve the quality of the bipolar signal and the  test beam experiment will help with its calibration.
\item   Vertexing.
\item Reconstruction efficiency for low energy particles. The reconstruction algorithm suffer from the lose of efficiency for low energy particle or particles which leave less than 200-300 hits. Training the algorithms on a low energy particles from the test beam will improve the quality and efficiency of the reconstructed objects.
\end{itemize}
%
%\begin{table}[h]
%\centering
%\begin{tabular}{|c|c|c|}
%\hline
%Particle     & Momenta (GeV)                                                                                       & Exposure/bin (total)  \\ \hline
%$\pi ^+ $   & 0.2-1.0 (100MeV bins), 1.0-10.0 ( 200MeV bins)    &  500 (26.5k)     \\ \hline
%$\pi ^- $    & 0.2-1.0 (100MeV bins), 1.0-10.0 ( 200MeV bins)    &  500 (26.5k)     \\ \hline
%$e^+$       & 0.2-10  (100MeV bins), 1.0-10.0 ( 200MeV bins)    &  500 (26.5k)       \\ \hline
%$e^- $       & 0.2-10  (100MeV bins), 1.0-10.0 ( 200MeV bins)    &  500 (26.5k)       \\ \hline
%$\mu^+$   & 0.2-1.0 (100MeV bins), 1.0-10.0 ( 200MeV bins)    &  500 (26.5k)     \\ \hline
%$\mu^-$    & 0.2-1.0 (100MeV bins), 1.0-10.0 ( 200MeV bins)    &  500 (26.5k)     \\ \hline
%$p$          &  0.2-1.5 (100MeV bins), 1.5-10.0 ( 200MeV bins)    &  500 (27.8k)     \\ \hline
%$\bar p$   &  0.2-1.5 (100MeV bins), 1.5-10.0 ( 200MeV bins)    &  500 (27.8k)     \\ \hline
%$K^+$      &  0.2-1.5 (100MeV bins), 1.5-10.0 ( 200MeV bins)    &  500 (27.8k)     \\ \hline
%$K^- $      &  0.2-1.5 (100MeV bins), 1.5-10.0 ( 200MeV bins)    &  500 (27.8k)     \\ \hline
%\end{tabular}\caption{Data sample requirements for the development of the reconstruction algorithms. The most important are  the low momenta particles where the showers are more likely to have different topologies. }
%\end{table}


\subsubsection{Cross section measurements}
Precise measurement of the  absorption and charge exchange of pions and kaons. Pion absorption is a large part of the pion nucleon cross section from 50 MeV to 500MeV with no data above about 1GeV pion kinetic energy. 
{\bf Add plots and values for known cross sections wit errors} 
\begin{itemize}
\item pion absorption on argon - Kotlinski, EPJ 9, 537 (2000)
\item pion cross section as a function of A - Gianelli PRC 61, 054615 (2000)
\end{itemize}
There is not currently a satisfactory theory describing absorption. The Valencia group (Vicente-Vacus NPA 568, 855 (1994)) developed model of    the pion-nucleus reaction with fairly good agreement, although not in detail. The actual  mechanism of multi-nucleon absorption
 is not well understood. 
 
\subsubsection{Charge sign determination}
It is not possible to determine charge of the particle on the event by event basis with non-magnetised LAr TPC detectors. A statistical separation will be studied which will make use of differences in muon versus antimuon capture cross sections and lifetime.
%However, the statistical analyst will be possible. We will fit the muon's half time which is different for muons and antimony due to different muon capture cross sections. 
For the $\mu^+$ for argon we expect about xx\% to be captured and for $\mu^-$ about yy\%. 

\subsubsection{Single track calibration}

\subsubsection{Shower calibration}

Reconstruction of neutrino energy depends of a quality of reconstruction of both electromagnetic and hadronic showers. 

- {\bf features of Hadronic shower in LAr TPC}
- {\bf features of electromagnetic shower in LAr TPC}
- Missing energy from neutral (Neutrons scattering)
% 
%\begin{table}[h]
%\centering
%\begin{tabular}{|c|c|c|}
%\hline
%Particle     & Momenta (GeV)                                                                                       & Exposure/bin (total)  \\ \hline
%$\pi ^+ $   & 0.2-1.0 (100MeV bins), 1.0-10.0 ( 200MeV bins)    &  1000 (48k)     \\ \hline
%$\pi ^- $    & 0.2-1.0 (100MeV bins), 1.0-10.0 ( 200MeV bins)    &  1000 (48k)     \\ \hline
%$e^+$       & 0.2-10  (100MeV bins), 1.0-10.0 ( 200MeV bins)    &  1000 (48k)       \\ \hline
%$e^- $       & 0.2-10  (100MeV bins), 1.0-10.0 ( 200MeV bins)    &  1000 (48k)       \\ \hline
%$\mu^+$   & 0.2-1.0 (100MeV bins), 1.0-10.0 ( 200MeV bins)    &  1000 (48k)     \\ \hline
%$\mu^-$    & 0.2-1.0 (100MeV bins), 1.0-10.0 ( 200MeV bins)    &  1000 (48k)     \\ \hline
%$p$          &  0.2-1.5 (100MeV bins), 1.5-10.0 ( 200MeV bins)    &  1000 (56k)     \\ \hline
%$\bar p$   &  0.2-1.5 (100MeV bins), 1.5-10.0 ( 200MeV bins)    &  1000 (56k)     \\ \hline
%$K^+$      &  0.2-1.5 (100MeV bins), 1.5-10.0 ( 200MeV bins)    &  1000 (56k)     \\ \hline
%$K^- $      &  0.2-1.5 (100MeV bins), 1.5-10.0 ( 200MeV bins)    &  1000 (56k)     \\ \hline
%\end{tabular}\caption{Data sample requirements for shower calibrations.  Currently about 1k particles is assumed per bin to include variations in the shower topologies. Details MC analysis is necessary. }
%\end{table}
%

\subsection{Other measurements} 
\subsubsection{Anti-proton annihilation }

A sample of antiproton would be useful to calibrate the $p$-$\overline{p}$ annihilation process. 
This would provide input to exotic B-violating process neutron oscillation (reference) modeling of 
subsequent  $n$-$\overline{n}$ annihilation. These events would be taggged in the mixed-mode beam.
Events should be at the lowest energies achievable in this beamline.


\subsubsection{Proton decay sensitivity and background samples}


The DUNE experiment in the deep underground location will improve sensitivity to detection of several modes of proton decay.
In particular, a first ever LAr detector of this scale underground will primarily improve sensitivity to 
proton decays with final state kaons such as  $p \rightarrow K^+ \overline{\nu}$. 
Sensitivity to this process is studied in \cite{bueno}. $K^+$ detector efficiencies are estimated to be $>$97\% in the
appropiate momentum range (500-800 MeV/c). The kaon samples requested in Table~\ref{pdktable} are needed to directly measure 
$K^+$ PID and detection efficiencies. Obtaining low energy kaons will likely be difficult in this beamline.
A sample of 13K beam kaons with 1~GeV/c momentum are requested to provide 2K stopping $K^+$ track samples for PID studies.
(only 15\% of $K^+$ at 1 GeV stop at 1~GeV/c).


\begin{table}[h]
\centering
\begin{tabular}{|c|c|c|}
\hline
Particle     & Momenta (GeV/c) & Exposure/bin  \\ \hline
\hline
K$^+$  &  1 & (13k)    \\ \hline
K$^+$  & 0.5, 0.7 & (5k)   \\ \hline
proton &  1  &  (1M)  \\ \hline
\end{tabular}\caption{Samples related to proton decay physics requirements.}
\label{pdktable}
\end{table}

A sizable sample of protons ($\sim 10^6$)
are requested to study the possible background contributions to  $p \rightarrow K^+ \overline{\nu}$.
This sample of  protons are needed to quantify the possibility that an interacting {\bf p} 
is  {\em mis-IDed as stopping K}. A proton interaction which produces neutrals and one charged pion 
(which is mis-IDed or subsequently decays to $\mu$) can fake the final state kaon signal.


\subsubsection{Supernova}
The energies of the electrons coming from CC $\nu_e$ interactions from Supernova will be in the order of 10s of MeV. 
The beam test cannot offer such low energy electron, but one can use the Michel electrons form $'mu$ decay to cover these energies. The SK used the Michel spectrum to calibrate the absolute energy scale. 


%\begin{table}[h]
%\centering
%\begin{tabular}{|c|c|c|}
%\hline
%Particle     & Momenta (GeV/c)    & Exposure/bin  \\ \hline
%$\mu^+$   & (0.2), 0.5, 1      &  10K    \\ \hline
%\end{tabular}\caption{Stopping . }
%\end{table}

