\label{appendix}
\section*{Appendix}
\subsection*{Cryostat}

The design parameters for the CERN prototype cryostat are listed in Table~\ref{tbl:cryogenics-design-parameters}.

\begin{table}[htpb]
\centering
\begin{tabular}{|p{.4\textwidth}|p{.5\textwidth}|}
\hline
\textbf{Design Parameter} & \textbf{Value} \\ \hline
Type of structure & Membrane cryostat \\ \hline
Membrane material    &  SS 304/304L, 316/316L or equivalent. \\ \hline
Fluid & Liquid argon (LAr)  \\ \hline
Other materials upon approval.\\ \hline
 Outside reinforcement (support structure)  &  Steel enclosure with metal liner to isolate the outside from the insulation space, standing on legs to allow for air circulation underneath. \\ \hline
 Total cryostat volume  &  538 m$^3$ \\ \hline
 Total LAr volume  &  483 m$^3$ \\ \hline
LAr total mass   & 673,000 kg  \\ \hline
Minimum inner dimensions (flat plate to flat plate).   &  7.8 m (W) x 8.9 m (L) x 8.1 0 (H) \\ \hline
Depth of LAr   &  7.2 m (0.82 m ullage, same as LBNF) \\ \hline
Primary membrane   &   1.2 mm thick SS 304L corrugated stainless steel\\ \hline
Secondary barrier system   &  GTT design; 0.07 mm thick aluminum between fiberglass cloth. Overall thickness 1 mm located between insulation layers.  \\ \hline
 Insulation  &  Polyurethane foam (0.9 m thick from preliminary calculations) \\ \hline
Maximum static heat leak   &  10 W/m$^2$ \\ \hline
LAr temperature   & 88 +/- 1K  \\ \hline
Operating gas pressure   &  Positive pressure. Nominally 70 mbarg ($\sim$1 psig) \\ \hline
 Vaccuum  &  No vacuum \\ \hline
 Design pressure  &  350 mbarg ($\sim$5 psig) + LAr head (1,025 mbarg) \\ \hline
Design temperature   &  77 K (liquid nitrogen temperature for flexibility) \\ \hline
Temperature of all surfaces in the ullage during operation   & \textless 100 K  \\ \hline
Leak tightness   & $10^{-6}$ mbar*l/sec   \\ \hline
Maximum noise/vibration/microphonics inside the cryostat   & LAr pump outside the cryostat  \\ \hline
Beam window   & Precise location TBD. Multiple beam windows installed on the upstream side of the cryostat  \\ \hline
 Accessibility after operations  & Capability to empty the cryostat in 30 days and access it in 60 days after the end of operations. \\ \hline
  Lifetime / Thermal cycles &  Consistent with liquid argon program. TBD. \\ \hline
 \end{tabular}
\caption{Design requirements for the membrane cryostat}
\label{tbl:cryogenics-design-parameters}
\end{table}

\textbf{Insulation system and secondary membrane: }
%
The membrane cryostat requires insulation applied to all internal surfaces of the outer support structure 
and roof in order to control the heat ingress and hence required refrigeration heat load. 
To avoid bubbling of the liquid argon inside the tank, the maximum static heat leak is 10 W/m$^2$ for the floor and the sides and 15 W/m$^2$ for the roof, higher to account for the penetrations that increase the heat budget. Preliminary calculations show that these values can be obtained using 0.9 m thick insulation panels of polyurethane foam.
Given an 
average thermal conductivity coefficient for the insulation material of 0.0283 W/(m$\cdot$K), the heat input 
from the surrounding steel is expected to be about 3.2~kW total. It assumes that the hatches are foam 
insulated as well. This is shown in Table~\ref{tbl:heat-load-calc}.

\begin{table}[htpb]
\centering
\begin{tabular}{|p{.15\textwidth}|p{.15\textwidth}|p{.15\textwidth}|p{.15\textwidth}|p{.15\textwidth}|}
\hline
 \textbf{Element} & \textbf{Area ($m^2$)}  &  \textbf{K ($W/mK$)} & \textbf{$\Delta$ T ($K$)}
 & \textbf{Heat Input ($W$)}\\ \hline
Base   & 83  & 0.0283   &205   & 534 \\ \hline
End walls  &  153 & 0.0283  &  205 &  986 \\ \hline
Side walls   & 172  & 0.0283  &  205 & 1,108 \\ \hline
Roof  &  83 & 0.0283  & 205  &  550\\ \hline
   &   &   &   &  \\ \hline
Total   &   &   &   & 3,162 \\ \hline
\end{tabular}
\caption{Heat load calculation for the membrane cryostat (insulation thickness=0.9~m). }
\label{tbl:heat-load-calc}
\end{table}


\textbf{Top cap:}
%
Table~\ref{tbl:cryostat-top-parameters} presents the list of the design parameters for the top of the cryostat.
%
\begin{table}[htpb]
\centering
\begin{tabular}{|p{.3\textwidth}|p{.7\textwidth}|} % AH {|p{.4\textwidth}|p{.5\textwidth}|}
\hline
 \textbf{Design Parameter} & \textbf{Value} \\ \hline
 Configuration &  Removable metal plate reinforced with trusses/I-beams anchored to the membrane cryostat support structure. Contains multiple penetrations of various sizes and a manhole. Number, location and size of the penetrations TBD. The hatches shall be designed to be removable. If welded, provisions shall be made to allow for removal and re-welding six (6) times.\\ \hline
Plate/Trusses non-wet material  &  Steel if room temperature.
SS 304/304 or equivalent if at cryogenic temperature
\\ \hline
Wet material  & SS 304/304L, 316/316L or equivalent. 
Other materials upon approval.
 \\ \hline
 Fluid & Liquid argon (LAr) \\ \hline
Design pressure  & 350 mbarg (~5 psig) \\ \hline
Design temperature  & 77 K (liquid nitrogen temperature for flexibility) \\ \hline
Inner dimensions  & To match the cryostat \\ \hline
Maximum allowable roof deflection  & 0.003 m (differential between APA and CPA) \\ \hline
Maximum static heat leak  & \textless 15 W/m$^2$  \\ \hline
 Temperatures of all surfaces in the ullage during operation & \textless 100 K \\ \hline
Additional design loads  &  -	Top self-weight \\
 & -	Live load (488 kg/m$^2$)\\
& -	Electronics racks (400 kg in the vicinity of the feed through)\\
& -	Services (150 kg on every feed through)
\\ \hline
TPC anchors  & %Capacity: AH
Number and location TBD. Minimum 6.
 \\ \hline
 Hatch opening for TPCs installation &  3.550 m x 2.000 m (location TBD)\\ \hline % , to . in numbers
Grounding plate  &  1.6 mm thick copper sheet brazed to the bottom of the top plate\\ \hline
Lifting fixtures  & Appropriate for positioning the top at the different parts that constitute it. \\ \hline
Penetrations  &  1 LAr In, 1 Purge GAr In, 1 Vent GAr In \\ 
& 2 Pressure Safety Valves, 2 Vacuum Safety Valves \\ 
& 1 GAr boil off to condenser \\ 
& 1-2 Liquid level sensors \\ 
& 1-2 Instrumentation \\ 
& 1 Temperature sensors feedthroughs? \\ 
& 1 LAr for cool down, 1 GAr for cool down \\ 
& 1 TPC signal, 3 TPC feedthroughs \\
& 1 Photon Detector for APA (Cold) \\
& Calibration dc\\ \hline
Lifetime / Thermal cycles  & Consistent with the liquid argon program TBD. \\ \hline
\end{tabular}
\caption{Design parameters for the top of the cryostat}
\label{tbl:cryostat-top-parameters}
\end{table}


