\label{singlephasedet}

\subsection{DUNE detector plans}

The far detector for the DUNE collaboration will be a series of four liquid argon time projection chambers (TPC), each in a cryostat that holds a fiducial/active/total LAr mass of 10.0/13.3/16.9~kt. The TPCs will be instrumented with photon detection. It is planned that the first 10~kt detector will be ready for installation in the 2022 timeframe. 
The design for the first 10~kt detector is a submerged wire plane-based TPC with electronic readout also in the liquid argon.  Designs of this style are referred to as single-phase detectors as the charge generation, drift, and detection all occurs in the liquid argon phase.  This style TPC features no charge amplification before collection, thereby making a very precise charge measurement possible. 


To achieve DUNE's goals, a detector is needed that is much larger than ICARUS, the largest LAr TPC detector built to date. The former long-baseline neutrino experiment (LBNE) developed a scalable far detector design shown in Fig.~\ref{fig:fardet-overview} that would scale-up LAr TPC technology by roughly a factor of 40 compared to the ICARUS T600 detector. To achieve this scale-up, a number of novel design elements need to be employed. A membrane cryostat typical for the liquefied natural gas industry will be used instead of a conventional evacuated cryostat.  The wire planes or anode plane assemblies (APAs) will be factory-built as planar modules that are then installed into the cryostat. The modular nature of the APAs allow the size of the detector to be scaled up to at least 40~kt fiducial mass. Both the analog and digital electronics will be mounted on the wire planes inside the cryostat in order to reduce the electronic noise, to avoid transporting analog signals large distances, and to reduce the number of cables that penetrate the cryostat. 

The scintillation photon detectors will employ light collection paddles to reduce the required photo-cathode area and thereby cost.  Designs being considered are also more compact than the photomuliplier tubes solution used elsewhere.

Many of the aspects of the design are being tested in a small scale prototype at Fermilab but given the very large scale of the detector elements a full-scale test is critical. 
Since the recent formation of the new DUNE collaboration a combined detector design team is emerging. 
Ideas from this new collaboration have the potential to modify the detector design for the additional three far detector modules
which are foreseen for DUNE.
The detector design described here is the LBNE detector design chosen by the DUNE collaboration as the reference design for the first 10~kt 
detector module and also adopted as the basis for the DUNE-PT.


\begin{figure}[!htb]
\centering
\begin{minipage}[b]{1.0\textwidth}
\begin{center}
\includegraphics[width=.75\textwidth]{figures/fardet-3D.png}
%\includegraphics[width=0.7\textwidth]{EndView-sketch.png}
\end{center}
\end{minipage}
\caption{\small 3D model of the design for the first DUNE single-phase detector. Shown is a 5~kt fiducial volume detector which would need to be lengthened for the 10~kt design. The present DUNE plan calls for the construction of four 10~kt detectors. }
\label{fig:fardet-overview} 
\end{figure}

The engineering goals of the single-phase APA/CPA detector test can be broken into five broad categories: 
\begin{enumerate}
	\item TPC performance, mechanical and electrical verification, 
	\item photon detection light yield verification,
	\item calibration strategy verification,
	\item argon contamination mitigation verification, 
	\item production and installation procedure verification
\end{enumerate}

	 The goals related to mechanical testing are to test the integrity of the detector. In the current design, each APA measures 2.3 m by 6.0 m and includes 2560 wires and associated readout channels. Given the complexity of these assemblies, a test where the detector can be thermally cycled and tested under operating conditions is highly advisable prior to mass production. The mechanical support of the APAs can be tested to verify that the mechanical design is reliable and will accommodate any necessary motion between the large wire planes. The impact of vibration isolation between the cryostat roof and the detector can also be tested. Finally, an improvement over existing cryostat designs is the possibility to move the pumps external to the main cryostat. This will reduce any mechanical coupling to the detector and also greatly improve both reliability and ease of repair.

	 
	 The electrical testing goals are to insure that the high voltage design is robust and that the required low electronic noise level can be achieved. As the detector scale increases so does the capacitance and the stored energy in the device. The design of the field cage and high voltage cathode planes needs to be such that HV discharge is unlikely and that if the event occurs no damage to the detector or cryostat results. The grounding and shielding of large detectors is also critical for low noise operation. By testing the full scale elements one insures that the grounding plan is fully developed and effective. Large scale tests of the resulting design will verify the electrical model of the detector. 

	 Research at Fermilab utilizing the Materials Test Stand~\cite{mat-test-stand} has shown that electronegative contamination to the ultra-pure argon from all materials tested is negligible if the material is immersed in the liquid argon. This implies that the dominant source of contamination originates from the gas ullage region and the room temperature connections to the detector. Careful design of the ullage region to insure that all surfaces and feedthroughs are cold is expected to greatly reduce the sources of contamination over what exists in present detectors. 
	 
\subsection{DUNE-PT}

%\subsubsection{Overview of the CERN Single-Phase test Detector}

This section presents the design details of a single-phase prototype detector based on the design by the former LBNE collaboration. 
The DUNE detector design is very modular and the DUNE-PT will be constructed from modular components of exactly the same design.

\begin{figure}[htb]
\centering
\begin{minipage}[b]{1.0\textwidth}
\begin{center}
\includegraphics[width=0.40\textwidth]{figures/CERN_single_TPC}
\includegraphics[width=.59\textwidth]{figures/TPC-3D-section.jpg}
\end{center}
\end{minipage}
\caption{\small 3D model of the CERN single-phase detector TPC (left) and inserted in the cryostat (right).}
\label{fig:CERNdet-overview}
\end{figure}

The TPC consists of alternating anode plane assemblies (APAs) and cathode plane assemblies (CPAs), with field-cage panels enclosing the four open sides between the anode and cathode planes.  Fig.~\ref{fig:CERNdet-overview} shows a sectioned view for the planned TPC 
by itself and inside the cryostat at CERN.  A uniform electric field is created in the volume between the anode and cathode planes. A charged particle traversing this volume leaves a trail of ionization. The electrons drift toward the anode plane, which is constructed from multiple layers of sense wires, inducing electric current signals in the front-end electronic circuits connected to the wires.

The TPC will be assembled from elements that are of the same size and materials as those planned for the first DUNE far detector module.  

The overall size of the TPC has been determined based on the desired particle containment in order to address the required physics measurements (see Section \ref{detbeamtest}). The TPC will have a 3-APA wide active volume and consists of two drift volumes with a drift length of 3.6~m each (see Fig.~\ref{fig:CERNdet-overview}).  
The APAs have an active (total) area measuring 2.29 m (2.32 m) wide and 5.9 m (6.2 m) high. The combination of the three APAs determines the overall TPC length to be 7.0~m. There will be a cathode plane (CPA) in the center between the two rows of APAs.  
The overall width of the TPC will be determined by a combination of the drift distances and the thicknesses of the two APA planes and the 
CPAs and amounts to 7.4~m.  
The overall height of the TPC is determined by the height of the APA which is 6.2~m.  The TPC dimensions are summarized in 
Table~\ref{table:TPC-dim}.
%
The minimum internal size of the cryostat is also indicated in Table \ref{table:TPC-dim} and was determined by adding the necessary mechanical and electrical clearances to the computed size of the TPC.  
 
\begin{table}[h]
\centering
\begin{tabular}{|c|c|}
\hline
\textbf{ Component } & dimensions [m]  \\ \hline \hline
APA  (active) &  $2.29 (wide) \times 5.9 (high)$ \\ \hline
APA  (external) &  $2.32 (wide) \times 6.2 (high)$ \\ \hline
TPC (active)       & $7.0 (long) \times 7.2 (wide) \times 5.9 (high)$  \\ \hline
TPC (external)       & $7.3 (long) \times 7.4 (wide) \times 6.2 (high)$  \\ \hline
cryostat (internal) &  $8.9 (long) \times 7.8 (wide) \times 8.1 (high)$  \\ \hline
\end{tabular}
\caption{Dimensions of DUNE-PT.}
\label{table:TPC-dim}
\end{table} 

Along with the APAs and CPAs, the TPC will include a field cage that surrounds the entire assembly to ensure a uniform drift field in the TPC's active volume. -

%This is a series of fiberglass I-beams for the structural elements.  These I-beams will be tiled with large copper sided FR4 panels to create the field cage.  Each panel will be connected with a series of resistors.  The field cage will also be connected to the CPAs through a capacitor assembly.

All of this will be supported by rows of I-beams supported from a mechanical structure above the cryostat.  The hangers for these I-beams will pass through the insulated top cap.  There will be a series of feedthrough flanges in the top cap of the cryostat to bring in and take out services for the TPC.  One HV feed-through is foreseen for the CPA row and one signal feed-through for each of the APAs.

The design also foresees the option to have the two APA rows mounted at 2.5~m from the central CPA each.
A reduced drift distance between the APA and CPA represents a deviation from the DUNE far detector design but is potentially very
useful in order to lessen the impact of space charge effects.
Due to the operation of DUNE-PT on the surface space charge 
 effects are expected to be larger than for underground operation at SURF.
We foresee to calibrate out any space charge effects for a 3.6~m drift distance using laser beam calibration and comics. 
At the same time we maintain the  possibility for a second test with reduced drift length if the uncertainties associated 
with our calibration limit the precision of measurements.
The cryostat would have to be emptied and the planes shifted to the 2.5~m drift distance.
 





%The plan is to have the CPA located in the center of the cryostat with APAs on each side near the walls of the cryostat membrane.  The above dimensions preserve the ability to reverse the order of the TPC rows by placing the CPA next to the wall of the cryostat and the APAs in the center.  However, this can only be done with the shorter 2.5m drift distance.  This reversed configuration at the 3.6m drift distance would place the CPAs too close to the membrane and risk high voltage discharge and and thereby possible damage to the membrane.  


% File from Bo Yu on the TPC component design
\input{APA-CP-FieldCage}

\subsubsection{Photon detection system}




The photon detection system (PDS) of the DUNE far detector will utilize liquid argon scintillation light to trigger on non-beam events and to
determine the prompt event time primarily for non-beam events but also for beam events. 
Timing information will be useful in determining the t$_0$ of cosmic rays, including those which 
overlap with beam events and events from radiological decays. This timing information allows proper spatial event reconstruction, including 
the reconstruction and rejection of background events.\\
%
While the TPC will have spatial resolution that is far superior to a photon detection system, there is no intrinsic precise determination of the event time and the drift time for TPC events can be up to milliseconds. The photon detection system can determine the start of an event occurring in the TPC volume (or entering the volume) to about 6 ns. For beam-triggered events the performance of the photon detection timing information can be crosschecked and evaluated during the detector beam test. In addition to providing trigger functionality, the PDS may be able to improve the event energy reconstruction by providing drift length dependent corrections to energy loss of the drifting electrons.
In the absence of an external electric field, a charged particle passing through liquid argon will produce about 44,000 photons ($\lambda$ = 128~nm) per MeV of deposited energy. 
At higher electric drift fields the number of photons will be smaller due to reduced recombination, but at 500 V/cm the yield is about 20,000 photons per MeV. Roughly one-third of the photons are prompt, 2-6~ns, and two-thirds are generated after a delay of 1.1-1.6~$\mu$s. LAr is highly transparent to the 128~nm VUV photons with a Rayleigh scattering length and absorption length of 55$\pm$5~cm \cite{rayleigh} and $>$200~cm \cite{absorption} respectively. The relatively large light yield makes the scintillation process an excellent candidate for triggering and  determination of t$_0$ for non-beam related events. Detection of the scintillation light may also be helpful in background rejection and possibly
provide improvements for energy reconstruction.

Several prototypes of photon detection systems for single phase liquid argon detectors have been developed by the former LBNE photon detector group over the past few years. There are currently three prototypes under consideration for use in the first module of the DUNE far detector, a baseline design along with two alternate designs. A decision on the design to be deployed in the CERN test will be made in late 2015. 
DUNE-PT provides the first full scale test of the photon detectors which will be fully integrated into a 
full scale TPC anode plane assembly. 

The present reference design for the photon detection system is based on acrylic bars that are 200~cm long and 7.6~cm wide, which are coated with a layer of tetraphenyl-butadiene (TPB). The wavelength shifter converts incoming VUV (128 nm) scintillation photons
%   with a conversion efficiency of ~50\% \cite{conversion-eff}
to longer wavelength photons which are characterized by an emission spectrum with a peak wavelength of 430~nm.
About 50\% of the converted photons will be emitted into the bar.
  A fraction of the wavelength-shifted optical photons are internally reflected to the bar's end where they are detected by SiPMs whose QE curve is well matched to the 430 nm wavelength-shifted photons. All PD prototypes are currently using SensL MicroFC-6K-35-SMT 6 mm $\times$ 6 mm devices \cite{sensl}. 

A full 6 m long APA will be divided into 5 bays with 2 PD modules (paddles) instrumenting each bay. The paddles will be inserted into the frames after the TPC wires have been strung allowing  final assembly at the CERN test location. Two alternative designs are also under consideration. 


One alternate design attempts to increase the geometrical acceptance of the photon detectors by using large acrylic TPB coated plates with imbedded WLS fibers for readout. In this design the number of required SiPMs and readout channels per unit detector area covered with photon detection panels would be significantly reduced to keep the overall cost for the photon detection system at or below the present design while increasing the geometrical acceptance at the same time. The prototype consists of a TPB-coated acrylic panel embedded with a S-shaped wavelength shifting (WLS) fiber. The fiber is read out by two SiPMs, coupled to either end of the fiber, and serves to transport the light over long distances with minimal attenuation. The double-ended fiber readout has the added benefit to provide some position dependence to the light generation along the panel by comparing relative signal sizes and arrival times in the two SiPMs. 



The third design under consideration was motivated by increasing the attenuation length of the PD paddles and allowing collection of 400 nm photons coming from anywhere in the active volume of the TPC.  The fiber-bundle design is based on a thin TPB coated acrylic radiator located in front of a close packed array of WLS fibers. This concept is designed so that roughly half of the photons converted in the radiator are incident on the bundle of fibers, the wavelength shifting fibers are Y11 UV/blue with a 4\% capture probability. The fibers are then read out using SiPMs at one end. The Y11  Kuraray fibers have mean absorption and emission wavelengths of about 440 nm and 480 nm respectively.  The attenuation length of the Y11 fibers is given to be greater than 3.5 m at the mean emission wavelength, which will allow production of full-scale (2 m length) photon detector paddles.


The PD system tested at the CERN neutrino platform will be based on technology selected later in 2015. The technology selection process will be based on a series of tests planned for 6 months utilizing large research cryostats at Fermilab and Colorado State University. The primary metric used for comparison between the three technologies will be photon yield per unit cost. In addition to this metric, PD threshold and reliability will also serve as inputs to the final decision. A technical panel will be assembled to make an unbiased decision. 



\subsubsection{TPC and PDS readout}
\input{readout}

\subsubsection{DAQ, Slow control and monitoring}
\input{daq}




