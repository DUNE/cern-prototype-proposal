%\subsection{Calibration}

The accuracy of track and kinematics reconstruction and particle identification largely relies on the knowledge of electric field map, purity map, and temperature. The electric field, designed to be 500 V/cm, could vary at different locations due to sagged wires, misalignment, imperfect field cage, etc. It could also be distorted by the space charge. The electrons and ions generated by the passing high energy particles drift in opposite directions to anode and cathode planes, respectively. The electrons drift at about 1.6 mm/$\mu s$ and take 2.25 ms to travel the maximum distance between APAs and CPAs. The ions, drifting at $\sim$cm/s, can take up to 6 minutes to travel the same distance. Constant cosmic ray passing the detector imposes an even bigger space-charge effect to the detector. Distortion of the electric field could result in cm of uncertainty in position reconstruction and can also make a difference in electron-ion recombination and light output. The temperature of liquid argon affects the electron drift velocity by about -1.9\%/K. Precision measurement of temperature could be easily achieved with commercial silicon diode sensors down to tens of mK. Purity of argon directly affects the amount of surviving electrons, which would be used to estimate the amount of energy deposited in that wire space. This energy deposition, dE/dx, is a key quantity for particle identification. Monitoring argon purity is also essential for the operation of the experiment.

Multiple means should be employed to measure the electric field, the purity of liquid argon and its temperature. Here is a list of major calibration equipment.
\begin{itemize}
\item Gas purity analyzers
\item Liquid purity monitors
\item Temperature sensors
\item Laser calibration system
\item Muon detector system
\end{itemize}

The gas purity analyzers for oxygen and H$_{2}$O is commercially available. These analyzers measure purity down to a few parts per billion (ppb). They are not directly measuring the liquid argon purity in the detector but can be installed outside of the detector and take samples from various points of the entire cryogenic system. They provide an overview of the entire detector system.

Liquid purity monitor is a small size TPC with light sources that generates electrons via photoelectric effect and electronics to read out the amount of electrons in the cathode plane and anode plane. Electron lifetime could be calculated by comparing the survived electrons and the generated electrons, therefore, to estimate the argon purity. Multiple liquid purity monitors should be installed at locations that are close to and far from the recirculation inlet and at different height.

The temperature of liquid argon would vary by the pressure of the detector by a little less than 1 K/psi. It would also vary by elevation in the liquid. Silicon diode sensors with accuracy down to $\sim$ 20 mK should be installed at multiple locations in the detector.

To measure and calibrate electric field, purity, electron lifetime, drifting speed in-situ, a laser system and a muon detector are used. Both systems use ionization-particle paths to perform the mentioned physics quantities and have their pros and cons.

The laser calibration system employs a high power ultra-violet (UV) laser to ionize the liquid argon. The 266 nm UV photons have energy of 4.66 eV. Three photons could excite one argon molecule (ionization potential for liquid argon is 13.78 eV). Laser is directed in the TPC region via a steerable feedthrough, which allows reflecting laser to various areas. The laser energy is at about 10 mJ, which is 10$^{16}$ UV photons, and has sufficient energy to produce a straight (long Rayleigh scattering length) and uniform ionization path. Due to the size of the laser beam, about 1 cm, the electrons are generated in a relative large space, and the electron-ion recombination effect is negligible compares to the cosmic ray induced ionization. In the proposed configuration of TPC with a CPA (or APA) in the center and two APAs (or CPAs) in the side, the laser feedthrough will be installed outside of the TPC region and in the same plane to the central wire plane assembly. Therefore the laser can be directed to both CPA-APA regions. The field cage is designed with multiple slits to allow laser passing to the inside of the TPC. Position detecting systems such as SiPMs will be installed on the other side of the field cage to measure the position of the laser.

The cosmic detector system serves as an alternative tool to the laser calibration system. Cosmic ray with energy of a few GeV has a uniform dE/dx in liquid argon at about 2 MeV/cm and generates about 18,000 electrons in the 3 mm wire space. In the TPC region, about 20 cosmic muons are expected in every event. Muon detectors are preferred to be installed to measure cosmic ray passing horizontally to the cryostat but with a small angle to allow sufficient statistics. A disadvantage of using muon to using laser is that the muon would be scattered multiple times in passing the more than 6 meters tall liquid argon and will not be straight anymore.
